%%% Lecture 4

%\section{Spazi di Moore}
%
%
%\textbf{COSE sta ripetendo la costruzione di ieri.}
%
%Costruiamo un CW complesso $X$ di dimensione $k+1$,
%il cui scheletro è formato da
%\begin{itemize}
%	\item una $0$-cella $\ast$;
%	\item le $k$-celle date da $\bigvee_{G_{0}} S^{k}$;
%	\item le $(k+1)$-celle avranno come bordo le $k$-celle sopra.
%\end{itemize}
%
%\textcolor{red}{Riscrivere questa parte nella lezione di ieri per aggiustare.}

\section{Torri di Whitehead}

\lecture[Torri di Whitehead. Fattorizzazione di Moore-Postnikov. Unicità spazi di Eilenberg-MacLane CW complessi. Funtorialità della costruzione degli spazi di Eilenberg-MacLane. Definizione della classe fondamentale di un $K(\pi,n)$. Teorema di rappresentabilità della coomologia. Esempi.]{2023-03-07}

Fissiamo un punto base $x \in X$ e consideriamo la fibra omotopica $F$
dell' $n$-sezione di Postnikov $X \to \tau_{\le n} X$.
Dato che i due spazi hanno la stessa omotopia fino al
grado $n$ e per $q > n$ si ha $\pi_{q}(\tau_{\le n} X,\ast) = 0$,
allora la successione esatta lunga in omotopia è
\begin{equation*}
	\begin{tikzcd}[column sep=small]
		\dots \ar[r] & \pi_{n+1}(F,\ast) \ar[r, "\sim"]
		& \pi_{n+1}(X,x) \ar[r]
		% & \pi_{q+1}(\tau_{\le n}X, \ast) \ar[r]
		& 0 \ar[r]
		& \pi_{n}(F,\ast) \ar[r]
		& \pi_{n}(X,x) \ar[r, "\sim"]
		& \pi_{n}(\tau_{\le n} X,\ast) \ar[r]
		& \dots
	\end{tikzcd}
\end{equation*}
quindi deduciamo che la fibra ha omotopia
\begin{equation*}
	\pi_{q}(F,\ast) =
	\begin{cases}
		0\,, \quad & \text{se } q \le n\,; \\
		\pi_{q}(X,x)\,, \quad & \text{se } q > n\,.
	\end{cases}
\end{equation*}
Siccome la situazione in omotopia è l'opposto di quello che succede
con le sezioni di Postnikov, denotiamo la fibra $\tau_{\ge n} X := F$.

\begin{ex}
	La fibra $\tau_{\ge 1}X$ è la componente connessa per archi del punto base $x \in X$,
	se $X \to \pi_{0}(X,x)$ è continua.\todo{Che significa?}
\end{ex}

\begin{ex}
	La sezione $\tau_{\ge 2}X$ è uno spazio semplicemente  connesso,
	con la stessa omotopia superiore di $X$: deduciamo che è il
	rivestimento univesale di $X$ (nel caso in cui $X$ sia connesso per archi
	e ammette rivestimento universale).
\end{ex}

\begin{exercise}
	La mappa $\tau_{\ge n+1} X \to X$ è unica a meno di omotopia \emph{puntata}
	ed è la \textbf{mappa terminale} nella categoria $\textbf{HoTop}_{\ast}$
	delle mappe da spazi $n$-connessi in $X$, i.e. se $Y$
	è uno spazio con omotopia banale fino al grado $n$, 
	allora ogni mappa continua $Y \to X$ fattorizza attraverso $\tau_{\ge n+1}X$.
	\begin{equation*}
		\begin{tikzcd}
			Y \ar[dr] \ar[d, dashed, "\exists !"] & \\
			\tau_{\ge n} X \ar[r] & X \,.
		\end{tikzcd}
	\end{equation*}
\end{exercise}

Partendo da $Y=X$ e ripetendo la costruzione induttivamente,
otteniamo il diagramma
\begin{equation}\label{TW}
	\begin{tikzcd}
		\dots \ar[d] & \\	
		\tau_{\ge 2}X \ar[d, "t^{2}"'] \ar[ddr] & \\
		\tau_{\ge 1}X \ar[d] \ar[dr] & \\
		\tau_{\ge 0}X  \ar[r, equals] 
		& X ,,
	\end{tikzcd}\tag{TW}
\end{equation}
chiamato \textbf{torre di Whitehead}.
In modo analogo a quanto accade per le torri di Postnikov,
si verifica che la fibra di $t^{2}$
è uno spazio di Eilenberg-MacLane.

Esiste una costruzione più generale, 
che enunciamo nel seguente teorema (che non dimostriamo).
\begin{thm}[\textbf{Fattorizzazione di Moore-Postnikov}]
	Data una mappa $f:X \to Y$ tra spazi topologici connessi per archi,
	esiste una successione di spazi $T_{0}f, T_{1}f,\dots, T_{\infty}f$
	e un diagramma commutativo
	\begin{equation*}
		\begin{tikzcd}
			X \ar[rrrr, "f"] \ar[d, "\iota_{\infty}"'] 
			\ar[rrd, "\iota_{n}"', shorten=2ex]
			& & & & Y \\
			T_{\infty} f \ar[r] 
			& \dots \ar[r]
			& T_{n} \ar[r] \ar[rru,"s_{n}", shorten=2ex]
			& \dots \ar[r]
			& T_{0} f \ar[u,"s^{0}"'] \,,
		\end{tikzcd}
	\end{equation*}
	tale che $\pi_{q}(\iota_{n})$ sia un isomorfismo per ogni $q<n$
	e $\pi_{q}(s_{n})$ sia un isomorfismo per ogni $q \le n$.
	\begin{proof}
		Vedi \parencite[hatcher]{Theorem~4.71}
	\end{proof}
\end{thm}


\section{Rappresentabilità della coomologia}

	Adesso che conosciamo la costruzione degli spazi $K(\pi,n)$,
	possiamo vedere che questi spazi hanno una natura ``algebrica'',
	che ci permettono di studiare la coomologia di spazi topologici.
	
	
	\begin{lemma}
		Sia $n > 0$ intero e $(Y,y)$ uno spazio puntato tale che $\pi_{q}(Y,y)=0$
		per $q \ne n$. Posto $G := \pi_{n}(Y,y)$, sia $M = K(\pi,n)$.
		La mappa indotta in omotopia
		\begin{equation*}
			\pi_{n} : [M,Y]_{\ast} \longrightarrow \Hom_{\cat{Grp}}(\pi,G)
		\end{equation*}
		è un isomorfismo. In particolare, se $\pi=G$, segue che $Y$
		è omotopicamente equivalente a $M$.
		\begin{proof}
			Sia $F_{0} \to F_{1} \to \pi \to 0$ una risoluzione libera di gruppi abeliani,
			e consideriamo la corrispondente successione di cofibrazione
			\begin{equation*}
				\bigvee_{I_{1}} S^{n} \longrightarrow
				\bigvee_{I_{0}} S^{n} \longrightarrow
				M\,.
			\end{equation*}
			Se denotiamo con $X$ il cono della mappa di incollamento celle
			\begin{equation*}
				X := C\left( \bigvee_{I_{1}} S^{k} \to \bigvee_{I_{0}} S^{k} \right)\,,
			\end{equation*}
			allora per la dimostrazione della \hyperref[CW-est]{Proposizione~\ref{CW-est}}\todo{Aiuto, è giusto?}
			sappiamo che $\tau_{\le n} X \simeq X$; dato che la sezione
			di Postnikov $X \to M$ è iniziale tra le mappe da $X$
			agli spazi $Y$ con omotopia nulla sopra $n$,
			questo significa che $[M,Y]_{\ast} \to [X,Y]_{\ast}$ è un isomorfismo.
			Quindi ci basta mostrare che
			\begin{equation*}
				\widetilde{\pi_{n}} : [X,Y]_{\ast} \longrightarrow \Hom_{\cat{Grp}}(\pi,G)
			\end{equation*}
			è un isomorfismo.
			Siccome $X$ è definito da una successione di cofibrazione,
			quindi una successione co-esatta, applicando $[-,Y]_{\ast}$
			otteniamo la successione esatta di insiemi puntati
			\begin{equation*}
				\begin{tikzcd}
					\left[ \bigvee_{I_{1}} S^{n},  Y\right]_{\ast} \longleftarrow
					\left[ \bigvee_{I_{0}} S^{n}, Y \right]_{\ast} \longleftarrow
					\left[ M, Y \right]_{\ast} \longleftarrow
					\left[ \bigvee_{I_{1}} S^{n+1},  Y\right]_{\ast}\,,
				\end{tikzcd}
			\end{equation*}
			dove abbiamo usato che la sospensione commuta con fare il bouquet:
			\begin{equation*}
				\Sigma \bigvee S^{k} \simeq \bigvee S^{k+1}\,.
			\end{equation*}
			Riscrivendo $[S^{n},Y]_{\ast} = \pi_{n}(Y,y) \simeq \Hom(\Z,G)$,
			si ha
			\begin{equation*}
				\left[\bigvee_{I_{i}} S^{n},Y \right]_{\ast} \simeq \Hom(F_{i},G)\,,
			\end{equation*}
			quindi la successione esatta di sopra diventa quindi
			\begin{equation*}
				\begin{tikzcd}
					\Hom(F_{1},G) 
					& \Hom(F_{0},G) \ar[l, "f"']
					& {[X,Y]_{\ast}} \ar[l]
					& 0 \,, \ar[l] 
				\end{tikzcd}
			\end{equation*}
			dove $\ker f = \Hom(\pi,G)$.
			
			In particolare, se $\pi = G$ possiamo considerare l'unica classe
			in $[M,Y]_{\ast}$ corrispondente all'identità $\cat{1}_{G} \in \Hom(G,G)$.
			Questa mappa è un'equivalenza omotopica debole,
			dunque se $Y$ è un CW complesso deduciamo che è un'equivalenza omotopica.
		\end{proof}
	\end{lemma}

	Quindi due CW complessi $K(\pi,n)$ sono omotopicamente equivalenti tramite
	una mappa \emph{unica} (\emph{a meno di omotopia}) che induce l'identità su $\pi_{n}$.
	\begin{cor}
		Per $n > 0$ esiste un funtore
		\begin{equation*}
			K(-,n) : \Ab \longmapsto \cat{HoCW}_{\ast}\,,
			\quad \pi \longmapsto K(\pi,n)
		\end{equation*}
		che è unico a meno di isomorfismo.
		Inoltre, per $n=1$ questo funtore si estende a 
		\begin{equation*}
			K(-,1) : \Grp \longmapsto \cat{HoCW}_{\ast}\,.
		\end{equation*}
	\end{cor}
	
	\begin{oss}
		Grazie ai funtori $K(-,n)$ possiamo definire, in maniera puramente
		\emph{topologica}, dei nuovi invarianti per gruppi, 
		rispettivamente l'\textbf{omologia} e la \textbf{coomologia di gruppi}
		a coefficienti in $G$:
		\begin{align*}
			\pi \longmapsto K(\pi,n) \longmapsto H_{*}(K(\pi,n);G)\,,
			\quad \pi \longmapsto K(\pi,n) \longmapsto H^{*}(K(\pi,n);G)\,.
		\end{align*}
	\end{oss}
	
	\begin{ex}
		La coomologia di $\Z/2$ a coefficienti in $\Z/2$ è
		\begin{equation*}
			\Z/2 \longmapsto H^{\ast}(\R\PP^{\infty}; \Z/2) \simeq \Z/2[x]\,.
		\end{equation*}
	\end{ex}
	
	Dato $n>0$ e $Y$ uno spazio $(n-1)$-connesso,
	il \textbf{Teorema dei Coefficienti Universali} ci dice
	che, per un gruppo abeliano $G$, si ha la successione esatta corta
	\begin{equation*}
		\begin{tikzcd}
			0 \ar[r]
			& \Ext(H_{q-1}(Y);G) \ar[r]
			& H^{q}(Y;G) \ar[r]
			& \Hom(H_{q}(Y);G) \ar[r]
			& 0\,
		\end{tikzcd}
	\end{equation*}
	 per ogni $q>0$, e quindi la coomologia $H^{q}(Y;G) = 0$ per $q < n$.
	 Quando $q=n$, si annulla il termine $\Ext$ e quindi la coomologia
	 è proprio il duale
	 \begin{equation*}
	 	H^{n}(Y;G) \simeq \Hom(H_{q}(Y),G)\,.
	 \end{equation*}
	Se $G=\pi_{n}(Y,y)$, allora l'inverso dell'isomorfismo di Hurewicz
	appartiene a $\Hom(H_{n}(Y),G)$, dunque corrisponde a una classe
	$\iota_{n} \in H^{n}(Y;G)$.
	\begin{df}
		Per $Y = K(\pi,n)$, otteniamo una classe canonica
		\begin{equation*}
			\iota_{n} \in H^{n}\big(K(\pi,n);\pi\big)
		\end{equation*}
		chiamata \textbf{classe fondamentale}.
	\end{df}
	La classe $\iota_{n}$ ci fornisce una trasformazione naturale
	\begin{equation}\label{classe-k-pi-n}
		\rho: [X,K(\pi,n)]_{\ast} \longrightarrow H^{n}(X;\pi)
		\quad [f] \longmapsto f^{*}\iota_{n}\,.
	\end{equation}
	che permette di dare una visione più \emph{geometrica} ai gruppi di coomologia:
	infatti, nel corso di Elementi di Topologia Algebrica abbiamo ottenuto $H^{n}$
	tramite un processo \emph{algebrico astratto}, 
	dualizzando il complesso di catene singolari di uno spazio $X$;
	invece, adesso possiamo interpretare le classi di coomologia
	come di uno spazio $X$ come mappe (a meno di omotopia) 
	a valori in un opportuno spazio di Eilenberg-MacLane.
	
	\begin{thm}\label{coh-rep}
		Se $X$ ha il tipo di omotopia di un CW complesso,
		allora $\rho$ è un isomorfismo.
		\begin{proof}
			Fissato $\pi$, consideriamo degli spazi di Eilenberg-MacLane
			$K(\pi,n)$ per $n \ge 0$. Per prima cosa capiamo
			la struttura di gruppo che esiste su membro di sinistra: %$[X,K(\pi,n)]_{\ast}$:
			se $n=0$, allora $K(\pi,0)=\pi$ con la topologia discreta, dunque
			\begin{equation*}
				[X,K(\pi,0)]_{\ast} = \Set{f:X \to \pi | f \text{ è localmente costante} }
				= H^{0}(X;\pi)\,.
			\end{equation*}
			
			C'è una mappa naturale $K(\pi,n) \to \Omega K(\pi,n+1)$
			che è isomorfismo per $\pi_{n}$, quindi mi dà un nuovo modello per $K(\pi,n)$,
			così come anche $K(\pi,n) \to\Omega^{2} K(\pi, n+2)$.
			Si deduce che $[-,K(\pi,n)]_{\ast}$ è un funtore a valori in $\Ab$
			perché
			\begin{equation*}
				\left[X,\Omega^{2}K(\pi,n+2)\right]_{\ast} \simeq 
				\left[\Sigma^{2} X, K(\pi,n+2)\right]_{\ast}\,.
			\end{equation*}
			Mostriamo che la mappa $\rho: [X,K(\pi,n)]_{\ast} \to \widetilde{H}^{n}(X;\pi)$ 
			è un omomorfismo di gruppi: 
			usando lo ``strozzamento'' $\Sigma X \to \Sigma X \vee \Sigma X$
			per costruire la somma
		\begin{equation*}
			\widetilde{H}^{n+1}(\Sigma X;\pi) \times \widetilde{H}^{n+1}(\Sigma X;\pi)
			\longrightarrow \widetilde{H}^{n+1}(\Sigma X \vee \Sigma X;\pi)
			\longrightarrow \widetilde{H}^{n}(\Sigma X;\pi)\,,
		\end{equation*}
		questo induce l'operazione di gruppo su $\widetilde{H}^{n}(\Sigma X;\pi)$.
		Analogamente, applicando $[-,K(\pi,n)]_{\ast}$ otteniamo un gruppo dato che
		\begin{equation*}
			[X,K(\pi,n)]_{\ast} = [X,\Omega K(\pi,n+1)]_{\ast} = [\Sigma X,K(\pi,n+1)]_{\ast}
		\end{equation*}
		e lo strozzamento della sospensione di $X$ induce la somma come prima.
		
		Mostriamo che $\rho$ induce un isomorfismo per induzione sullo scheletro di $X$.
		Se $X=X^{(0)}$, allora
		\begin{align*}
			\Hom_{\cat{HoTop}_{\ast}}(X^{(0)},\pi) &= \widetilde{H}^{0}\left(X^{(0)};\pi\right)\,, \\
			\left[ X^{(0)}, K(\pi,n) \right] = 0 
			&= \widetilde{H}^{n}\left(X^{(0)};\pi\right)\,, \quad \text{per } n>0\,.
		\end{align*}
		e questo conclude il passo base. 
		
		Se assumiamo $X$ connesso di dimensione $q$ finita, c'è una successione di cofibrazione
		\begin{equation*}
			\bigvee_{} S^{q-1} \longrightarrow X^{(q-1)} \longrightarrow X^{(q)}
		\end{equation*}
		che è co-esatta, quindi $[-,K(\pi,n)]_{\ast}$ induce una successione esatta in 
		coomologia ridotta. Usando il passo induttivo,
		$\rho$ è un isomorfismo per $\bigvee S^{q-1}$ e per $X^{(q-1)}$,
		che hanno dimensione al più $q-1$, quindi usando
		il \textbf{Lemma dei 5} scopriamo che è isomorfismo anche per $X^{(q)}$.
		
		Se $X$ ha dimensione infinita, sfruttiamo un argomento di limite
		e il fatto che, per ogni $N > q \ge 0$, abbiamo il diagramma commutativo
		\begin{equation*}
			\begin{tikzcd}
				{\left[ X^{(N)}, K(\pi,n) \right] } \ar[r, "\simeq", "\rho"'] \ar[d, "\simeq"]
				& {\widetilde{H}^{q}\left(X^{(N)};\pi\right)} \ar[d, "\simeq"] \\
				{\left[ X^{}, K(\pi,n) \right] } \ar[r, dashed]
				& \widetilde{H}^{q}(X;\pi)\,.
			\end{tikzcd}
		\end{equation*}
		\end{proof}
	\end{thm}
	
	\begin{ex}
		Dato che $K(\Z,1) = S^{1}$, allora $H^{1}(X;\Z) = [X,S^{1}]_{\ast}$.
	\end{ex}
	
	\begin{ex}
		Dato $X$ uno spazio topologico, vediamo che la sua coomologia
		\begin{equation*}
			H^{1}(X;\Z/2) = [X;\R\PP^{\infty}]_{\ast}\,.
		\end{equation*}
		Vedremo più avanti che una mappa $X \to \R\PP^{\infty}$
		induce un \emph{fibrato in rette}, e mappe omotope
		inducono fibrati isomorfi; deduciamo quindi che 
		$H^{1}(X;\Z/2)$ parametrizza le classi d'equivalenza di fibrati in rette su $X$.
	\end{ex}






