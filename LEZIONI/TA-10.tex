%%% Lezione 10


\lecture[Lezione 10.]{2023-03-28}

\begin{df}
Sia $xi = (E,p)$ un fibrato universale per $G$ sula base $B$.
Dato un $G$-fibrato principale $\xi'= (E',p')$ su $X$,
chiamiamo \textbf{mappa classificante} di $\xi'$
la mappa $f:X \to B$ tale che $\xi' \simeq f^{*}\xi$.
\end{df}

\begin{oss}
	I fibrati universali sono \emph{unici} nel seguente senso:	
	dati $\xi, \xi'$ due $G$-fibrati universali le cui basi sono
	CW complessi, allora $B$ e $B'$ sono omotopicamente equivalenti.
	Infatti, per universalità esistono due mappe 
	$f:B \to B'$ e $g:B' \to B$ tali che $f^{*}\xi' \simeq \xi$
	e $ g^{*}\xi \simeq \xi'$.  Per funtorialità del pullback
	segue che $(g \circ f)^{*}\xi = f^{*}(g^{*}\xi) = f^{*} \xi' = \xi$
	e quindi per il \hyperref[teo-classificazione]{Teorema di classificazione}
	deduciamo che $g \circ f \simeq \cat{1}_{B}$;
	analogamente si vede che $f \circ g \simeq \cat{1}_{B'}$.
\end{oss}

Vogliamo trovare una caratterizzazione dei fibrati universali.

\begin{thm}
	Un $G$-fibrato principale $\xi = (E,p)$ su $B$ è universale
	se e solo se $E$ è connesso per archi e debolmente contraibile.
	\begin{proof}
		Sia $\xi$ è universale. Per ogni $n \ge 0$,
		data una mappa $f:S^{n} \to B$ dove $S^{n}=\de D^{n+1}$,
		vogliamo dimostrare che $f$ si estende a tutto il disco.
		Definiamo la mappa
		\begin{equation*}
			\widetilde{f} : S^{n} \times G \longrightarrow E\,,
			\quad \widetilde{f}(x,g) := f(x) \cdot g\,,
		\end{equation*}
		notando che è una bundle map che copre $f$.
		Per ipotesi, $\widetilde{f}$ si estende a una bundle map 
		$\widetilde{f}':D^{n+1} \times G \to E$; componendo con la proiezione,
		consideriamo dunque $f':D^{n+1} \to E$ l'applicazione indotta su $D^{n+1}$
		da $\widetilde{f}'$. Per costruzione, si vede che $f'$ estende $f$,
		quindi $\pi_{i}(E,\ast) = 0$ per ogni $i \ge 0$.
		
		Viceversa, ora supponiamo che una qualsiasi bundle map
		$h:S^{n} \times G \to E$ si estende a tutto $D^{n+1} \times G \to E$.
		Data una mappa $h_{1}:S^{n} \to E$ definita da $h_{1}(x) = h(x,1)$,
		per ipotesi si estende a $\widetilde{h}_{1}:D^{n+1} \to E$.
		Ma allora possiamo estendere la mappa a $D^{n+1} \times G$
		facendo agire il gruppo sulla destra, cioè
		\begin{equation*}
			\widetilde{h} : D^{n+1} \times G \longrightarrow E\,,
			\quad \widetilde{h}(x,g) := \widetilde{h}_{1}(x) \cdot g\,.
		\end{equation*}
		Per costruzione, questa mappa estende $h$ a $S^{n} \times G$ poiché
		\begin{equation*}
			h(x,g) = h(x,1 \cdot g) = h(x,1) \cdot g = \widetilde{h}_{1}(x) \cdot g\,.
		\end{equation*}
		Sia ora $X$ un CW complesso e $L \subset X$ un suo sottocomplesso.
		Consideriamo $\pi':E' \to X$ un $G$-fibrato principale e
		supponiamo sia data una bundle map $h:(p')^{-1}(L) \to E$.
		Allora riusciamo a estenderla a tutto $E'$ per induzione sullo scheletro di $X$:
		per prima cosa estendiamo $h$ a $L \cup X^{(0)}$. Se $v \in X^{(0)} \setminus L$,
		allora mando $E'_{v}$ su una qualunque fibra $E_{w}$, con $w \in B$.
		Questo estende $h$ a una bundle map $h_{0}:p^{-1}(L \cup X^{(0)}) \to E$.
		Per induzione, suppongo di aver esteso $h$ a $p^{-1}(L \cup X^{(n-1)})$
		e costruisco l'estensione su $p^{-1}(L \cup X^{(n)})$.
		Se $c:D^{n} \to X^{(n)}$ è la mappa caratteristica di una $n$-cella di $X^{(n)}$,
		con $c':\de D^{n} = S^{n-1} \to X^{n-1}$ funzione di attaccamento,
		anche la restrizione a $\de D^{n}$ è banale.
		Siccome per induzione abbiamo già costruito la bundle map
		$h': E(c^{*}\xi)\vert_{S^{n-1}} : S^{n-1} \times G \to E$,
		allora per il passo precedente si estende a $D^{n} \times G$.
	\end{proof}
\end{thm} 


Ricordiamo che nella categoria dei CW complessi, l'equivalenza debole
corrisponde all'equivalenza omotopica, quindi lo spazio totale
di un $G$-fibrato principale è contraibile.
Un $G$-CW-complesso libero e contraibile si indica con $EG$
e il suo spazio delle orbite $BG := EG/G$.
La mappa di proiezione $p:EG \to BG$ è un $G$-fibrato universale
e la base $BG$ si dice \textbf{spazio classificante} del gruppo $G$.


Il problema è come costruire $BG$.
Quando $G$ è discreto, notiamo che $BG$ è uno spazio
di Eilenberg-MacLane di tipo $K(G,1)$:
la fibra è discreta e $\xi$ è un rivestimento con fibre omeomorfe a $G$.
Ricordiamo che $p_{*} : \pi_{i}(E,\ast) \to \pi_{i}(B,\ast)$
è in \emph{isomorfismo} per ogni $i \ge 2$, pertanto
\begin{equation*}
	\pi_{i}(B,\ast) \simeq 
	\begin{cases}
		0\,, \quad &\text{se }i \ne 1\,; \\
		G\,, \quad &\text{se } i = 1\,.
	\end{cases}
\end{equation*}
In breve, per CW complessi, $BG$ ha $\pi_{1}(B,\ast) \simeq G$ e rivestimento universale contraibile.

\begin{ex}
	\begin{rmnumerate}
		\item Se $G=\Z$, abbiamo visto che $K(\Z,1) \simeq S^{1}$.
		
		\item Identificando $G = \Z/2$ con $S^{0}$, la filtrazione
		\begin{equation*}
			\begin{tikzcd}
				S^{0} \ar[r, hook]
				& S^{1} \ar[r, hook] \ar[d]
				& S^{2} \ar[r, hook] \ar[d]
				& \dots \ar[r, hook]
				& S^{\infty} \ar[d] \\
				& \R\PP^{1} \ar[r, hook]
				& \R\PP^{2} \ar[r, hook]
				& \dots
				& \R\PP^{\infty}
			\end{tikzcd}
		\end{equation*}
		mostra che $S^{\infty}$ è un CW complesso,
		dunque la mappa $S^{\infty} \to \R\PP^{\infty}$ è un fibrato universale su $\Z/2$,
		dato che $S^{\infty}$ è contraibile.
		
		\begin{fact}
			Se $G$ è un gruppo dotato di torsione, allora il suo spazio classificante
			è necessariamente un CW complesso di dimensione infinita.
			Questo fatto ha una motivazione profonda,
			ed è legato all'\textbf{omologia del gruppo} $G$.
			Infatti, dato un gruppo, si può definire un invariante 
			$H_{*}(G)$ in termini puramente algebrici:
			 nel caso di gruppi di torsione, esistono infiniti
			 gruppi di omologia $H_{n}(G)$ non banali.
			Ad esempio
			\begin{equation*}
				H_{2n-1}(\Z/r) \simeq \Z/r \ne 0\,.
			\end{equation*}
			D'altra parte, ammesso di sapere che per ogni gruppo $G$
			esiste lo spazio classificante $BG$,
			allora potremmo definire $H_{*}(G) := H_{*}(BG;\Z)$.
			Queste due nozioni coincidono, e questo permette
			di studiare i gruppi da un punto di vista topologico.
			Per uni'introduzione, si veda .
		\end{fact}
		
		\item Se $G = S^{1} \subset \C$, considero la filtrazione
		\begin{equation*}
			S^{1} \subset S^{3} \subset S^{5} \subset \dots 
			\subset S^{2n-1} \subset S^{2n+1} \subset
			\dots \subset S^{\infty}\,,
		\end{equation*}
		dove $S^{2n-1} \subset \C^{n}$ è la sfera unitaria immersa in uno spazio vettoriale complesso.
		L'azione di $G$ sulle sfere è per moltiplicazione, quindi deduciamo che
		per ogni $i \ge 1$ si ha $S^{2i-1} / S^{1} \simeq \C\PP^{i}$.
		Dalla filtrazione sopra deduciamo che 
		$S^{1} \hookrightarrow S^{\infty}  \to \C\PP^{\infty}$ 
		è un fibrato principale con spazio totale contraibile,
		quindi è universale.
		Si noti che $S^{1}$ \emph{non} è un gruppo discreto e 
		quindi lo spazio classificante $BG$ \emph{non} è necessariamente 
		di tipo $K(G,1)$.
	\end{rmnumerate}
\end{ex}

\begin{oss}
	Se $G$ agisce liberamente su uno spazio contraibile $E$,
	e $H \subset G$ è un gruppo che agisce ancora liberamente su $E$,
	allora $E \to E/H = BH$ è un $H$-fibrato universale.
\end{oss}

\begin{thm}[\textbf{Peter-Weyl}]\label{Peter-Weyl}
	Ogni gruppo di Lie compatto ha una rappresentazione fedele finito-dimensionale
	e unitaria (i.e. $U_{n} \subset \mathrm{GL}(n,\C)$).
\end{thm}

Alla luce di questo teorema, per studiare gli spazi di classificazione
di gruppi di Lie compatti è sufficiente studiare
il caso delle matrici unitarie $U(n)$. Infatti,
se $G \hookrightarrow U(n)$, allora $EG \simeq E\,U(n)$
e lo spazio classificante si otterrà come quoziente $BG \simeq E\,U(n) / G$.
Vogliamo dunque trovare un modello conveniente per lo spazio classificante di $U(n)$.

\begin{lemma}
	Sia $B$ compatto e di Hausdorff. Ogni fibrato vettoriale $\xi = (E,p)$
	ha un morfismo iniettivo in un fibrato banale.
	\begin{proof}
		Consideriamo un ricoprimento banalizzante $U_{1}, \dots, U_{k}$
		e consideriamo $\{ u_{i} \}$ una partizione dell'unità associata.
		Sia 
		\begin{equation*}
			\Phi_{i} : p^{-1}(U_{i}) \longrightarrow U_{i} \times \R^{n}\,,
			\quad \Phi_{i}(e) = (p(e), \phi_{i}(e))\,,
		\end{equation*}
		dove $\phi_{i}$ induce isomorfismi sulle fibre. Definiamo
		\begin{equation*}
			\Phi : E \longrightarrow \left( \R^{n} \right)^{k} \simeq \R^{nk}\,,
			\quad \Phi(e) = \Big( u_{1}(p(e)) \phi_{1}(e), \dots, u_{k}(p(e)) \phi_{k}(e) \Big)
		\end{equation*}
		e questa ci permette di trovare il seguente morfismo in un fibrato banale:
		\begin{equation*}
			E \longrightarrow B \times \R^{nk}\,,
			\quad e \longmapsto \Big( p(e), \Phi(e) \Big)\,.
		\end{equation*}
	\end{proof}
\end{lemma}

\begin{cor}
	Se $B$ è uno spazio compatto, ogni fibrato vettoriale $\xi$
	ha un complemento ortogonale $\xi^{\perp}$ e la somma
	di Whitney $\xi \oplus \xi^{\perp}$ è banale.
	\begin{proof}
		Dato un fibrato banale $\eta$ in cui $\xi$ si immerge,
		è sufficiente dotare $\eta$ di una metrica Riemanniana e 
		considerare l'ortogonale fibra per fibra.
	\end{proof}
\end{cor}


\begin{oss}
	Se $\xi$ ha rango $n$, allora $\phi(E_{b})$ è un sottospazio di dimensione $n$
	in qualche $\R^{nk}$, quindi un elemento $\phi^{n}(\R^{nk})$.
	Troviamo così una bundle map
	\begin{equation*}
		\begin{tikzcd}
			E \ar[d, "p"] \ar[r, "\phi'"] 
			& E(\gamma_{nk,n}) \ar[d] \\
			B \ar[r, "\phi"] & Gr_{n}(\R^{nk})\,,
		\end{tikzcd}
	\end{equation*}
	quindi $\xi = f^{*}\gamma_{nk,n}$.
	
	Se vogliamo classificare tutti i fibrati vettoriali di rango $n$ su $B$,
	dato che $\R^{nk}$ ha dimensione arbitrariamente grande,
	allora è necessario passare alla grassmaniana su $\R^{\infty}$,
	dove ricordiamo che è dato dalla filtrazione
	\begin{equation*}
		\R \subset \R^{2} \subset \dots \subset \R^{n} \subset \dots \subset \R^{\infty}\,,
	\end{equation*}
	e quindi possiamo vederlo come l'insieme delle successioni
	$(x_{1}, x_{2}, \dots)$ definitivamente nulle.
	Indichiamo allora con $Gr_{n}(\R^{\infty})$ l'insieme dei sottospazi
	$n$-dimensionali in $\R^{\infty}$.
	Generalizzando la mappa di Gauss che abbiamo nel caso di varietà differenziabili,
	per fibrati vettoriali di rango $n$ su una base $B$ paracompatta
	si guarda $E(\gamma_{n}) \to Gr_{n}(\R^{\infty})$ è un fibrato universale.
	Per una discussione dettagliata, vedi \parencite[$\S 5$]{mi-sta}.
\end{oss}

\begin{ex}
	Detta $\Vv_{n}(\C^{n+h})$ la varietà di Stiefel degli $n$-frames ortonormali in $\C^{n+h}$,
	la proiezione $p: \Vv_{n}(\C^{n+h}) \to Gr_{n}(\C^{n+h})$ è un fibrato principale,
	con fibre $U(n)$. Notiamo che $U(q)$ agisce su $S^{2n-1} \subset \C^{n}$,
	con stabilizzatore $U(q-1)$, quindi $U(q)/U(q-1) \simeq S^{2q-1}$.
	Usando il fatto che, dati sottogruppi $H \subset K \subset G$, il quoziente $G/H \to G/K$
	è un fibrato con fibre $K/H$, allora otteniamo il seguente diagramma
	\begin{equation*}
		\begin{tikzcd}
			U(h+1)/U(h) \simeq S^{2h+1} \ar[r, hook] \ar[d] & U(n+h)/U(h) \ar[d] \ar[r, "\simeq"] & \Vv_{n}(\C^{n+h}) \ar[d] \\
			U(h+2)/U(h+1) \simeq S^{2h+3} \ar[r, hook] \ar[d] & U(n+h)/U(h+1) \ar[d] \ar[r, "\simeq"] & \Vv_{n-1}(\C^{n+h}) \ar[d] \\
			U(h+3)/U(h+2) \simeq S^{2h+5} \ar[r, hook] \ar[d] & U(n+h)/U(h+2) \ar[d] \ar[r, "\simeq"] & \Vv_{n-2}(\C^{n+h}) \ar[d] \\
			\vdots & \vdots \ar[d] & \vdots \ar[d] \\
			& U(n+h)/U(n+h-1) \ar[r, "\simeq"] & \Vv_{1}(\C^{n+h}) \simeq S^{2(n+h)-1}\,.
		\end{tikzcd}
	\end{equation*}
	Usando la successione esatta dei fibrati, si deduce che $\Vv_{n}(\C^{n+h})$
	è uno spazio $2n$-connesso.
	Quindi, per trovare il fibrato universale facciamo il \emph{limite} su $n$
	e si ottiene $$\Vv_{n}(\C^{\infty}) \to Gr_{n}(\C^{\infty})\,.$$
%	\begin{equation*}
%		\begin{tikzcd}
%			\Vv_{n}(\C^{\infty}) \ar[d] \\
%			Gr_{n}(\C^{\infty})\,.
%		\end{tikzcd}
%	\end{equation*}
	
	Si può studiare analogamente il caso \emph{reale} sostituendo alle matrici
	unitarie $U(n)$ i gruppi ortogonali $O(n)$.
\end{ex}
