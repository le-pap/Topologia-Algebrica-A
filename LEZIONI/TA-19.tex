%%% Lezione 19

\lecture[Lez.19.]{2023-05-15}

\section{Sospensione}

Più volte durante il corso abbiamo usato la classica fibrazione
\begin{equation*}
	\Omega X \hookrightarrow PX \longrightarrow X\,.
\end{equation*}
Infatti, questa nasce considerando i funtori tra
spazi topologici puntati
\begin{equation*}
	\begin{tikzcd}
		\cat{Top}_{*} \ar[r, "\Sigma"] & \cat{Top}_{*} \ar[l, "\Omega"]
	\end{tikzcd}
\end{equation*}
che sono tra loro aggiunti: più precisamente, dotando
gli spazi di funzioni della topologia compatto-aperta,
per ogni coppia di spazi topologici $X,Y$ abbiamo degli 
\emph{omeomorfismi} naturali
\begin{equation*}
	hom(\Sigma X, Y) \simeq hom(X, \Omega Y)\,,
\end{equation*}
dove i morfismi di aggiunzione sono dati esplicitamente da
\begin{align*}
	\sigma_{X} : X \longrightarrow \Omega \Sigma X\,, \quad
	ev_{Y} : \Sigma \Omega Y \longrightarrow Y\,, \\
	\sigma_{X}(x)(t) := [x,t] \in \Sigma X\,, \quad
	ev_{Y}(\omega, t) := \omega(t)\,. 
\end{align*}

\begin{prop}
	Dato uno spazio $X$ connesso per archi e semplicemente connesso,
	sia $\{E_{*,*}^{n}\}_{n \ge 2}$ la successione spettrale
	associata alla fibrazione $\Omega X \hookrightarrow PX \xrightarrow{\pi} X$.
	L'omomorfismo di trasgressione
	\begin{equation*}
		H_{n}(X) = E^{2}_{n,0} \supset E_{n,0}^{n}
		\overset{\longrightarrow}{d^{n}} E^{n}_{0,n-1} 
		\longleftarrow E_{0,n-1}^{2} = H_{n-1}(\Omega X)
	\end{equation*}
	è indotto dall'inversa della composizione
	\begin{equation*}
		\overline{H}_{n-1}(\Omega X) \simeq \overline{H}_{n}(\Sigma \Omega X)
		\longrightarrow H_{n}(X)\,.
	\end{equation*}
	\begin{proof}
		Considero la mappa\todo{specificare che C indica i cammini liberi}
		\begin{equation*}
			\phi : C \Omega X \longrightarrow P X\,, \quad
			\omega_{t}(u) := \omega(tu)\,,
		\end{equation*}
		cioè $\Omega X$ è contraibile in $PX$.
		Considero allora la valutazione
		\begin{equation*}
			ev' : C \Omega X \longrightarrow X\,, \quad
			ev'(t, \omega) := \omega(t)\,.
		\end{equation*}
		Abbiamo quindi il diagramma:
		\missingfigure{Digramma pazzo sgravato}
		che mostra che $ev_{*} \circ \Sigma$ è l'inversa della trasgressione
		$\de_{*} \circ \pi^{-1}_{*} : H_{n}(X) \to H_{n-1}(\Omega X)$.
	\end{proof}
\end{prop}

\begin{oss}
	Possiamo interpretare la valutazione $ev : \Sigma \Omega X \to X$ in 
	coomologia in un gruppo abeliano $A$ attraverso il seguente diagramma
	\begin{equation*}
		\begin{tikzcd}
			\overline{H}^{n}(X;A) \ar[r] \ar[ddd, "\simeq"]
			& \overline{H}^{n-1}(\Omega X; A) \ar[d] \\
			& {[\Omega X, K(A,n-1)]_{*}} \ar[d] \\
			& {[\Omega X, \Omega K(A,n)]_{*}} \ar[d] \\
			{[X, K(A,n)]_{*}} \ar[r, "ev^{*}"]
			& {[\Sigma \Omega X, K(A,n)]_{*}}\,.
		\end{tikzcd}
	\end{equation*}
\end{oss}

\begin{prop}\label{ev-mod-c}
 	Sia $\Cc$ un anello di Serre. Siano $n > 1$ e $X$ semplicemente connesso
 	tale che $H_{i}(X) \in \Cc$ per ogni $i < n$.
 	Allora
 	\begin{equation*}
 		ev_{*} : \overline{H}_{i-1}(\Omega X) \longrightarrow \overline{H}_{i}(X)
 	\end{equation*}
 	è un isomorfismo $\mathrm{mod}\Cc$, per ogni $i < 2n-1$, e
 	epimorfismo $\mathrm{mod}\Cc$ per $i=2n-1$.
 	\begin{proof}
 		Per quello che abbiamo visto
 		\begin{equation*}
 			\overline{H}_{i}(\Omega X) \in \Cc\,, \quad \text{per ogni } i < n-1\,,
 		\end{equation*}
 		quindi analizzando la sequenza spettrale della fibrazione
 		$\Omega X \subset P X \to X$ notiamo che
 		\missingfigure{paginetta che vabbé tanto prima che ne stampi una ci rivediamo l'anno prossimo.}
 		Solito discorso: in grado $(2n-1,0)$ parte un differenziale che tocca
 		un quadrante non banale in posizione $(n,n-1)$, e quindi possiamo
 		trovare un epi su un quoziente; più in basso invece abbiamo via libera e i differenzali
 		partono dalla riga zero-esima per arrivare, come isomorfismi, alla colonna zero-esima.
 		Si ottiene così il triangolo commutativo
 		\begin{equation*}
 			\begin{tikzcd}
 				H_{2n-1}(X) \ar[rr, "\simeq"] & & \frac{H_{2n-2}(\Omega X)}{\operatorname{im}d^{n}} \\
 				& H_{2n-2}(\Omega X) \ar[ul, two heads, "ev_{*}"] \ar[ur] & \,.
 			\end{tikzcd}
 		\end{equation*}
 	\end{proof}
\end{prop}

Se applichiamo il funtore di sospensione alla mappa $\sigma_{X}$ di aggiunzione,
allora otteniamo il triangolo commutativo
\begin{equation*}
	\begin{tikzcd}
		\Sigma X \ar[rr, "\Sigma \sigma_{X}"] \ar[dr, "\cat{1}_{\Sigma X}"]
		& & \Sigma \Omega \Sigma X \ar[dl, "ev_{\Sigma X}"] \\
		& \Sigma X \,,
	\end{tikzcd}
\end{equation*}
che in omologia risotta induce lo spezzamento
\begin{equation*}
	\begin{tikzcd}
		\overline{H}_{i}(X) \ar[d, "\simeq"] \ar[rr, "(\simeq_{X})_{*}"]
		& & H_{i}(\Omega \Sigma X) \ar[d, "\simeq"] \\
		\overline{H}_{i+1}(\Sigma X) \ar[rr, "\Sigma (\sigma_{X})_{*}"] \ar[dr, "\simeq"]
		& & H_{i+1}(\Sigma \Omega \Sigma X)   \ar[dl, "(ev_{X})_{*}"]\\
		& H_{i+1}(\Sigma X) \,. &
	\end{tikzcd}
\end{equation*}
Per la \hyperref[ev-mod-c]{Proposizione~\ref{ev-mod-C}},
se $X$ è $(n-1)$-connesso oppure $H_{i}(X) \in \Cc$ per ogni $i < n$,
allora la mappa 
\begin{equation*}
	(ev_{\Sigma X})_{*} : H_{i}(\Omega \Sigma X) \longrightarrow H_{i+1}(\Sigma X)
\end{equation*}
è un isomorfismo $\mathrm{mod} \Cc$ per ogni $i < 2n$, come pure $(\sigma_{X})_{*}$.
Applicando il {Teorema di Whitehead mod C} si ottiene quindi
\begin{thm}[Sospensione di Freudenthal $\mathrm{mod}\Cc$]
	Sia $\Cc$ un ideale di Serre aciclico, $n \ge 1$
	e $X$ uno spazio topologico semplicemente connesso,
	tale che $\overline{H}_{i}(X) \in \Cc$, per ogni $i < n$. Allora
	l'omomorfismo di sospensione
	\begin{equation*}
		\pi_{i}(X, \ast) \longrightarrow ...
	\end{equation*}
\end{thm}

Devo recuperare\todo{finire di copiare.}

\section{Operazioni di Steenrod}

Per il resto della sezione lavoreremo sempre 
in coomologia a coefficienti in $\Z/2\Z$,
e per semplicità scriveremo solamente $H^{*}(X)$.
Che operazioni modulo $2$ conosciamo in coomologia?

\begin{ex}
	Un esempio banale potrebbe essere semplicemente
	$x \mapsto x^{m}$, e se $m = 2^{n}$ allora è una mappa additiva.
\end{ex}

\begin{ex}
	Ricordiamo che abbiamo già parlato dell'\textbf{omomorfismo di Bockstein}
	\begin{equation*}
		\beta_{2} : H^{i}(X) \longrightarrow H^{i+1}(X)\,,
	\end{equation*}
	dato dall'omomorfismo di connessione della successione in coomologia,
	ottenuta applicando il funtore $\Hom(C_{*}(X), - )$ alla successione
	esatta corta
	\begin{equation*}
		\begin{tikzcd}
			0 \ar[r] & \Z/2\Z \ar[r]
			& \Z/4\Z \ar[r]
			& \Z/2\Z \ar[r]
			& 0\,.
		\end{tikzcd}
	\end{equation*}
\end{ex}

\begin{exercise}
	Provare a calcolare l'omomorfismo di Bockstein per $X = \R\PP^{2}$
	e dimostrare che
	\begin{equation*}
		\beta_{2} : H^{1}(\R\PP^{2}) \longrightarrow H^{2}(\R\PP^{2})
	\end{equation*}
	è un isomorfismo.
\end{exercise}

\begin{thm}\label{steenrod-uniche}
	Esiste un'unica famiglia di trasformazioni naturali additive
	\begin{equation*}
		Sq^{k} : H^{n}(X) \longrightarrow H^{n+k}(X)\,, \quad \text{con } n,k \ge 0\,,
	\end{equation*}
	tali che soddisfino:
	\begin{itemize}
		\item $Sq^{0}x = x$ ;
		\item $Sq^{k}x = x^{2}$ se $|x|=k$;
		\item $Sq^{k}x = 0$ se $k > |x|$;
		\item vale la \textbf{formula di Cartan}
		\begin{equation}\label{formula-cartan}
			Sq^{k}(x \cup y) = \sum_{i + j = k} Sq^{i}(x) \cup Sq^{j}(y)\,.
		\end{equation}
	\end{itemize}
	Chiameremo queste trasformazioni naturali le \textbf{operazioni di Steenrod}.
\end{thm}

Se mettiamo insieme tutti quadrati di Steenrod,
possiamo vedere $Sq$ come un morfismo che si ``spalma''
su tutto l'anello di coomologia
\begin{equation*}
	Sq : H^{*}(X) \longrightarrow H^{*}(X)\,, \quad
	Sq(x) = Sq^{0}(x) + Sq^{1}(x) + Sq^{2}(x) + \dots
\end{equation*}
Allora la formula di Cartan~\eqref{formula-cartan} in questo caso
diventa la condizione di moltiplicatività
\begin{equation*}
	Sq(x \cup y) = Sq(x) \cup Sq(y)\,.
\end{equation*}
Per il \textbf{Lemma di Yoneda}, le trasformazioni naturali
$Sq^{k}$ possono essere rappresentate dalla coomologia
dello spazio classificante, o meglio
\begin{equation*}
	H^{n}(K_{n}) \longrightarrow H^{n+k}(K_{n})\,, \quad K_{n} := K(\Z/2\Z,n)\,.
\end{equation*}

\begin{prop}
	Le operazioni di Steenrod sono \textbf{stabili} e, per ogni $n$ e $q$,
	il seguente diagramma commuta:
	\begin{equation*}
		\begin{tikzcd}
			\overline{H}^{q}(X) \ar[r, "Sq^{n}"] \ar[d, "\Sigma"]
			& \overline{H}^{q+n}(X) \ar[d, "\Sigma"] \\
			\overline{H}^{q+1}(\Sigma X) \ar[r, "Sq^{n}"]
			& \overline{H}^{q+n+1}(\Sigma X)\,.
		\end{tikzcd}
	\end{equation*}
	\begin{proof}
		La sospensione $\Sigma$ è indotta dal
		cross product definito dal \textbf{Teorema di Künneth}
		\begin{equation*}
			\overline{H}(S^{1},\ast) \otimes \overline{H}^{*}(X,x_{0})
			\longrightarrow H^{*}(S^{1} \times X, S^{1} \vee X) \simeq \overline{H}^{*}(S^{1} \wedge X)\,, 
			\quad \omega_{1} \otimes \alpha \longmapsto \pi_{1}^{*}(\omega_{1}) \cup \pi_{2}^{*}(\alpha)\,,
		\end{equation*}
		dove $\omega_{1}$ è il generatore della coomologia ridotta di $S^{1}$ in grado $1$.
		\begin{equation*}
			\begin{tikzcd}
				\alpha \ar[r, mapsto] \ar[d, mapsto, "Sq^{n}"]
				& \pi_{1}^{*}(\omega_{1}) \cup \pi_{2}^{*}(\alpha)
				\ar[d, mapsto, "Sq^{n}"] \\
				Sq^{n} \alpha \ar[r, mapsto, dashed]
				& Sq^{n}(\pi_{1}^{*}(\omega_{1}) \cup \pi_{2}^{*}(\alpha))\,,
			\end{tikzcd}
		\end{equation*}
		dove il termine in basso a destra per la formula di Cartan è proprio
		\begin{equation*}
			Sq^{n}(\pi_{1}^{*}(\omega_{1}) \cup \pi_{2}^{*}(\alpha))
			= \sum_{i+j=n} Sq^{i}(\pi_{1}^{*}(\omega_{1}) \cup Sq^{j}(\pi_{2}^{*}(\alpha))
			= \pi_{1}^{*}(\omega_{1}) \cup Sq^{n}\pi_{2}^{*}(\alpha)\,,
		\end{equation*}
		dove abbiamo usato che $Sq^{n}(\pi_{1}^{*}(\omega_{1}))=0$ per ogni $n > 0$.
	\end{proof}
\end{prop}

Conosciamo già un'operazione di Steenrod:
\begin{prop}
	La prima operazione di Steenrod è l'omomorfismo di Bockstein, i.e. $Sq^{1}=\beta_{2}$.
	\begin{proof}
		Dimostriamo il teorema per induzione su $n$.
		Facciamo vedere che $Sq^{1}$ è non banale e alza di $1$
		il grado della coomologia. Siccome soddisfa tutte le proprietà descritte dal
		teorema e anche $\beta_{2}$ soddisfa le condizioni del teorema,
		allora per unicità devono essere uguali.
		Ma lo devo copiare perché non so come scriverlo, c'è la solita paginetta
		della SS della fibrazione dei loop.
	\end{proof}
\end{prop}

\begin{lemma}
	Il prodotto cup in $H^{*}(\Sigma X)$ è banale
	in grado $> 0$.
	\begin{proof}
		Segue perché la sospensione $\Sigma X$ 
		si può scrivere come l'unione di due coni
		$\Sigma X = C_{+}X \cup C_{-}X$,
		che sono due sottospazi contraibili.
		Ma allora
		\begin{equation*}
			\begin{tikzcd}
			H^{*}(\Sigma X, \ast) \otimes H^{*}(\Sigma X, \ast)
			\ar[r, "\smile"] & H^{*}(\Sigma X, \ast) \\
			H^{*}(\Sigma X, C_{+}X) \otimes H^{*}(\Sigma X, C_{-}X)
			 \ar[r, "\smile"] & H^{*}(\Sigma X, \Sigma X) = 0 \ar[u] \,.
			\end{tikzcd}
		\end{equation*}
	\end{proof}
\end{lemma}