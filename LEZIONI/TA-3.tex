%%% Lecture 3

\lecture[Esistenza delle sezioni di Postnikov. Unicità delle sezioni di Postnikov a meno di equivalenza omotopica debole. Costruzione della torre di Postnikov. Spazi di Moore e spazi di Eilenberg-MacLane.]{2023-03-06}

% L'ultima volta abbiamo concluso la lezione affermando che esistono
% le \textbf{sezioni di Postnikov}.

\begin{ex}
	La sezione $P_{0}$ è uno spazio debolmente omotopicamente
	equivalente a $\pi_{0}(X)$.
\end{ex}

\begin{ex}
	Sia $X$ uno spazio connesso per archi.
	La sezione $P_{1}(X)$ è uno spazio connesso per archi con
	\begin{equation*}
		\pi_{1}(P_{1}(X),\ast) \simeq \pi_{1}(X)
	\end{equation*}
	e con rivestimento universale \emph{debolmente contraibile}\footnote{Uno spazio $Y$ è \textbf{debolmente contraibile} se $\pi_{k}(Y,\ast) = 0$, per ogni $k>0$.}.
\end{ex}

\begin{df}
	Uno spazio topologico $X$
	si dice \textbf{asferico} se il suo 
	rivestimento universale è contraibile.
\end{df}

\begin{oss}
	Per ogni gruppo $\pi$,
	esiste un CW complesso $K$ connesso per archi e asferico
	tale che $\pi_{1}(K,\ast) = \pi$ .
	Infatti, possiamo sempre considerare una presentazione
	\begin{equation*}
		\begin{tikzcd}
			F_{1} \ar[r, "a"] & F_{0} \ar[r] & \pi \ar[r] & 0\,, 
		\end{tikzcd}
	\end{equation*}
	con $F_{1}$ e $F_{0}$ gruppi liberi\footnote{Ricordiamo il \textbf{Teorema di Nielsen-Schreier}: un sottogruppo di un gruppo libero è a sua volta libero.};
	quindi, denotando con $I_{j}$ un insieme libero di generatori per $F_{j}$,
	possiamo definire una mappa
	\begin{equation*}
		\begin{tikzcd}
			\bigvee_{I_{1}} S^{1} \ar[r, "\phi_{a}"] & \bigvee_{I_{0}} S^{1}\,.
		\end{tikzcd}
	\end{equation*}
	Allora, grazie al \textbf{Teorema di Van Kampen} si può verificare
	che il mapping cone $C_{\phi_{a}}$ è un CW complesso
	che ha il gruppo fondamentale desiderato;
	dato che vogliamo uno spazio asferico, allora consideriamo
	la sua prima sezione di Postnikov
	$P_{1}\left(C_{\phi_{a}}\right)$.
\end{oss}

\begin{oss}
	Se $X$ è un CW complesso, 
	posso ottenere $P_{n}(X)$ come CW complesso.
\end{oss}

Ci chiediamo dunque se le sezioni di Postnikov siano \emph{uniche}
in qualche senso.
\begin{prop}\label{CW-est}
	Sia $n \ge 0$ e $Y$ uno spazio topologico tale che $\pi_{q}(Y, \ast)=0$,
	per ogni $q>n$. Dato un CW complesso relativo $(X,A)$ in cui
	tutte le celle in $X \setminus A$ hanno dimensione $\ge n+2$,
	allora la mappa di restrizione
	\begin{equation*}
		[X,Y] \longrightarrow [A,Y]
	\end{equation*}
	è bigettiva. Se ci sono anche ($n+1$)-celle,
	allora è iniettiva.
	\begin{proof}
		Usiamo che l'ipotesi $\pi_{q}(Y,\ast)=0$, per ogni $q > n$,
		è equivalente al fatto che	
		le mappe $S^{q} \to Y$ con immagine nella componente
		connessa contenente $\ast$ si estendono a tutto $D^{q+1}$.
		\begin{itemize}
			\item \textbf{Surgettività}: data $f:A \to Y$,
			per ogni $q \ge n+2$ la mappa di incollamento
			\begin{equation*}
				S^{q-1} \longrightarrow X^{q-1} \longrightarrow Y
			\end{equation*}
			è omotopicamente banale per ipotesi,
			dunque si estende a una mappa $D^{q} \to Y$.
			Ripetendo questo procedimento per tutte
			le celle in $X \setminus A$, definiamo così
			un'estensione $\overline{f}:X \to Y$;
			
			\item \textbf{Iniettività}:
			consideriamo la coppia
			\begin{equation*}
				\Big(X \times I, (X \times \de I) \cup (A \times I) \Big)
			\end{equation*}
			come un CW complesso relativo.
			Se $q \ge n+1$, per ipotesi sull'omotopia ogni $q$-cella di $X \setminus A$
			è bordo di una cella di dimensione $q+1$,\todo{È giusto?}
			quindi ogni coppia di mappe $f_{0},f_{1}:X \to Y$ 
			le cui restrizioni ad $A$ sono omotope inducono un'omotopia
			\begin{equation*}
				F : \Big( X \times \de I \Big) \cup \Big( A \times I \Big)
				\longrightarrow Y\,.
			\end{equation*}
			Ripetendo il ragionamento del punto precedente,
			possiamo estendere $F$ una cella alla volta,
			fino a ottenere un'omotopia $\widetilde{F}:X \times I \to Y$
			tra $f_{0}$ e $f_{1}$.
		\end{itemize}
	\end{proof}
\end{prop}

\begin{cor}
	Sia $X$ un CW complesso $n$-connesso e
	$Y$ uno spazio con omotopia concentrata in grado al più $n$\footnote{Cioè tale che $\pi_{q}(Y,\ast) = 0$ per ogni $q > n$.}.
	Ogni mappa $f:X \to Y$ è omotopa a una costante.
	\begin{proof}
		Per ipotesi, possiamo considerare $X$
		come un CW complesso senza celle di dimensione $0< d < n+1$.
		Poniamo $A = \{\ast\}$ il punto base di $X$.
		Siccome l'$n$-scheletro di $X$ consiste del solo punto base,
		i.e. $X_{n} = \{\ast\}$, allora la coppia $(X,\ast)$
		soddisfa le ipotesi della \hyperref[CW-est]{Proposizione~\ref{CW-est}}
		e abbiamo una bigezione $[X,Y] \to [\ast,Y]$.
	\end{proof}
\end{cor}

Sia $f:X \to Y$ e consideriamo le mappe $X \to P_{m}(X)$ e $Y \to P_{n}(Y)$.
Se $m \ge n$, allora per la \hyperref[CW-est]{Proposizione~\ref{CW-est}}
abbiamo una bigezione $[X,P_{n}(Y)] \simeq [P_{m}(X),P_{n}(Y)]$,
quindi \emph{esiste un'unica} mappa (\emph{a meno di omotopia}) tra
le sezioni di Postnikov che fa commutare il seguente diagramma:
\begin{equation*}
	\begin{tikzcd}
		X \ar[r, "f"] \ar[d] & Y \ar[d] \\
		P_{m}(X) \ar[r, "\exists !", dashed] & P_{n}(Y)\,.
	\end{tikzcd}
\end{equation*}
Se consideriamo il caso particolare in cui $X=Y$ e $f =\cat{1}_{X}$,
allora per ogni $m \ge n$ abbiamo un'unica mappa
\begin{equation*}
	\begin{tikzcd}
		& X \ar[dl] \ar[dr] & \\
		P_{m}(X) \ar[rr, dashed, "\exists !"] & & P_{n}(X)\,,
	\end{tikzcd}
\end{equation*}
dunque ponendo $m=n$ si deduce che deve essere un'equivalenza omotopica \emph{debole}:
questo significa che tutte le costruzioni 
che soddisfino le condizioni del \hyperref[sez-Postnikov]{Teorema~\ref{sez-Postnikov}}
producono spazi topologici debolmente equivalenti.
Quindi le sezioni di Postnikov sono uniche in questo senso!
Questo implica che se è possibile ottenere $P_{n}(X)$ 
come CW complessi (e.g. quando $X$ è un CW complesso),
allora le sezioni di Postnikov sono uniche a meno di equivalenza omotopica.

\begin{oss}
	La mappa $X \to P_{m}(X)$ è \emph{iniziale} nella
	categoria $\cat{HoTop}$ degli spazi topologici e 
	funzioni continue a meno di omotopia
	verso gli spazi con omotopia banale in grado $> n$:
	più esplicitamente, se $Y$ è uno spazio con $\pi_{q}(Y,\ast)=0$
	per $q > n$, allora ogni mappa $X \to Y$ fattorizza in maniera
	unica (a meno di omotopia) attraverso l'$n$-esima sezione di Postnikov:
	\begin{equation*}
	\begin{tikzcd}
		& P_{n}(X) \ar[d, "\exists !", dashed] \\
		X  \ar[r] \ar[ur] & Y \,.
	\end{tikzcd}
\end{equation*}
\end{oss}

Partendo da $Y=P_{0}(X)$ e $n=1$, reiterando la costruzione sopra
otteniamo una sorta di diagramma limite chiamato
\textbf{torre di Postnikov}.
\begin{equation}\label{TP}
	\begin{tikzcd}
		& \dots \ar[d] \\
		& P_{2}(X) \ar[d] \\
		& P_{1} (X) \ar[d] \\
		X \ar[uur] \ar[ur] \ar[r] & P_{0}(X) \,.
	\end{tikzcd} \tag{TP}
\end{equation}

\section{Spazi di Eilenberg-MacLane}

\begin{prop}\label{Moore-exist}
	Dato $k \ge 1$ e $A$ un gruppo abeliano,
	esiste un CW complesso $X$ tale che
	\begin{equation*}
		\widetilde{H}_{k}(X) = A\,,
		\quad \widetilde{H}_{q}(X) = 0 \text{ per } q \ne k\,.
	\end{equation*}
	Se $k > 1$, il complesso $X$ può essere costruito
	semplicemente connesso.
	\begin{proof}
		Consideriamo una presentazione
		\begin{equation*}
			\begin{tikzcd}
				0 \ar[r] & F_{1} \ar[r, "a"] & F_{0} \ar[r] & A \ar[r] & 0\,, 
			\end{tikzcd}
		\end{equation*}
		con $F_{1}$ e $F_{0}$ gruppi abeliani liberi,
		e sia $G_{i}$ una base di generatori per $F_{i}$.
		Consideriamo allora il bouquet di circonferenze
		\begin{equation*}
			X^{(k)} := \bigvee_{a \in G_{0}} S^{k}_{a}
		\end{equation*}
		e definiamo la mappa di incollamento
		\begin{equation*}
			\alpha : \bigsqcup_{b \in G_{1}} S^{k}_{b}
			\longrightarrow X^{(k)}
		\end{equation*}
		indotta tramite l'omomorfismo $F_{1} \longrightarrow F_{0}$
		dato da $b \longmapsto \sum_{a \in G_{0}} n_{a} a$, cioè:
		se $n_{a} \ne 0$, allora il disco $b$-esimo si incolla su $S_{a}^{k}$
		con una mappa $S_{b}^{k} \to S_{a}^{k}$ di grado $n_{a}$.
		\missingfigure{C'è un disegnetto con le sferette che si incollano.}
		Per costruzione $\alpha$ realizza $F_{1} \to F_{0}$ in omologia,
		quindi otteniamo
		\begin{equation*}
			\widetilde{H}_{q}(X) =
			\begin{cases}
				A\,, \quad &\text{se } q = k\,; \\
				0\,, \quad &\text{altrimenti.}
			\end{cases}\qedhere
		\end{equation*}
	\end{proof}
\end{prop}

\begin{df}
	Uno spazio $X$ come nella \hyperref[Moore-exist]{Proposition~\ref{Moore-exist}}
	si chiama \textbf{spazio di Moore} e si indica con $M(A,k)$.
\end{df}

\begin{fact}
	Una costruzione analoga permette di costruire
	anche CW complessi $X$ di dimensione $2$,
	il cui gruppo fondamentale è $\pi_{1}(X,\ast) = \pi$,
	per un qualsiasi gruppo $\pi$ prefissato.
\end{fact}

Il \textbf{Teorema di Hurewicz} ci dice che l'omotopia di
uno spazio di Moore $M(\pi, n)$, con $n \ge 2$, 
coincide con l'omologia
fino al grado $n$, ma in grado superiore non sappiamo nulla;
possiamo ``uccidere'' la sua omotopia in grado superiore a $n$
considerando il ``troncato'' $n$-esimo, ovvero la sezione di Postnikov
\begin{equation*}
	\tau_{\le n}M := P_{n}(M)\,,
\end{equation*}
così da ottenere uno spazio con omotopia $\pi_{n}(\tau_{\le n}M, \ast) = \pi$ 
e $\pi_{q}(\tau_{\le n}M,\ast)=0$ se $q \ne n$.

\begin{df}
	Uno spazio $X$ tale che $\pi_{n}(X,\ast)=\pi$ e
	\begin{equation*}
		\pi_{q}(X,\ast) = 0\,, \quad \text{se } q \ne n
	\end{equation*}
	si chiama \textbf{spazio di Eilenberg-MacLane} di tipo
	$K(\pi,n)$.
\end{df}

\begin{ex}
	Abbiamo già incontrato degli spazi di Eilenberg-MacLane nel
	corso di Elementi di Topologia Algebrica, come ad esempio:
	\begin{itemize}
		\item $K(\Z,1) = S^{1}$, il cui rivestimento universale è la retta $\R$;
		\item $K(\Z/2,1) = \R\PP^{\infty}$, 
		il cui rivestimento universale è la sfera $S^{\infty}$;
		\item $K(\Z,2) = \C\PP^{\infty}$, 
		il cui rivestimento universale è sempre la sfera $S^{\infty}$.
	\end{itemize}
\end{ex}

\begin{ex}
Se consideriamo la torre di Postnikov
\begin{equation*}
	\begin{tikzcd}
		& \dots	\ar[d] \\	
		& \tau_{\le 2}X \ar[d, "t_{2}"] \\
		& \tau_{\le 1}X \ar[d] \\
		X \ar[r] \ar[ur] \ar[uur]
		& \tau_{\le 0}X  \,,
	\end{tikzcd}
\end{equation*}
la fibra omotopica di $t_{2}$ è lo\footnote{Nella prossima lezione dimostreremo, infatti, che gli spazi di Eilenberg-MacLane sono unici a meno di equivalenza omotopica.} spazio di Eilenberg-MacLane
$F = K\big(\pi_{2}(X,\ast),2\big)$.\todo{Fai questo esercizio.}
\end{ex}




