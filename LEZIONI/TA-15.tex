 %%% Lezione 15


\lecture[Omologia dello spazio dei loop su una sfera. Successione spettrale come funtore e teorema di confronto di successioni spettrali. Due dimostrazioni del teorema di successione spettrale di Serre.]{2023-04-24}

	Abbiamo ricavato nuovamente il teorema di Hurewicz
	per gruppi di omotopia superiore,
	assumendo il risultato vero solo per $\pi_{1}$.\todo{recupera}
	
	\begin{ex}
		Studiamo i loop sulla sfera $S^{n}$.
		Supponiamo che $n > 1$ così da avere la base semlicemente connessa.
		Dalla fibrazione
		\begin{equation*}
			\begin{tikzcd}
				\Omega S^{n} \ar[r] & PS^{n} \ar[r] & S^{n}
			\end{tikzcd}
		\end{equation*}
		otteniamo la successione spettrale
		\begin{equation*}
			E^{2}_{p,q} = H_{p}\big(S^{n},H_{q}(\Omega S^{n}) \big)\,.
		\end{equation*}
		La colonna $p=0$ è lo $0$-esimo gruppo di omologia di uno spazio connesso
		a coefficienti in $H_{q}(\Omega S^{n})$, quindi la prima colonna è in realtà
		l'omologia della fibra $H_{q}(\Omega S^{n})$. 
		Tutte le altre colonne hanno gruppi banali, fino a quando
		$p = n$, in cui ritroviamo l'omologia in grado $n$ di $S^{n}$
		a coefficienti in $H_{q}(\Omega S^{n})$, che è isomorfa a $\Z$ per $q=n-1$.
		\missingfigure{Devo mettere una bella pagina di sta roba.}
		
		Procedendo per induzione, questo dimostra che per ogni $k \in \N$
		l'omologia di $\Omega S^{n}$ è
		\begin{equation*}
			H_{k(n-1)}(\Omega S^{n}) \simeq \Z
		\end{equation*}
		ed è banale in tutti gli altri gradi.\todo{non ho capito la roba della formula di Künneth e il limite...}
	\end{ex}
	
	
	
	\section{Confronto di successioni spettrali}
	
	
	La costruzione del \hyperref[SS-machine]{Teorema~\ref{SS-machine}}
	che associa ad un complesso filtrato graduato $(C,d,F)$ una successione spettrale
	è \emph{funtoriale}, nel senso che un morfismo tra complessi filtrati e graduati
	induce in maniera naturale un morfismo di successioni spettrali.
	
	\begin{thm}
		Sia $\tau : C \to C'$ una mappa di complessi differenziali graduati e filtrati.
		Supponiamo che le filtrazioni in omologia 
		\begin{equation*}
			\bigcup_{s} F_{s}H_{n} = H_{n}
		\end{equation*}
		siano convergenti e limitate dal basso.
		Se per qualche $r \ge 1$ la mappa $\tau^{r} : E^{r} \to (E')^{r}$ è un isomorfismo,
		allora per tutte le pagine $t \ge r$ si ha $E^{t} \simeq (E')^{t}$ e
		la mappa in omologia
		\begin{equation*}
			\tau_{*} : H_{*}(C) \longrightarrow H_{*}(C')
		\end{equation*}
		è un isomorfismo.
		\begin{proof}
			\`E chiaro che se le successioni spettrali
			coincidono alla pagina $r$, siccome le pagine successive
			sono determinate da quelle precedenti, allora
			$\tau^{t}$ è un isomorfismo per ogni $t \ge r$.
			In particolare, anche il limite
			\begin{equation*}
				\tau^{\infty} : E^{\infty} \longrightarrow (E')^{\infty}
			\end{equation*}
			è un isomorfismo.
			
			Abbiamo un diagramma commutativo
			\begin{equation*}
				\begin{tikzcd}
					0 \ar[r]
					& F_{s-1}H_{n}(C) \ar[r] \ar[d]
					& F_{s}H_{n}(C) \ar[r] \ar[d]
					& E^{\infty}_{s,n-s} \ar[r] \ar[d]
					& 0 \\
					0 \ar[r]
					& F_{s-1}H_{n}(C') \ar[r]
					& F_{s}H_{n}(C') \ar[r]
					& (E')^{\infty}_{s,n-s} \ar[r]
					& 0 \,.
				\end{tikzcd}
			\end{equation*}
			Fissato $n$, per valori abbastanza piccoli di $s$
			si ha $F_{s-1}H_{n}(C) = 0$ e $F_{s-1}H_{n}(C')$ perché la filtrazione è limitata dal basso;
			di conseguenza, per induzione su $s$ otteniamo
			\begin{equation*}
				\tau_{*} : F_{s}H_{n}(C) \simeq F_{s}H_{n}(C')\,,
				\quad \text{per ogni } s\,.
			\end{equation*}
			Siccome la filtrazione dà tutta l'omologia, si conclude che
			\begin{equation*}
				H_{*}(C) = \bigcup_{s} F_{s}H_{*}(C)
				\simeq  \bigcup_{s} F_{s}H_{*}(C') = H_{*}(C')\,. \qedhere
			\end{equation*}
		\end{proof}
	\end{thm}
	
	
	\begin{proof}[Dimostrazione della successione spettrale di Serre~\ref{Serre-SS}]
		A meno di approssimazione CW, possiamo assumere che la base $B$
		sia un CW complesso. Detto $B^{(n)}$ l'$n$-scheletro di $B$,
		consideriamo la filtrazione data dagli scheletri
		\begin{equation*}
			\emptyset = B^{(-1)} \subset B^{(0)} \subset B^{(1)} \subset \dots
			\subset B^{(n)} \subset B^{(n+1)} \subset \dots \subset B\,.
		\end{equation*}
		Tramite $\pi : E \to B$, 
		ponendo $E^{i} := \pi^{-1}\left(B^{(i)}\right)$
		otteniamo una filtrazione su $E$ e di conseguenza
		questa induce una filtrazione sulle sue catene singolari $C_{*}(E)$.
		
		Definiamo $E^{0}_{p,q} := C_{p+1}\left( E^{p}, E^{p-1} \right)$
		e consideriamo $d^{0}$ come il differenziale indotto da
		$C_{*}(E)$. Mettendo in moto la macchina del
		\hyperref[SS-machine]{Teorema~\ref{SS-machine}}, si ottiene
		\begin{equation*}
			E^{1}_{p,q} = H_{p+q}\left( E^{p}, E^{p-1} \right)
			\simeq H_{p+q}\left( E^{p}, \pi^{-1}(B^{(p)} \setminus \cup_{i}\{c_{i}\} \right)
		\end{equation*}
		dove l'isomorfismo di destra è dato per \emph{escissione},
		rimuovendo i centri $c_{i}$ dei dischi $D^{p}$ nel $p$-scheletro.
		Segue quindi che
		\begin{align*}
			E^{1}_{p,q} &\simeq H_{p+q}\left( E^{p}, \pi^{-1}(B^{(p)} \setminus \cup_{i}\{c_{i}\} \right) \\
			&\simeq H_{p+q}\left( \cup_{i} \pi^{-1}(D^{p}), \cup_{i} \pi^{-1}(D^{p} \setminus \{c_{i}\} \right) \\
			&\simeq \bigoplus_{i} H_{p+q}\left( \pi^{-1}(D^{p}), \pi^{-1}(D^{p} \setminus \{c_{i}\} \right) \\
			&\simeq \bigoplus_{i} H_{p+q}\left( \pi^{-1}(D^{p}), \pi^{-1}(\de D^{p}) \right) \,.
		\end{align*}
		A meno di approssimazione cellulare,
		possiamo supporre che $\pi$ sia banale sul disco $D_{i}^{p}$
		con fibra $F_{i}:= \pi^{-1}(c_{i})$; in questo modo,
		la \textbf{formula di Künneth} ci dà
		\begin{align*}
			H_{p+q}\left( \pi^{-1}(D^{p}), \pi^{-1}(\de D^{p}) \right)
			&\simeq H_{p+q}(D_{i}^{p} \times F_{i}, \de D^{p} \times F_{i}) \\
			&\simeq H_{p+q}(D_{i}^{p}, \de D^{p}) \otimes H_{p+q}(F_{i}) \\
			&\simeq H_{p+q}(F_{i})\,,
		\end{align*}
		dove abbiamo sfruttato che l'omologia del disco $D^{p}$
		relativa al suo bordo è banale, tranne in grado $p$
		dove è un modulo libero di rango $1$.
		Segue dunque che
		\begin{equation*}
			E^{1}_{p,q} \simeq \bigoplus_{i} H_{q}(F)\,, 
		\end{equation*}
		dove $i$ indicizza le celle di $B$.
		
		Nel caso in cui $B$ è semplicemente connesso,
		allora posso trovare una trivializzazione globale,
		cioè definire una mappa CW da $F$ a $F'$ \todo{Non ho capito} ...
		Abbiamo quindi che $E^{1}_{p,q} = C^{\mathrm{cell}}_{p}(B) \otimes H_{q}(F)$
		e il differenziale $d^{1}$ corrisponde al morfismo di connessione indotto
		dalla tripla $(B^{(p)},B^{(p-1)},B^{(p-2)})$, 
		cioè il differenziale di $C^{\mathrm{cell}}_{*}(B)$.
		Questo produce la seconda pagina
		\begin{equation*}
			E^{2}_{p,q} = H_{p}\big(B; H_{q}(F)\big)\,.
		\end{equation*}
		
		Se $B$ \emph{non} è semplicemente connesso,
		allora considero $\widetilde{B}$ il rivestimento universale.
		Dal procedimento di prima si ottiene così...\todo{finire}
	\end{proof}
	
	\begin{proof}[Dimostrazione alternativa (Dress)]
		Mostriamo quest'altra dimostrazione che non fa uso della struttura CW.
		Data $\pi : E \to B$, costruiamo un ``arricchimento'' del
		complesso delle catene singolari: infatti,
		definiamo i \textbf{bisimplessi singolari} di $\pi$ come
		\begin{equation*}
		 	\mathrm{Sin_{s,t}}(\pi) :=
		 	\Set{(f,\sigma) | f : \Delta^{s} \times \Delta^{t} \to E, \sigma : \Delta^{s} \to B
		 	\text{ tale che soddisfino } (\ast) }\,,
		 \end{equation*} 
		 dove la condizione $(\ast)$ è che il seguente quadrato commuti
		 \begin{equation}
		 \begin{tikzcd}
		 	\Delta^{s} \times \Delta^{t} \ar[r, "f"] \ar[d] 
		 	& E \ar[d, "\pi"] \\
		 	\Delta^{s} \ar[r, "\sigma"] & B\,.
		 \end{tikzcd}
		 \end{equation}
		 Possiamo così definire un funtore
		 \begin{equation*}
		 	\mathrm{Sin}_{\bullet,\bullet}(\pi) : \Delta^{op} \times \Delta^{op}
		 	\longrightarrow \CSet\,,
		 \end{equation*}
		 e indichiamo con $R\mathrm{Sin}_{\bullet,\bullet}(\pi)$
		 l'$R$-modulo libero generato dai bisimplessi singolari.
		 Possiamo definire due differenziali $\de'$ e $\de''$ su questo modulo:\todo{copia formule}
		 Come nell'Esempio di Cech-deRham, 
		 poniamo $d = \de' + (-1)^{q}\de''$ su $R\mathrm{Sin}_{\bullet,\bullet}(\pi)$
		 e consideriamo due filtrazioni:
		 \begin{enumerate}
		 	\item per ``diagonali sinistre'' data da
		 	\begin{equation*}
		 		F_{p}\left( \mathrm{Sin}_{\bullet,\bullet}(\pi) \right)_{n} :=
		 		\bigoplus_{s+t=n\,, t \le p} \mathrm{Sin}_{s,t}(\pi)\,,
		 	\end{equation*}
		 	su cui consideriamo il differenziale $d^{0}=\de$.
		 	Dato un bisimplesso $(f, \sigma) \in \mathrm{Sin}_{s,t}(\pi)$,
		 	considero $\widetilde{f}$ tramite aggiunzione e considero il diagramma
		 	\begin{equation*}
			\begin{tikzcd}
				\Delta^{s} \ar[dr, dashed] \ar[rrd, "\widetilde{f}", bend left=20pt] 
				\ar[ddr, "\sigma"', bend right=20pt] & & \\
				& c^{-1}\left( E^{\Delta^{t}} \right) \ar[d] \ar[r] & E^{\Delta^{t}} \ar[d, "\pi"] \\
				& B \ar[r, "c"] & B^{\Delta^{t}}\,,
			\end{tikzcd}
			\end{equation*}
			dove $c$ manda $b \in B$ nel $t$-simplesso singolare costante in $b$.
			Per aggiunzione, i dati $(f,\sigma)$ e $(\widetilde{f},\sigma)$
			sono equivalenti e quindi, posto $E'_{t} := c^{-1}\left( E^{\Delta^{t}} \right)$,
			notiamo che 
			\begin{equation*}
				R\mathrm{Sin_{s,t}}(\pi) \simeq C_{s}(E'_{t})\,.
			\end{equation*}
			Dato che i simplessi singolari sono contraibili,
			allora $c : B \to B^{\Delta^{t}}$ è un'equivalenza omotopica
			e quindi $E'_{t}$ è omotopicamente equivalente a $E^{\Delta^{t}}$
			tramite la mappa $c : E \to E^{\Delta^{t}}$ che manda tutto in un simplesso costante.
			Ne deduciamo che le catene singolari sono $C_{s}(E'_{t}) \simeq C_{s}(E)$
			e quindi la prima pagina della successione spettrale è
			\begin{equation*}
				E'_{s,t} = H_{s}(E)\,, \quad s,t, \ge 0\,.
			\end{equation*}
			Dato che $d^{1} = \pm \de''$, allora concludiamo che
			\begin{equation*}
				E^{2}_{p,q} \simeq
				\begin{cases}
					0\,, \quad &\text{se } t \ne 0\,; \\
					H_{s}(E)\,, \quad &\text{se } t = 0\,.
				\end{cases}
			\end{equation*}
			
			\item Fissato un $s$-simplesso $\sigma$, considero il diagramma
			\begin{equation*}
			\begin{tikzcd}
				\Delta^{s} \times \Delta^{t} \ar[dr, dashed] \ar[rrd, "f", bend left=15pt] 
				\ar[ddr, "\mathrm{pr_{1}}"', bend right=20pt] & & \\
				& \sigma^{-1}\left( E \right) \ar[d, "\pi_{\sigma}"'] \ar[r] & E \ar[d, "\pi"] \\
				& \Delta^{s} \ar[ur, dashed] \ar[r, "\sigma"] & B\,,
			\end{tikzcd}
			\end{equation*}
			e per aggiunzione questo corrisponde a 
			\begin{equation*}
			\begin{tikzcd}
				\Delta^{t} \ar[dr, dashed] \ar[rrd, "f", bend left=15pt] 
				\ar[ddr, "\mathrm{pr_{1}}"', bend right=20pt] & & \\
				& j^{-1}(\ast) \ar[d] \ar[r] & E^{\Delta^{s}} \ar[d, "\pi"] \\
				& \ast \ar[r, "j:\ast \mapsto \sigma"] & B^{\Delta^{s}}\,.
			\end{tikzcd}
			\end{equation*}
			Notiamo che $j^{-1}(\ast) = \Gamma(\Delta^{s},\sigma^{-1}(E))$
			sono le sezioni da $\Delta^{s}$ in $\sigma^{-1}(E)$
			e quindi poniamo
			\begin{equation*}
				E^{0}_{s,t} = \bigoplus_{\sigma : \Delta_{s} \to B} 
				C_{t}\left( \Gamma(\Delta^{s},\sigma^{-1}(E)) \right)\,.
			\end{equation*}
			La prima pagina che otteniamo è
			\begin{equation*}
				E^{1}_{s,t} = \bigoplus_{\sigma : \Delta_{s} \to B} 
				H_{t}\left( \Gamma(\Delta^{s},\sigma^{-1}(E)) \right)\,,
			\end{equation*}
			con il differenziale $d^{1}$ dato dalla somma a segni alterni
			delle facce del modulo simpliciale $E^{1}_{s,t}$.
			
			Un morfismo $\phi : [s'] \to [s]$ nella categoria simpliciale
			induce una mappa continua
			\begin{equation*}
				\phi^{*} : \Gamma(\Delta^{s},\sigma^{-1}(E))
				\longrightarrow \Gamma(\Delta^{s'},(\sigma \circ \phi)^{-1}(E))
			\end{equation*}
			e quindi $\phi^{*}:E^{1}_{s,t} \to E^{1}_{s',t}$.
			Usando il fatto che $\pi$ è una fibrazione e che $\Delta^{s}$
			è contraibile, deduciamo che $\sigma^{-1}(E) \to \Delta^{s}$
			è banale e quindi abbiamo un'equivalenza omotopica
			\begin{equation*}
				\Gamma(\Delta^{s}, \sigma^{-1}(E)) \simeq F_{\sigma}\,.
			\end{equation*}
			...\todo{finire}
		 \end{enumerate}
	\end{proof}