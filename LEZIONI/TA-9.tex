 %%% Lezione 9


\lecture[Lezione 9.]{2023-03-27}



\begin{thm}
	C'è un isomorfismo di funtori $\mathrm{Princ}_{\mathrm{GL}(n,\R)} \simeq \mathrm{Vect}_{n}(B)$.
	\begin{proof}
		\begin{itemize}
			\item \textbf{Caso semplice: sia $X$ contraibile.} Allora ogni fibrato su $X$ è banale.
			Per mostrarlo è sufficiente costruire una sezione di $p:P \to X$.
			Dato $x_{0} \in X$, sappiamo che $\cat{1}_{X}$ è omotopa alla mappa costante in $x_{0}$.
			Preso un punto $q \in p^{-1}(x_{0})$, solleviamo la costante a una mappa $c:X \to P$
			tale che $p(q)=x_{0}$.
			\begin{equation*}
				\begin{tikzcd}
					& P \ar[d, "p"] \\
					X \ar[r, "x_{0}"] \ar[ur, "c"] & X\,.
				\end{tikzcd}
			\end{equation*}
			Presa $H: X \times I \to X$ un'omotopia di contrazione tale che
			 $H_{0} = x_{0}$ e $H_{1} = \cat{1}_{X}$, 
			 siccome $p$ è una fibrazione, 
			 allora $H$ si solleva a $\widetilde{H}_{t} : X \to P$ tale
			 che $\widetilde{H}_{0}=c$ e $\widetilde{H}_{1}$ è una sezione di $p$.
			 Quindi...
			 
			 \item \textbf{Caso generale: sia $X$ CW complesso.}
		\end{itemize}
	\end{proof}
\end{thm}

\begin{df}
	Chiamiamo \textbf{$G$-CW-coppia} una coppia CW $(X,A)$ 
	con un'azione (destre) di $G$ su $X$ tale che esista una filtrazione
	\begin{equation*}
		A =: F^{-1}X \subset F^{0}X \subset F^{1}X \subset \dots  
		\subset F^{n}X \subset \dots
		\subset \bigcup_{n \ge 0} F^{n}X = X
	\end{equation*}
	tale che $F^{n}X$ è ottenuto da $F^{n-1}X$ attaccando delle celle equivarianti del tipo
	\begin{equation*}
		D^{n} \times \left( H \backslash G \right)\,,
	\end{equation*}
	con $H \subset G$ sottogruppo e $H \backslash G = \{hG\}$ con azione di $G$ sulla destra.
	Possiamo vedere questa costruzione come il pushout del seguente diagramma:
	\begin{equation*}
	 	\begin{tikzcd}
	 		\coprod_{i} \de D_{i}^{n} \times \left( H \backslash G \right)
	 	\end{tikzcd}
	 \end{equation*}
	 dove $X$ ha la topologia del limite diretto (o topologia debole). 
\end{df}

\begin{oss}
	Banalmente, ogni CW complesso è un $G$-CW-complesso con $G=1$.
	...
\end{oss}

Se $\xi:P \xrightarrow{p} X$ è un fibrato principale su $G$, dove $X$ è un CW complesso,
lo scheletro di $X$ si solleva a un $G$-CW-complesso su $P$:
\begin{equation*}
	\begin{tikzcd}
		E(c^{*} \xi) \ar[r, "\widetilde{c}"] \ar[d] & P \ar[d] \\
		D^{n} \ar[r, "c"] & X\,,
	\end{tikzcd}
\end{equation*}
dove $c^{*}\xi$ è banale, con spazio totale $D^{n} \times G$.

\begin{df}
	Dati due fibrati $\xi = (E,p), \xi'=(E',p')$ con stessa fibra $F$,
	chiamiamo\textbf{bundle map} di $\xi$ su $\xi'$ un'applicazione continua
	$h:E \to E'$ che manda ogni fibra $E_{b}$ omeomorficamente in una fibra $E'_{b'}$.
	Quindi, questa mappa induce $\overline{h} : B \to B'$ che rende commutativo il diagramma
	\begin{equation*}
		\begin{tikzcd}
			E \ar[d, "p"] \ar[r, "h"] & E' \ar[d, "p'"] \\
			B \ar[r, "\overline{h}"] & B'\,.
		\end{tikzcd}
	\end{equation*}
	Se $\xi$ e $\xi'$ sono fibrati vettoriali richiediamo che
	$h_{b}:E_{b} \to E'_{b'}$ sia un isomorfismo lineare.
	Se $\xi$ e $\xi'$ sono fibrati principali sullo stesso gruppo $G$,
	richiediamo che sia $G$-invariante, i.e. $h(e \cdot g) = h(e) \cdot g$.
\end{df}

\begin{prop}
	Se $h:\xi \to \xi'$ è una bundle map,
	allora c'è un isomorfismo $g:\xi \simeq \overline{h}^{*}\xi'$
	tale che $h=\mathrm{pr}_{2} \circ g$.
	\begin{equation*}
		\begin{tikzcd}
			E(\xi) \ar[r, "\simeq"] \ar[d, "p"]
			& E(\overline{h}^{*}\xi') \ar[r, "\mathrm{pr}_{2}"] \ar[d]
			& E(\xi') \ar[d, "p'"] \\
			B \ar[r, equals] 
			& B \ar[r, "\overline{h}"]
			& B'\,.
		\end{tikzcd}
	\end{equation*}
	Se inoltre $\xi$ e $\xi'$ sono fibrati pringipali su $G$,
	allora gli isomorfismi sono $G$ invarianti.
	\begin{proof}
		Definiamo $g: E(\xi) \to E(\overline{h}^{*}\xi')$ come
		\begin{equation*}
			g(e) = \left( p(e), h(e) \right)\,.
		\end{equation*}
		Si verifica che $g$ è continua e manda ciascuna fibra $E_{b}$
		isomorficamente sulle fibre $E({\overline{h}^{*}\xi')_{b}}$,
		cioè è bigettiva con inversa continua (è sufficiente verificarlo
		localmente sulle banalizzazioni),
		e quindi concludiamo che è un isomorfismo.
	\end{proof}
\end{prop}

In parole povere, avere una bundle map è equivalente
a dire che il fibrato che stiamo considerando è in realtà un pullback.

\textbf{Torniamo alla dimostrazione:}\todo{Aggiustare l'ordine di queste cose!!!}

Per il caso semplice, sappiamo che 
$\mathrm{pr}_{1}^{*}:\mathrm{Princ}_{G}X \to \mathrm{Princ}_{G}(X  \times I)$
è surgettiva, quindi ogni fibrato $\xi : P \xrightarrow{p} X \times I$
è il pullback di qualche fibrato $p':E' \to B$, cioè
\begin{equation*}
	P \simeq \Set{ \big( (x,t),e' \big) | x = p'(e') }\,.
\end{equation*}
In altri termini si ha
 $$P = P\vert_{X \times \{0\}} \times I : (e,t) \to (p(e),t) \in X \times I \,,$$
e quindi basta dimostrare che ogni fibrato su $X \times I$
è il pullback lungo la restrizione a $X \times \{0\} \simeq X$.

Per la Proposizione ci basta trovare una bundle map
\begin{equation*}
	h : P \to P_{0}:=p^{-1}(X \times \{0\})\,.
\end{equation*}
L'idea è quella di estendere l'identità $P_{0} = P_{0}$
successivamente a una bundle map
\begin{equation*}
	P_{n} := p^{-1}\Big(X \times \{0\} \cup (X_{n-1} \times I) \Big)\,,
\end{equation*}
dove $Y_{n}$ si ottiene da $Y_{n-1}$ attaccando delle celle $D^{n-1} \times I$
come pushout del diagramma:
\begin{equation*}
	\begin{tikzcd}
		\coprod \Big( \de D^{n-1} \times I \Big) \ar[r] \ar[d]
		& \coprod \Big( \de D^{n-1} \times I \Big) \ar[d] \\
		Y_{n-1} \ar[r] & Y_{n}\,.
	\end{tikzcd}
\end{equation*}
Cella per cella, vogliamo estendere la mappa da $Y_{n-1}$ a tutto $Y_{n}$:
\begin{equation*}
	\begin{tikzcd}
		\Big( (D^{n-1} \times \{0\}) \cup (\de D^{n} \times I) \Big) \times G \ar[rr] \ar[d]
		& & \xi\vert_{(D^{n-1} \times \{0\}) \cup (\de D^{n} \times I) } \ar[d, "h_{n-1}"] \\
		(D^{n-1} \times I) \times G \ar[r] \ar[rr, bend right, "\widetilde{h}_{n}"]
		& \xi\vert_{D^{n-1} \times I} \ar[r, "h_{n}", dashed]
		& \xi\vert_{D^{n-1} \times \{0\}}
	\end{tikzcd}
\end{equation*}

Devo trovare $\widetilde{h}$ che estende $h_{n-1} \circ d$.
Dato che abbiamo l'omeomorfismo di coppie
\begin{equation*}
	\beta: \Big(D^{n-1} \times I , (D^{n-1} \times \{0\}) \cup (\de D^{n} \times I) \Big) 
	\simeq \Big(D^{n-1} \times I, D^{n-1} \times \{0\} \Big)\,,
\end{equation*}
allora per la (\hyperref[HLP]{\textbf{HLP}}) possiamo sollevare la seguente omotopia
\begin{equation*}
	\begin{tikzcd}
		D^{n-1} \times \{0\} \ar[r, "h_{n-1}"]
		& P_{0} \ar[d] \\
		D^{n-1} \times I \ar[r, "\mathrm{pr}_{1} \circ \beta"] \ar[ur, dashed]
		& X \times \{0\}\,, 
	\end{tikzcd}
\end{equation*}
e quindi, facendo agire $G$, riusciamo a costruire una bundle map 
\begin{equation*}
	h_{n} : (D^{n-1} \times I) \times G \longrightarrow P_{0}\,,
	\quad \left(\overline{h}(x,t),g \right) \longmapsto \overline{h}(x,t) \cdot g\,.
\end{equation*}

\begin{cor}
	Se $f,g : X \to B$ sono mappe omotope, allora $f^{*}\xi \simeq g^{*}\xi$.
	Segue che la mappa $[X,B] \to \mathrm{Princ}_{G}(X)$ è ben definita.
\end{cor}

\begin{df}
	Un $G$-fibrato principale $p:E \to B$ si dice \textbf{universale} 
	se per ogni $G$-fibrato principale $\xi' = (E',p')$ su un CW complesso $X$,
	per ogni $L \subset X$ sottocomplesso e per ogni bundle map $h^{-1}(L) \to E$,
	allora $h$ si estende a una bundle map $h:E' \to E$.
	\begin{equation*}
		\begin{tikzcd}
			XXX
		\end{tikzcd}
	\end{equation*}
\end{df}

Nel linguaggio dei $G$-CW-complessi, la definizione è equivalente a
dare un fibrato principale su $G$ tale che per ogni $G$-CW-coppia $(E',A')$,
allora ogni mappa $G$-equivariante $A' \to E$ si estende a una mappa $h':E' \to E$.

\begin{thm}[\textbf{Teorema di classificazione}]
	Sia $\xi = (E,p)$ un $G$-fibrato universale su $B$.
	Per ogni CW complesso $X$, si ha una \textbf{bigezione}
	\begin{equation*}
		[X,B] \longrightarrow \mathrm{Princ}_{G}(X)\,,
		\quad [f] \longmapsto f^{*}\xi\,. 
	\end{equation*}
	\begin{proof}
		Siccome mappe omotope hanno pullback isomorfi,
		sappiamo che la mappa è ben definita.
		
		\textbf{Surgettività:} sia $p':E' \to X$ un fibrato principale su $G$.
		Per la definizione di $\xi$, prendendo il sottocomplesso $L = \emptyset$,
		per la Proposizione
		si ha una bundle map $h : E' \to E$ tale che $\xi' = \overline{h}^{*}\xi$,
		dove $\overline{h}:X \to B$ è indotta da $h$.
		
		\textbf{Iniettività:} siano $f_{0},f_{1}:X \to B$ tali che $f_{0}^{*}\xi \simeq f_{1}^{*}\xi$.
		Dato che il prodotto $X \times I$ è un CW complesso, consideriamo il suo sottocomplesso
		$L = X \times \{0,1\}$ e notiamo che il fibrato
		\begin{equation*}
			\begin{tikzcd}
				\xi' \ar[r, equals]
				& f_{0}^{*}\xi \times I \ar[r, " p'' "]
				& X \times I
			\end{tikzcd}
		\end{equation*}
		è isomorfo a $(f_{0} \circ \mathrm{pr}_{1})^{*}\xi$
		(questo ci dice che a ogni tempo stiamo parametrizzando sempre lo stesso fibrato pullback).
		
		Siano $h_{i}:f_{i}^{*}\xi 	\to \xi$ le due bundle map.
		Definiamo 
		\begin{equation*}
			h: \left( p'' \right)^{-1}(L) \longrightarrow E\,,
		\end{equation*}
		tale che $h\vert_{\left( p'' \right)^{-1}(X \times \{0\})} = h_{0}$
		e $h\vert_{\left( p'' \right)^{-1}(X \times \{1\})} = h_{1} \circ \kappa$,
		dove $\kappa : E(f_{0}^{*}\xi) \to E(f_{1}^{*}\xi)$ deriva
		dall'ipotesi $f_{0}^{*}\xi \simeq f_{1}^{*}\xi$.
		Quindi $h$ è una bundle map che su $X \times \{1\}$
		induce la stessa mappa di $h_{1}$, cioè $f_{1}$.
		
		Per universalità di $\xi$, sappiamo che $h$ si estende a una bundle map $h':E(\xi') \to E$,
		e l'applicazione indotta da $h'$ dà un'omotopia
		\begin{equation*}
			\overline{h'} : X \times I \longrightarrow B\,,
			\quad \overline{h'}_{0} = f_{0},\, \overline{h'}_{1} = f_{1}\,.
		\end{equation*}
	\end{proof}
\end{thm}

\begin{thm}
	Se $\xi$ è un $G$-fibrato principale con spazio totale $E(\xi)$
	debolmente contraibile, allora è universale.
\end{thm}