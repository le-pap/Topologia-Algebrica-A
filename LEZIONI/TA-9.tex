 %%% Lezione 9

\section{Spazi classificanti}

\lecture[Lezione 9.]{2023-03-27}

Ricordiamo che il \hyperref[vect-htp]{Teorema~\ref{vect-htp}}
ci dà importante risultato di classificazione dei fibrati vettoriali
\emph{a meno di omotopia}: infatti, esso afferma che 
mappe omotope producono pullback isomorfi.
Avevamo lasciato la dimostrazione in sospeso, 
ma alla luce dell'isomorfismo 
$\mathrm{Vect}_{n} \simeq \mathrm{Princ}_{\mathrm{GL}(n,\R)}$
notiamo che l'enunciato segue come caso particolare del seguente
\begin{thm}\label{princ-I-invariante}
	Dato $G$ un gruppo topologico qualsiasi,
	il funtore $\mathrm{Princ}_{G}$ è $I$-invariante nella categoria dei CW complessi, 
	cioè per ogni CW complesso $X$ % \todo{In realtà lo vediamo per un CW complesso} 
	si ha una bigezione naturale
	\begin{equation*}
		\mathrm{pr}_{1}^{*} : \mathrm{Princ}_{G}(X) \simeq \mathrm{Princ}_{G}(X \times I)\,.
	\end{equation*}
\end{thm}
Lo scopo di oggi è sviluppare gli strumenti adatti a dimostrare questo risultato.

\begin{lemma}\label{contraibile-banale}
	Se $X$ è uno spazio contraibile, allora $\mathrm{Princ}_{G}(X) = \Set{X \times G}$.
	In altre parole, il fibrato banale è l'unico fibrato principale su uno spazio contraibile.
	\begin{proof}
		 È sufficiente costruire una sezione di $p:P \to X$.
			Dato $x_{0} \in X$, sappiamo che $\cat{1}_{X}$ è omotopa alla mappa costante in $x_{0}$;
			preso un punto $q \in p^{-1}(x_{0})$, solleviamo la costante a una mappa $c:X \to P$
			tale che $p(q)=x_{0}$.
			\begin{equation*}
				\begin{tikzcd}
					& P \ar[d, "p"] \\
					X \ar[r, "x_{0}"'] \ar[ur, "c"] & X\,.
				\end{tikzcd}
			\end{equation*}
			Presa $H: X \times I \to X$ un'omotopia di contrazione tale che
			 $H_{0} \equiv x_{0}$ e $H_{1} = \cat{1}_{X}$, 
			 siccome $p$ è una fibrazione, 
			 allora $H$ si solleva a $\widetilde{H} : X \times I \to P$ tale
			 che $\widetilde{H}_{0}=c$ e $\widetilde{H}_{1}$ solleva l'identità di $X$.
			 Questo significa che $\widetilde{H}_{1}$ è una sezione di $p$,
			 perciò il fibrato è $P \simeq X \times G$. 
	\end{proof}
\end{lemma}

\begin{df}
	Chiamiamo \textbf{$G$-CW-coppia} una coppia $(X,A)$ di CW complessi
	con un'azione (destra) di $G$ su $X$ tale che esista una filtrazione $G$-invariante
	\begin{equation*}
		A =: F^{-1}X \subset F^{0}X \subset F^{1}X \subset \dots  
		\subset F^{n}X \subset \dots
		\subset \bigcup_{n \ge 0} F^{n}X = X
	\end{equation*}
	tale che $F^{n}X$ è ottenuto da $F^{n-1}X$ attaccando delle ``\emph{celle equivarianti}'' 
	del tipo
	% \begin{equation*}
		$D^{n} \times \left( H \backslash G \right)$,
	% \end{equation*}
	dove $H \subset G$ è sottogruppo, $H \backslash G = \{Hg | g \in G\}$
	e con azione di $G$ sulla destra:
	possiamo vedere questa costruzione come il pushout del seguente diagramma:
	\begin{equation*}
	 	\begin{tikzcd}
	 		\coprod_{i} \de D_{i}^{n} \times \left( H_{i} \backslash G \right) 
	 		\ar[r] \ar[d, "\coprod \phi_{i}"'] 
	 		& \coprod_{i} D_{i}^{n} \times \left( H_{i} \backslash G \right) 
	 		\ar[d, "\coprod \psi_{i}"]  \\
	 		F^{n-1}X \ar[r] & F^{n}X\,,
	 	\end{tikzcd}
	 \end{equation*}
	 dove $\phi_{i}$ è la mappa di incollamento del disco $D^{n}_{i}$.
	 Lo spazio $X$ così ottenuto ha la topologia del limite diretto (o \emph{topologia debole}). 
	 Se $A=\emptyset$, chiameremo $X$ un \textbf{$G$-CW-complesso}.
\end{df}

%%\begin{oss}
%%	Banalmente, ogni CW complesso è un $G$-CW-complesso con $G=1$.
%%	Se $G$ è un gruppo con dimensione $> 0$, allora 
%%\end{oss}

\begin{oss}
	Se $\xi=(P,p)$ è un $G$-fibrato principale su un CW complesso $X$,
	allora lo scheletro di $X$ si solleva a un $G$-CW-complesso su $P$,
	con sottogruppo $H=\{1\}$.
	Infatti, se $c:D^{n} \to X$ è la funzione caratteristica di una $n$-cella di $X$,
	allora $c^{*}\xi$ è un fibrato banale perché $D^{n}$ è contraibile;
	quindi abbiamo il quadrato cartesiano
	\begin{equation*}
		\begin{tikzcd}
		D^{n} \times G = c^{*}\xi \ar[r, "\widetilde{c}"] \ar[d]
		& P \ar[d, "p"] \\
		D^{n} \ar[r, "c"] & X\,,
		\end{tikzcd}
	\end{equation*}
	che descrive $P$ come $G$-CW-complesso.
\end{oss}

%Se $\xi:P \xrightarrow{p} X$ è un fibrato principale su $G$, dove $X$ è un CW complesso,
%lo scheletro di $X$ si solleva a un $G$-CW-complesso su $P$:
%\begin{equation*}
%	\begin{tikzcd}
%		E(c^{*} \xi) \ar[r, "\widetilde{c}"] \ar[d] & P \ar[d] \\
%		D^{n} \ar[r, "c"] & X\,,
%	\end{tikzcd}
%\end{equation*}
%dove $c^{*}\xi$ è banale, con spazio totale $D^{n} \times G$.

\begin{df}
	Dati due fibrati $\xi = (E,p)$ su $B$ e  $\xi'=(E',p')$ su $B'$ con stessa fibra $F$,
	chiamiamo \textbf{bundle map} di $\xi$ su $\xi'$ un'applicazione continua
	$h:E \to E'$ che manda ogni fibra $E_{b}$ omeomorficamente in una fibra $E'_{b'}$.
	Come conseguenza, questa mappa induce $\overline{h} : B \to B'$ che rende commutativo il diagramma
	\begin{equation*}
		\begin{tikzcd}
			E \ar[d, "p"'] \ar[r, "h"] & E' \ar[d, "p'"] \\
			B \ar[r, "\overline{h}"] & B'\,.
		\end{tikzcd}
	\end{equation*}
	Se $\xi$ e $\xi'$ sono fibrati vettoriali, richiediamo che
	$h_{b}:E_{b} \to E'_{b'}$ sia un isomorfismo lineare.
	Se $\xi$ e $\xi'$ sono fibrati principali sullo stesso gruppo $G$,
	richiediamo che $h_{b}$ sia $G$-equivariante, i.e. $h_{b}(e \cdot g) = h_{b}(e) \cdot g$.
\end{df}


\begin{prop}\label{bundle-map-pullback}
	Se $h:\xi \to \xi'$ è una bundle map,
	allora c'è un isomorfismo di fibrati $g:\xi \simeq \overline{h}^{*}\xi'$
	tale che $h=\mathrm{pr}_{2} \circ g$.
	\begin{equation*}
		\begin{tikzcd}
			E(\xi) \ar[r, "\simeq"] \ar[d, "p"']
			& E(\overline{h}^{*}\xi') \ar[r, "\mathrm{pr}_{2}"] \ar[d]
			& E(\xi') \ar[d, "p'"] \\
			B \ar[r, equals] 
			& B \ar[r, "\overline{h}"]
			& B'\,.
		\end{tikzcd}
	\end{equation*}
	Se inoltre $\xi$ e $\xi'$ sono fibrati principali su $G$,
	allora gli isomorfismi sono $G$-equivarianti.
	\begin{proof}
		Definiamo $g: E(\xi) \to E(\overline{h}^{*}\xi')$ come
		\begin{equation*}
			g(e) = \left( p(e), h(e) \right)\,.
		\end{equation*}
		Si verifica che $g$ è continua e manda ciascuna fibra $E_{b}$
		isomorficamente su %lle fibre 
		$E({\overline{h}^{*}\xi')_{b}}$,
		cioè è bigettiva con inversa continua (è sufficiente verificarlo
		localmente sulle banalizzazioni),
		e quindi concludiamo che è un isomorfismo.
	\end{proof}
\end{prop}

In parole povere, avere una bundle map è equivalente
a dire che il fibrato che stiamo considerando è in realtà un pullback.
Finalmente siamo pronti per dimostrare il l'$I$-invarianza di $\mathrm{Princ}_{G}$.

	\begin{proof}[Dimostrazione del Teorema~\ref{princ-I-invariante}]
		La parte facile consiste nel dimostrare l'\textbf{iniettività} di $\mathrm{pr}_{1}^{*}$:
		dato che $\mathrm{pr}_{1}$ ammette la sezione 
		\begin{equation*}
			\iota_{0} : X \longrightarrow X \times I\,, \quad
			x \longmapsto (x,0)\,,
		\end{equation*}
		allora per funtorialità vediamo che 
		%\begin{equation*}
		$	\iota_{0}^{*} \circ \mathrm{pr}_{1}^{*} = \cat{1}_{\mathrm{Vect}(X)} $,
		%\end{equation*}
		da cui segue l'iniettività.
		
		La \textbf{surgettività} è più difficile, e la dividiamo in casi.
		\begin{itemize}
			\item \textbf{Caso semplice: sia $X$ contraibile.} 
			Siccome sia $X$, sia $X \times I$ sono contraibili, 			
			per il \hyperref[contraibile-banale]{Lemma~\ref{contraibile-banale}},
			sappiamo che su di essi esiste il solo fibrato banale,
			per cui concludiamo che la proiezione induce necessariamente una bigezione
			 \begin{equation*}
			 	 \mathrm{Princ}_{G}(X) \simeq \{ \ast \} \simeq \mathrm{Princ}_{G}(X \times I)\,.
			 \end{equation*}
			 
			 \item \textbf{Caso generale: sia $X$ CW complesso.}
			 %Dire che $\mathrm{pr}_{1}^{*}:\mathrm{Princ}_{G}(X) \to \mathrm{Princ}_{G}(X  \times I)$ è surgettiva, quindi 
Vogliamo dimostrare che ogni $G$-fibrato principale 
$\xi = (P,p)$ su $X \times I$
è della forma $\mathrm{pr}_{1}^{*}(\xi')$,
cioè il pullback di qualche fibrato $\xi = (E',p')$ su $X$, quindi
\begin{equation*}
	P \simeq \Set{ \big( (x,t),e' \big) | x = p'(e') }\,.
\end{equation*}
Identificando $X$ con $X \times \{0\} \subset X \times I$,
quello che vorremmo dimostrare è che
 $$P = P\vert_{X \times \{0\}} \times I : (e,t) \to (p(e),t) \in X \times I \,,$$
cioè che a ogni tempo $t \in I$, la fetta $P\vert_{X \times \{t\}}$ è una
copia di $P\vert_{X \times \{0\}}$.
Quindi basta dimostrare che ogni fibrato $\xi$ su $X \times I$
è il pullback lungo la restrizione a $X \times \{0\}$ tramite $\pi_{0} : (x,t) \mapsto (x,0)$.

Per la \hyperref[bundle-map-pullback]{Proposizione~\ref{bundle-map-pullback}}, 
ci basta trovare una bundle map
\begin{equation*}
	h : P \to P_{0}:=p^{-1}(X \times \{0\})\,.
\end{equation*}
L'idea è quella di estendere l'identità $P_{0} = P_{0}$
per passi successivi a una bundle map di 
\begin{equation*}
	h_{n} := p^{-1}(Y_{n}) \longrightarrow P_{0}\,,
\end{equation*}
dove $Y_{n} := (X \times \{0\}) \cup (X_{n-1} \times I)$ 
si ottiene da $Y_{n-1}$ attaccando delle celle $D^{n-1} \times I$,
come descritto dalla costruzione pushout % del diagramma:
\begin{equation*}
	\begin{tikzcd}
		\coprod_{i} \Big( (D_{i}^{n-1} \times \{0\}) \cup (\de D_{i}^{n-1} \times I) \Big) \ar[r] \ar[d]
		& \coprod_{i} \Big( \de D_{i}^{n-1} \times I \Big) \ar[d] \\
		Y_{n-1} \ar[r] & Y_{n}\,.
	\end{tikzcd}
\end{equation*}
Cella per cella, vogliamo estendere la mappa da $Y_{n-1}$ a tutto $Y_{n}$:
dette $\alpha$ e $\alpha'$ banalizzazioni sui sottocomplesi opportuni,
supponendo di avere già costruito $h_{n-1}$ abbiamo il diagramma
\begin{equation*}
	\begin{tikzcd}
		\Big( (D^{n-1} \times \{0\}) \cup (\de D^{n-1} \times I) \Big) \times G 
		\ar[rr, "\alpha"] \ar[d, hook]
		& & \xi\vert_{(D^{n-1} \times \{0\}) \cup (\de D^{n} \times I) } 
		\ar[dl, hook] \ar[d, "h_{n-1}"] \\
		(D^{n-1} \times I) \times G \ar[r, "\alpha'"] \ar[rr, bend right=14, "\widetilde{h}_{n}"']
		& \xi\vert_{D^{n-1} \times I} \ar[r, "h_{n}", dashed]
		& \xi\vert_{D^{n-1} \times \{0\}}
	\end{tikzcd}
\end{equation*}

Devo trovare $\widetilde{h}_{n}$ che estende $h_{n-1} \circ \alpha$.
Dato che esiste un'omeomorfismo di coppie
\begin{equation*}
	\beta: \Big(D^{n-1} \times I, D^{n-1} \times \{0\} \Big) \simeq
	\Big(D^{n-1} \times I , (D^{n-1} \times \{0\}) \cup (\de D^{n} \times I) \Big) \,,
\end{equation*}
allora per la (\hyperref[HLP]{\textbf{HLP}}) possiamo sollevare la seguente omotopia
\begin{equation*}
	\begin{tikzcd}
		D^{n-1} \times \{0\} \ar[r, "h'_{n-1}"] \ar[d, hook]
		& P_{0} \ar[d] \\
		D^{n-1} \times I \ar[r, "\pi_{0} \circ \beta"'] \ar[ur, "\overline{h}_{n-1}"', dashed]
		& X \times \{0\}\,, 
	\end{tikzcd}
\end{equation*}
dove $h'_{n-1}(x,0) = (\beta(x,0),1_{G})$.
Facendo agire $G$, riusciamo dunque a costruire una bundle map 
\begin{equation*}
	h_{n} : (D^{n-1} \times I) \times G \longrightarrow P_{0}\,,
	\quad \left((x,t),g \right) \longmapsto \overline{h}_{n}(x,t) \cdot g\,. \qedhere
\end{equation*}
		\end{itemize}
	\end{proof}

\begin{cor}
	Se $f,g : X \to B$ sono mappe omotope, allora $f^{*}\xi \simeq g^{*}\xi$.
	Segue che la funzione $[X,B] \to \mathrm{Princ}_{G}(X)$ è ben definita.
\end{cor}

\begin{df}
	Un $G$-fibrato principale $\xi:E \xrightarrow{p} B$ si dice \textbf{universale} 
	se per ogni $G$-fibrato principale $\xi' = (E',p')$ su un CW complesso $X$,
	per ogni $L \subset X$ sottocomplesso e per ogni bundle map $h:(p')^{-1}(L) \to E$,
	allora $h$ si estende a una bundle map $h':E' \to E$.
	\begin{equation*}
		\begin{tikzcd}
			& E' \ar[rrd, "h", dashed] \ar[dd, "p'", near end] & & \\
			(p')^{-1}(L) \ar[dd] \ar[ur, hook] \ar[rrr, crossing over] & & & E \ar[dd,"p"] \\
			& X \ar[rrd, "\overline{h}", dashed] & & \\
			L \ar[ur, hook] \ar[rrr] & & & B\,.
		\end{tikzcd}
	\end{equation*}
\end{df}

Nel linguaggio dei $G$-CW-complessi, la definizione è equivalente a
dare un fibrato principale su $G$ tale che per ogni $G$-CW-coppia $(E',A')$,
allora ogni mappa $G$-equivariante $A' \to E$ si estende a una mappa $h':E' \to E$.

\begin{thm}[\textbf{di classificazione}]\label{teo-classificazione}
	Sia $\xi = (E,p)$ un $G$-fibrato universale su $B$.
	Per ogni CW complesso $X$, si ha una \textbf{bigezione}
	\begin{equation*}
		[X,B] \longrightarrow \mathrm{Princ}_{G}(X)\,,
		\quad [f] \longmapsto f^{*}\xi\,. 
	\end{equation*}
	\begin{proof}
		Siccome mappe omotope hanno pullback isomorfi,
		sappiamo che la mappa è ben definita.
		
		\textbf{Surgettività:} sia $p':E' \to X$ un fibrato principale su $G$.
		Prendendo il sottocomplesso $L = \emptyset$ nella definizione di $\xi$,
		per la \hyperref[bundle-map-pullback]{Proposizione~\ref{bundle-map-pullback}}
		si ha una bundle map $h : E' \to E$ tale che $\xi' = \overline{h}^{*}\xi$,
		dove $\overline{h}:X \to B$ è indotta da $h$.
		
		\textbf{Iniettività:} siano $f_{0},f_{1}:X \to B$ tali che $f_{0}^{*}\xi \simeq f_{1}^{*}\xi$.
		Dato che il prodotto $X \times I$ è un CW complesso, consideriamo il suo sottocomplesso
		$L = X \times \{0,1\}$ e notiamo che il fibrato
		\begin{equation*}
			\begin{tikzcd}
				\xi' : f_{0}^{*}\xi \times I \ar[r, " p'' "]
				& X \times I
			\end{tikzcd}
		\end{equation*}
		è isomorfo a $(f_{0} \circ \mathrm{pr}_{1})^{*}\xi$, con $i =0,1$
		(questo ci dice che a ogni tempo stiamo parametrizzando sempre lo stesso fibrato pullback).
		Poste $h_{i}:f_{i}^{*}\xi 	\to \xi$ le due bundle map,
		definiamo la bundle map $h: \left( p'' \right)^{-1}(L) \to E$
%		\begin{equation*}
%			h: \left( p'' \right)^{-1}(L) \longrightarrow E\,,
%		\end{equation*}
%		tale che $h\vert_{\left( p'' \right)^{-1}(X \times \{0\})} = h_{0}$
%		e $h\vert_{\left( p'' \right)^{-1}(X \times \{1\})} = h_{1} \circ \kappa$,
		tale che
		\begin{equation*}
			h\vert_{\left( p'' \right)^{-1}(X \times \{0\})} = h_{0}\,, \quad
			h\vert_{\left( p'' \right)^{-1}(X \times \{1\})} = h_{1} \circ \kappa\,,
		\end{equation*}
		dove $\kappa : E(f_{0}^{*}\xi) \to E(f_{1}^{*}\xi)$ deriva
		dall'ipotesi $f_{0}^{*}\xi \simeq f_{1}^{*}\xi$.
		Quindi $h$ è una bundle map che su $X \times \{1\}$
		induce la stessa mappa di $h_{1}$, cioè $f_{1}$.
		
		Per universalità di $\xi$, sappiamo che $h$ si estende a una bundle map $h':E(\xi') \to E$,
		e l'applicazione indotta da $h'$ dà un'omotopia
		\begin{equation*}
			\overline{h'} : X \times I \longrightarrow B\,,
			\quad \overline{h'}_{0} = f_{0},\, \quad \overline{h'}_{1} = f_{1}\,. \qedhere
		\end{equation*}
	\end{proof}
\end{thm}

%\begin{thm}
%	Se $\xi$ è un $G$-fibrato principale con spazio totale $E(\xi)$
%	debolmente contraibile, allora è universale.
%\end{thm}