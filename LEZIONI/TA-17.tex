 %%% Lezione 17


\lecture[Descrizione della mappa di trasgressione nella successione spettrale di Serre. Classi di Serre di gruppi abeliani, anelli e ideali di Serre, anelli aciclici. Esempi. Verifica dell'aciclicità di alcune classi. Classi di Serre e successione spettrale di Serre. Enunciato del teorema di Hurewicz modulo classi di Serre.]{2023-05-08}

Come al solito, consideriamo una fibrazione $F \to E \to B$,
con $B$ e $F$ spazi connessi per archi.
Consideriamo una sistema locale $H_{*}(\pi^{-1})$ banale\todo{Devo scrivere bene}

\begin{df}
	Chiamiamo \textbf{trasgressione} l'omomorfismo
	$d^{n}:E^{n}_{n,0} \to E^{n}_{0,n-1}$.
	Un elemento $x \in E^{2}_{n,0}$ si dice \textbf{trasgressivo}.
\end{df}

Quindi gli elementi trasgressivi sono quelli che sopravvivono sino alla pagina $E^{n}_{n,0}$.

\begin{rmk}
	Consideriamo il caso $n=2$. In questo caso abbiamo
	\begin{equation*}
		d^{2}: H_{2}(B) \simeq E^{2}_{2,0} \to E^{2}_{0,1} \simeq H_{1}(F)\,.
	\end{equation*}
	In generale abbiamo $E^{n}_{n,0} \subset E^{2}_{n,0} \simeq H_{n-1}(F)$.
	Consideriamo la succesione esatta lunga
	\begin{equation*}
		\begin{tikzcd}
			\dots \ar[r]
			& H_{m}(F) \ar[r]
			& H_{m}(E) \ar[r]
			& H_{m}(E,F) \ar[r, "\de_{*}"]
			& H_{m-1}(F) \ar[r]
			& \dots 
		\end{tikzcd}
	\end{equation*}
	e la proiezione $\pi_{*}:H_{*}(E,F) \to H_{*}(B,b_{0})$.
\end{rmk}

Il seguente risultato ci permette di dare un'interpretazione a queste trasgressioni.
In particolare, ci permettono di trovare una costruzione funtoriale.

\begin{thm}
	La trasgressione nella successione di Serre in omologia coincide con
	la composizione
	\begin{equation*}
		\begin{tikzcd}
			H_{m}(B) \ar[r]
			& H_{m}(B,b_{0}) \ar[r, dashed, "\pi_{*}^{-1}"]
			& H_{m}(E,F) \ar[r, "\de_{*}"]
			& H_{m-1}(F)\,.
		\end{tikzcd}
	\end{equation*}
	In maniera duale, 
	la trasgressione nella successione di Serre in coomologia coincide con
	la composizione
	\begin{equation*}
		\begin{tikzcd}
			H^{m-1}(F) \ar[r, "\delta^{*}"]
			& H_{m}(E,F) \ar[r, dashed, "(\pi_{*})^{-1}"]
			& H_{m}(B,b_{0}) \ar[r]
			& H_{m}(B)\,.
		\end{tikzcd}
	\end{equation*}
	\begin{proof}
		Dimostriamo solamente il caso dell'omologia.
		A meno di approssimazione cellulare, supponiamo che $B$ sia un CW complesso;
		inoltre, siccome $B$ è connesso per archi, possiamo supporre
		che abbia una soa $0$-cella.
		Un elemento $x \in E^{n}_{n,0}$ è rappresentato da una xatena
		\begin{equation*}
			c \in C_{n}\left( \pi^{-1}(B^{(n)}) \right) \subset C_{n}(E)\,,
		\end{equation*}
		con bordo
		\begin{equation*}
			\de c \in C_{n-1}\left( \pi^{-1}(B^{(0)}) \right) 
			= C_{n-1}(F)\,.
		\end{equation*}
		In altre parole, $c$ è un ciclo relativo in
		\begin{equation*}
			C_{\bullet}\left( \pi^{-1}(B^{(n)}), F \right)\,.
		\end{equation*}
		L'identificazione di $E^{n}_{n,0}$ con $H_{n}(B)$
		si ha tramite la mappa
		\begin{equation*}
		\begin{tikzcd}
			E^{n}_{n,0} \ar[r]
			& H_{n}\left( \pi^{-1}(B^{(n)}), F \right) \ar[r, dashed, "\pi_{*}"]
			& H_{*}(B)\,,
			& {[c]} \ar[r, mapsto] & {[\pi_{\#}c]}\,,
		\end{tikzcd}
		\end{equation*}
		e inoltre $d^{n}:E^{n}_{n,0} \to E^{n}_{0,n-1}$ manda
		\begin{equation*}
			H_{n}\left( \pi^{-1}(B^{(n)}), F \right) 
			\longrightarrow H_{n-1}(F)\,, \quad
			{[c]} \longmapsto {[\de c]}\,.
		\end{equation*}
	\end{proof}
\end{thm}





\section{Classi di Serre}

Sia $X$ uno spazio topologico semplicemente connesso.
Per i teoremi di Hurewicz, se conosciamo $\overline{H}_{*}(X;\Z)$,
allora conosciamo anche l'omotopia $\pi_{*}(X,\ast)$.
In particolare, i coefficienti in $\Z$ ci danno il vantaggio di
conoscere bene i gruppi in questione, dato che
abbiamo a disposizione il \textbf{Teorema di struttura per gruppi abeliani}.
Ma se invece considerassimo i coefficienti in un campo,
ad esempio $\overline{H}_{*}(X;\Q)$, che cosa possiamo concludere
per $\pi_{*}(X,\ast) \otimes \Q$? Se l'omologia è un gruppo finito, 
allora possiamo dedurre qualcosa sui gruppi di omotopia?

\begin{df}
	Una \textbf{classe di Serre $\Cc$} di gruppi abeliani
	è una sottocategoria piena di $\Ab$ che contiene il gruppo banale $\cat{0} \in \Cc$,
	e per ogni successione esatta corta
	\begin{equation*}
		\begin{tikzcd}
			\cat{0} \ar[r] & A \ar[r] & B \ar[r] & C \ar[r] & \cat{0}\,,
		\end{tikzcd}
	\end{equation*}
	allora $A$ e $C$ appartengono a $\Cc$ se e solo se $B \in \Cc$.
\end{df}

\begin{oss}
	Una classe di Serre è chiusa per le seguenti operazioni:
	\begin{rmnumerate}
		\item isomomorfismo;
		\item sottogruppi;
		\item quozienti;
		\item estensioni.
	\end{rmnumerate}
	Inoltre, l'intersezione di classi di Serre
	è a sua volta una classe di Serre.
\end{oss}

\begin{ex}
 	Alcune classi di Serre interessanti sono:
 	\begin{rmnumerate}
 		\item i gruppi abeliani finiti $\Ab_{fin}$;
 		\item i gruppi abeliani finitamente generati $\Ab_{fg}$;
 		\item i gruppi abeliani di torsione $\Ab_{tor}$.
 		\item i gruppi abeliani di $p$-torsione $\Ab_{p-tor}$,
 		cioè quei gruppi $A$ tali che, per ogni $a \in A$, esiste $n \in \N$
 		tale che $p^{n}a=0$.
 	\end{rmnumerate}
\end{ex}

\begin{df}
	Sia $\Pp$ un insieme di primi in $\Z$. Indichiamo con $\Cc_{\Pp}$
	la classe dei gruppi abeliani di torsione tali che,
	se esiste $p \in \Pp$ tale che $p \vert \mathrm{ord}(a)$ per qualche $a \in A$,
	allora $A = \cat{0}$.
	Nel caso $\Pp = \{p\}$, allora scriviamo $\Cc_{p} := \Cc_{\{p\}}$.
\end{df}

Si verifica che $\Cc_{\Pp}$ è una classe di Serre.
Sia $\Z_{\Pp}$ la localizzazione di $\Z$ al complementare
dell'unione degli ideali $(p)$, al variare di $p \in \Pp$,
cioè
\begin{equation*}
	\Z_{\Pp} = \left( \Z \setminus \bigcup_{p \in \Pp} (p) \right)^{-1} \Z
	= \Set{\frac{a}{b} | a \in \Z, b \notin (p)\text{ per ogni } p \in \Pp}\,.
\end{equation*}

\begin{rmk}
	Si ha $A \in \Cc_{\Pp}$ se e solo se $A \otimes \Z_{\Pp} = 0$.  
\end{rmk}


\begin{df}
	Diciamo che $A=\cat{0} \mathrm{mod} \Cc$ se $A \in \Cc$.
	In maniera simile, diciamo che un omomorfismo
	di gruppi $f:A \to B$ è un \textbf{monomorfismo} $\mathrm{mod} \Cc$
	se $\ker f \in \Cc$, e invece è un \textbf{epimorfismo} $\mathrm{mod} \Cc$
	se $\mathrm{coker} f \in \Cc$. In particolare,
	$f$ è un isomorfismo $\mathrm{mod} \Cc$ se
	$\ker f, \mathrm{coker} f \in \Cc$.
\end{df}

\begin{prop}
	Sia $\Cc$ una classe di Serre, allora le classi
	di monomorfismi, epimorfismi e isomorfismi $\mathrm{mod} \Cc$
	sono chiuse per composizione.
	Più precisamente, dati $\alpha : A \to B$ e $\beta : B \to C$,
	se due mappe tra $\alpha, \beta, \beta\alpha$ sono
	mono/epi/iso $\mathrm{mod} \Cc$, allora anche il terzo lo è.
	\begin{proof}
		Tutte le affermazioni seguono dall'esattezza
		della successione di mappe tratteggiate nel diagrammone
		pazzo sgravato qui sotto:
			\begin{equation*}
			\begin{tikzcd}[column sep=small]
                     &                                                  & \mathbf{0} \arrow[rd]                     &                                                   &                                  &                                                             & \mathbf{0}                                               &                                                               &            \\
                     &                                                  &                                           & \ker \beta \arrow[rd] \arrow[rr, dashed]          &                                  & \operatorname{coker} \alpha \arrow[rrdd, dashed] \arrow[ru] &                                                          &                                                               &            \\
                     &                                                  &                                           &                                                   & B \arrow[rd, "\beta"] \arrow[ru] &                                                             &                                                          &                                                               &            \\
\mathbf{0} \arrow[r] & \ker \beta\alpha \arrow[rr] \arrow[rruu, dashed] &                                           & A \arrow[ru, "\alpha"] \arrow[rr, "\beta \alpha"] &                                  & C \arrow[rd] \arrow[rr]                                     &                                                          & \operatorname{coker} \beta\alpha \arrow[r] \arrow[ld, dashed] & \mathbf{0} \\
                     &                                                  & \ker \alpha \arrow[ru] \arrow[lu, dashed] &                                                   &                                  &                                                             & \operatorname{coker} \beta \arrow[rd] \arrow[ld, dashed] &                                                               &            \\
                     & \mathbf{0} \arrow[ru]                            &                                           & \mathbf{0} \arrow[lu, dashed]                     &                                  & \mathbf{0}                                                  &                                                          & \mathbf{0}                                                    &           
\end{tikzcd}
			\end{equation*}
	\end{proof}
\end{prop}	

\begin{exercise}
	Possiamo parlare di successioni esatte corte $\mathrm{mod} \Cc$
	e vale il \textbf{Lemma dei $5$} $\mathrm{mod} \Cc$.
\end{exercise}

\begin{rmk}
	Se $C_{\bullet}$ è un complesso di catene con tutti i termini $C_{n}$ 
	in una classe di Serre $Cc$, allora $H_{n}(C_{\bullet}) \in \Cc$, per ogni $n \in \Z$.
	Inoltre, se $F$ è una filtrazione di un gruppo abeliano $A \in \Cc$,
	allora per ogni $s \in \Z$ anche $\mathrm{gr}_{s}A \in \Cc$.
	Viceversa, se la filtrazione $F$ è finita e ogni
	$\mathrm{gr}_{s}A \in \Cc$, allora tutto il gruppo $A$ sta in $\Cc$.
\end{rmk}	

In particolare, sia $\{E^{r}_{s,t}\}$ una successione spettrale.
Se $E^{r}_{s,t} \in \Cc$ per ogni $s,t$, allora tutte le pagine
$E^{r}_{s,t}$ appartengono alla classe $\Cc$ per $r \ge 2$.
Se la successione è concentrata nel primo quadrante,
allora anche il limite $E^{\infty}_{s,t}$ è in $\Cc$.
Se $\{E^{r}_{s,t}\}$ è indotta da una filtrazione 
su un complesso $C_{\bullet}$ e $E^{r}_{s,t} \in \Cc$ per ogni $s+t=n$,
allora anche l'omologia $H_{*}(C_{\bullet}) \in \Cc$.

\begin{ex}
	Se $A \subset X$ e due tra $\overline{H}_{n}(X), \overline{H}_{n}(A), {H}_{n}(X,A)$
	sono in $\Cc$, allora anche il terzo appartiene alla classe $\Cc$.
\end{ex}

\begin{df}
	Una classe di Serre $\Cc$ è un \textbf{anello di Serre}
	se per ogni coppia di gruppi $A, B \in \Cc$, allora anche 
	$A \otimes B$ e $A \ast B := \Tor(A,B)$ sono in $\Cc$.
\end{df}

\begin{df}
	Una classe di Serre $\Cc$ si dice \textbf{ideale di Serre}
	se, dato $A \in \Cc$, per ogni gruppo abeliano $B$
	allora $A \otimes B, A \ast B \in \Cc$.
\end{df}

\begin{ex}
	Gli esempi di classi di Serre presentati in precedenza sono tutti anelli di Serre.
	Senza l'ipotesi ``finitamente generati'' in realtà sono ideali di Serre.
\end{ex}

Sia $\Cc$ una classe di Serre. Preso un gruppo $A \in \Cc$,
sappiamo che lo spazio classificante
\begin{equation*}
	BA = K(A,1)
\end{equation*}
ha omologia in grado $1$ data da $H_{1}(K(A,1)) = A \in \Cc$.
Motivati da questo esempio, diamo la seguente

\begin{df}
	Un anello di Serre $\Cc$ si dice \textbf{aciclico}
	se, per ogni $A \in \Cc$, lo spazio classificante di $A$
	è aciclico $\mathrm{mod}\Cc$, cioè
	\begin{equation*}
	 	\overline{H}_{n}(K(A,1)) \in \Cc\,, \quad n \in \N\,.
	 \end{equation*} 
\end{df}

\begin{ex}
	Verifichiamo che la classe $\Ab_{fin}$ dei gruppi abeliani finiti 
	è un anello di Serre aciclico.
	Ricordiamo che la formula di Künneth ci dà
	\begin{equation*}
		K(A \ast B,1) = K(A,1) \times K(B,1)\,.
	\end{equation*}
	Dato $C_{n}$ un gruppo ciclico di ordine $n$, 
	allora possiamo immergerlo $C_{n} \subset S^{1}$,
	che agisce a sinistra su $S^{\infty} \subset \C\PP^{\infty}$.
	Dunque deduciamo che
	\begin{equation*}
		BC_{n} = C_{n }\backslash S^{\infty}= \left(  C_{n }\backslash S^{1}\right) \times_{S^{1}} S^{\infty}\,,
	\end{equation*}
	e quindi la fibrazione 
	$$S^{1} \simeq C_{n} \backslash S^{1} \hookrightarrow S^{1} \backslash S^{\infty}
	\simeq \C \PP^{\infty}$$
	ci dà una successione spettrale di Serre
	dalla quale leggiamo che $H_{1}(BC_{n}) = C_{n}$ e $H^{2}(BC_{n})=C_{n}$.
	Questo implica che $d_{2}(e) = \pm n x$ e deduciamo che 
	l'anello di coomologia di $BC_{n}$ è 
	\begin{equation*}
		H^{*}(BC_{n}) = \Z[x]/(nx)\,, \quad \text{con } |x| = 2\,.
	\end{equation*}
	Questo mostra che l'omologia e la coomologia di $BC_{n}$
	sono in $\Ab_{fin}$, e quindi $\Ab_{fin}$ è aciclico.
\end{ex}

\begin{ex}
	Se $A \in \Ab_{tor}$, allora $A$ è un limite diretto di gruppi finiti.
	Siccome tutti i suoi sottogruppi sono di torsione,
	scrivendo anche $K(A,1)$ come limite diretto
	possiamo vedere che $\overline{H}_{q}(K(A,1))$ è di torsione,
	quindi $\Ab_{tor}$ è un anello di Serre aciclico.
	Per un ragionamento analogo, 
	anche $\Ab_{p-tor} \cap \Ab_{fin}$ e $\Cc_{p}$ sono aciclici.
\end{ex}

\begin{ex}
	Dato che $K(\Z,1) = S^{1}$, allora usando il \textbf{Teorema di struttura}
	e gli esempi precedenti notiamo che anche $\Ab_{fg}$ è aciclico.
\end{ex}

\begin{rmk}
	Sia $\Cc$ un ideale di Serre. Se $X$ è uno spazio topologico con
	\begin{equation*}
		H_{n}(X; \Z) = \cat{0} = H_{n-1}(X;\Z) \quad \mathrm{mod}\Cc\,,
	\end{equation*}
	allora per ogni $M \in \Ab$ si ha $H_{n}(X;M) = \cat{0} \mathrm{mod} \Cc$.
	Se $\Cc$ è solamente un anello, questa conclusione segue
	se $M \in \Cc$. 
\end{rmk}

\begin{prop}
	Sia $\pi : E \to B$ una fibrazione di Serre, con $B,F$ connessi per archi
	e $\pi_{1}(B, b_{0})$ agisce banalmente su $H_{*}(F)$.
	Sia $\Cc$ un ideale di Serre.
	Se per ogni $t > 0$ si ha $H_{t}(F) \in \Cc$, allora
	\begin{equation*}
		\pi_{*}:H_{*}(F) \longrightarrow H_{*}(B)
	\end{equation*}
	è un isomorfismo $\mathrm{mod}\Cc$.
	\begin{proof}
		Per il \textbf{Teorema dei Coefficienti Universali} si ha
		\begin{equation*}
			E^{r}_{s,t} = H_{s}(B;H_{t}(F)) \in \Cc\,, \quad \text{per } t>0\,.
		\end{equation*}
		Allora anche $E^{r}_{s,t} \in \Cc$ per $t >0$ e anche il
		limite $E^{\infty}_{s,t} \in \Cc$. Questo implica che $\pi_{*}$
		è un isomorfismo $\mathrm{mod}\Cc$.
	\end{proof}
\end{prop}

\begin{prop}\label{serre-mod-c}
	Sia $\pi : E \to B$ una fibrazione di Serre,
	con $B,F$ connessi per archi e base semplicemente connessa $H_{1}(B)=\cat{0}$.
	Se $\Cc$ è una classe di Serre tale che $H_{s}(B) \in \Cc$ per ogni $0 < s < n$
	e $H_{t}(F) \in \Cc$ per $ 0 < t < n-1$. Allora per ogni $i \le n$,
	la proiezione
	\begin{equation*}
			\pi_{*}:H_{i}(E,F) \longrightarrow 
			H_{i}(B,b_{0}) \in \Cc\,, 
		\end{equation*}
		è un isomorfismo $\mathrm{mod}\Cc$.
		\begin{proof}
			Usiamo la \textbf{versione relativa}  della successione di Serre.\todo{copia dim.}
		\end{proof}
\end{prop}

%Come conseguenza di questo fatto,
%otteniamo una nuova versione del Teorema di Hurewicz.
%
%\begin{cor}
%	Sia $\Cc$ un anello di Serre aciclico e $X$ uno spazio semplicemente connesso.
%	Per ogni $n \ge 2$, si ha
%	\begin{equation*}
%		\pi_{q}(X, \ast) \in \Cc \iff H_{q}(X) \in \Cc
%	\end{equation*}
%	per ogni $q < n$. In tal caso, la mappa di Hurewicz
%	\begin{equation*}
%		h_{n} : \pi_{n}(X) \longrightarrow H_{n}(X)
%	\end{equation*}
%	è un isomorfismo $\mathrm{mod} \Cc$.
%\end{cor}