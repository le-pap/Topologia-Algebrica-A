 %%% Lezione 12


\lecture[Join di spazi. Costruzione di Milnor del fibrato universale per un gruppo topologico. Insiemi simpliciali: realizzazione geometrica. Costruzione dello spazio simpliciale associato a un gruppo topologico. Bar complex.]{2023-04-04}


\section{Costruzione di Milnor}


Presentiamo una costruzione di $BG$ molto \emph{categoriale}.
Definiamo la \textbf{categoria simplesso} $\Delta$ come
la categoria con oggetti gli insiemi
\begin{equation*}
	[n] := \Set{0, 1, \dots, n}\,,
	\quad n \ge 0\,,
\end{equation*}
e come morfismi le funzioni \emph{non decrescenti} $f:[n] \to [m]$.

\begin{ex}
	$i$-codeg e cofacce
\end{ex}

\begin{lemma}
	Per ogni $i \le $, le cofacce e le mappe codegeneri soddisfano le seguenti relazioni:
	\begin{equation*}
		d^{i}d^{j} = d^{i}...
	\end{equation*}
\end{lemma}

\begin{fact}
	I morfismi $d^{i}$ e $s^{i}$ generano tutti i morfismi nella categoria $\Delta$.
\end{fact}

La categoria opposta $\Delta^{op}$ ha gli stessi oggetti di $\Delta$,
e morfismi
\begin{equation*}
	\Hom_{\Delta^{op}}([n],[m]) = \Hom_{\Delta}([m],[n])\,,
\end{equation*}
quindi vediamo che la cofaccia $i$-esima $d^{i}:[n-1] \to [n]$
corrisponde a un morfismo
\begin{equation*}
	d_{i}:[n] \longrightarrow [n-1]\,,
\end{equation*}
e...

Ricordiamo che un complesso simpliciale $X$ è un'unione di simplessi.
Se $X_{n}$ è un insieme di indici che parametrizza gli $n$-simplessi,
allora
\begin{equation*}
	X = \left( \bigcup_{n \ge 0} \Delta^{n} \times X_{n} \right) / \sim\,,
\end{equation*}
dove $\Delta^{n}$ è l'$n$-simplesso standard in $\R^{n+1}$,
e $\sim$ è la relazione d'equivalenza che identifica le coppie
allora ci descrive come non si capisce un cazzo.

Vogliamo generalizzare questa idea topologica a livello categoriale.
\begin{df}
	Un \textbf{simpliciale} $X$ è una collezione di insiemi $X_{n}$, con $n \ge 0$,
	e applicazioni $d_{i}:X_{n} \to X_{n-1}, s_{j}:X_{n} \to X_{n-1}$ che soddisfano
	le relazioni sopra. 
\end{df}

In modo più formale, si può dire che un insieme simpliciale è un funtore
$F : \Delta^{op} \to \CSet$ tale che
\begin{equation*}
	\begin{tikzcd}
		{[n]} \ar[rr, "F"] & & X_{n} \ar[d, "F(d_{i})"] \\
		{[n-1]} \ar[rr, "F"] \ar[u, bend right, "s_{j}"] & & X_{n-1} \ar[u, bend right, "F(s_{j})"]\,.
	\end{tikzcd}
\end{equation*}
Nel caso più generale, al posto di $\CSet$ si può considerare una categoria $\Cc$ qualsiasi
e definire un insieme simpliciale come un funtore $F:\Delta^{op} \to \Cc$.


Finora abbiamo parlato di spazi $BG$, ma non abbiamo mai assicurato la loro esistenza.
Descriviamo una costruzione funtoriale per associare
a un gruppo $G$ uno spazio topologico,
che in effetti sarà lo spazio classificante $BG$.

Il \textbf{join} di due spazi $X$ e $Y$
è l'unione di tutti i segmenti da un punto di $X$ e un punto di $Y$:
più precisamente definiamo
\begin{equation*}
	X \star Y = X \times I \times Y/ \sim
\end{equation*}
dove la relazione $\sim$ identifica gli estremi dei segmenti che arrivano
nello stesso punto, cioè
\begin{equation*}
	(x,0,y) \sim (x,0,y')\,, \quad (x,1,y) \sim (x',1,y)\,.
\end{equation*}
Indichiamo un punto del join come una combinazione convessa
$t_{1}x + t_{2}y := (x,t,y)$,.

\begin{ex}
	\begin{rmnumerate}
		\item Se $Y = \{y\}$, allora il join è il cono poichè
		$X \star Y \simeq X \times I/ X \times \{1\} \simeq CX$,
		quindi è contraibile.
		
		\item Se $Y = S^{0}$, allora il join è la sospensione:
		\begin{equation*}
			X \star Y = \Sigma X\,.
		\end{equation*}
		In particolare notiamo che $S^{n} \star S^{0} = S^{n+1}$
		e in maniera simile vale che $S^{n} \star S^{m} = S^{n+m+1}$.
		
		\item Se consideriamo $x_{0}, x_{1}, \dots, x_{n}$ punti distinti,
		allora il loro join è l'inviluppo convesso con vertici questi $n+1$ punti,
		quindi $x_{0} \star x_{1} \star \dots \star x_{n} \simeq \Delta^{n}$.
	\end{rmnumerate}
\end{ex}

\begin{exercise}
	La costruzione $X \star Y$ aumenta la connessione dello spazio:
	se $X$ è $(m-1)$-connesso e $Y$ è $(n-1)$-connesso,
	allora $X \star Y$ è $(m+n)$-connesso.
\end{exercise}


Abbiamo delle inclusioni ovvie $i:X \subset X \star Y$ e $j:Y \subset X \star Y$.

\begin{lemma}
	Le inclusioni $i$ e $j$ sono omotopicamente banali.
	\begin{proof}
		Fissato $y_{0} \in Y$, si ha
		\begin{equation*}
			X \hookrightarrow X \star y_{0} \subset X \star Y\,.
		\end{equation*}
		Siccome $X \star y_{0}$ è contraibile, allora deduciamo che
		$i \sim y_{0}$. Lo stesso ragionamento vale per $j$.
	\end{proof}
\end{lemma}


Sia $G$ un gruppo topologico e consideriamo il join iterato
\begin{equation*}
	G^{\star(k+1)} = \underbrace{G \star G \star \dots \star G}_{k+1 \text{ fattori}}\,,
\end{equation*}
i cui elementi verranno indicati con combinazioni convesse, cioè 
$\sum_{i=0}^{k}t'_{i}g_{i}$ con $\sum_{i=0}^{k}t'_{i}=1$ e tutti $t'_{i} \ge 0$.

Notiamo che c'è un'azione libera di $G$ su $G^{\star(k+1)}$,
data dall'azione diagonale
\begin{equation*}
	(g_{0}, t_{1}g_{1}, t_{2}g_{2}, \dots, t_{k}g_{k}) \cdot g
	= (g_{0}\cdot g, t_{1}g_{1}\cdot g, t_{2}g_{2}\cdot g, \dots, t_{k}g_{k}\cdot g)\,,
\end{equation*}
che con la notazione delle combinazioni convesse si scrive semplicemente 
$\left( \sum_{i=0}^{k}t'_{i}g_{i} \right) \cdot g = \sum_{i=0}^{k}t'_{i}(g_{i} \cdot g)$. 


\begin{exercise}
	C'è una mappa $G$-equivariante 
	\begin{equation*}
		\phi: \Delta^{k} \times G^{\star(k+1)} \longrightarrow G^{\star(k+1)}\,,
		\quad \Big((t_{0}, t_{1}, \dots, t_{k}), (g_{0}, g_{1}, \dots, g_{k}) \Big)
		\longmapsto \sum_{i=0}^{k}t_{i}g_{i}\,,
	\end{equation*}
	che è un omeomorfismo su  $(\Delta^{k})^{o} \times G^{\star(k+1)}$
	e $G$ agisce banalmente sul simplesso, mentre ha l'azione diagonale su $G^{\star(k+1)}$.
	In questo modo si ottiene un $G$-CW-complesso
	\begin{equation*}
		G^{\star(k+1)} = \left. \coprod_{i} \Delta^{k}_{i} \times G \middle/ \sim \right. \,.
	\end{equation*}
\end{exercise}

\begin{df}
	Dato un gruppo topologico $G$, definiamo lo spazio
	\begin{equation*}
		J(G) := \lim_{k \to \infty} G^{\star(k+1)}\,,
	\end{equation*}
	dove il limite è preso rispetto alle inclusioni 
	$i:G^{\star(k+1)} \subset G^{\star(k+2)}$.
	Siccome tutte le inclusioni sono $G$-equivarianti,
	notiamo che $G$ agisce su $J(G)$.
\end{df}

Siccome fare il join aumenta la connessione dello spazio,
consideriamo questa costruzione limite per ottenere
uno spazio contraibile.


\begin{thm}
	Il quoziente $p:J(G) \to J(G)/G$ è un principale universale per $G$.
	\begin{proof}
		Per l'osservazione precedente,
		$J(G)$ è uno spazio (debolmente) contraibile...
	\end{proof}
\end{thm}

\begin{df}
	Un \textbf{insieme simpliciale} $X_{*}$ in una categoria $\Cc$ 
	è una famiglia di oggetti $X_{n}$, con $n \in \N$, e applicazioni
	\begin{equation*}
		d_{i}: X_{n} \longrightarrow X_{n-1}\,, \quad s_{j}:X_{n} \longrightarrow X_{n-1}\,,
	\end{equation*}
	tali che soddisfino le seguenti relazioni:
	\begin{align*}
		d_{i}d_{j} &= d_{j-1} d_{i}\,, \quad \text{se } i<j\,,\\
		s_{i}s_{j} &= s_{j-1} s_{i}\,, \quad \text{se } i<j\,,\\
		d_{i}s_{j} &=
		\begin{cases}
			s_{j-1}d_{i}\,, \quad &\text{se } i < j\; \\
			1\,, \quad &\text{se }\,;\\
			s_{j}d_{j-1}\,, \quad &\text{se } i > j+1\,.
		\end{cases}
	\end{align*}
\end{df}

Sono le relazioni che si ottengono per i simplessi standard $\Delta^{n}$
---

Un insieme simpliciale si può vedere come un funtore
$F: \Delta^{op} \to \CSet$, dove per ogni $n \ge 0$ si ha un quadrato commutativo
\begin{equation*}
	\begin{tikzcd}
		{[n]} \ar[rr, "F"] & & X_{n} \ar[d, "F(d_{i})"] \\
		{[n-1]} \ar[rr, "F"] \ar[u, bend right, "s_{j}"] & & X_{n-1} \ar[u, bend right, "F(s_{j})"]\,.
	\end{tikzcd}
\end{equation*}
Per capire come effettivamente si attaccano i complessi,
realizzo le cofacce e le cogenerazioni in $\Delta$ 
paragonandolo al caso dei simplessi standard $\Delta^{n} = [e_{0}, e_{1}, \dots, e_{n}]$,
con $\{e_{i}\}$ base canonica di $\R^{n+1}$. Allora
\begin{equation*}
	d^{i}: \Delta^{n-1} \longrightarrow \Delta^{n}\,,
	\quad d^{i}(t_{1}, \dots, t_{n}) =
	\begin{cases}
		(t_{1}, \dots, t_{i-1}, 0, t_{i+1}, \dots, t_{n})\,, \quad &\text{se } i \ge 0\,; \\
		\left(1-\sum_{i=1}^{n} t_{i}, t_{1},  \dots , t_{n} \right)\,, \quad &\text{se } i = 0\,.
	\end{cases}
\end{equation*}
mentre le codegenerazioni sono date da
\begin{equation*}
	s^{i}: \Delta^{n-1} \longrightarrow \Delta^{n}\,,
	\quad s^{i}(t_{1}, \dots, t_{n}) =
	\begin{cases}
		(t_{1}, \dots, t_{i-1}, t_{i}+t_{i+1}, t_{i+2}, \dots, t_{n})\,, \quad &\text{se } i \ge 0\,; \\
		\left(t_{2}, t_{3},  \dots , t_{n} \right)\,, \quad &\text{se } i = 0\,.
	\end{cases}
\end{equation*}

\begin{df}
	La \textbf{realizzazione geometrica} di un insieme simpliciale $X_{*}$
	è lo spazio
	\begin{equation*}
		|X_{*}| = \left. \coprod_{n \ge 0} X_{n} \times \Delta^{n} \middle/ \sim \right. \,,
	\end{equation*}
	dove la relazione $\sim$ attacca i punti di un $(n-1)$-simplesso con
    quelli di un $n$-simplesso tramite la cofaccia, cioè
    \begin{equation*}
    	(t, d_{i}(x)) \sim (d^{i}(t),x)\,, \quad \text{con } (t,x) \in \Delta^{n-1} \times X_{n}\,,
    \end{equation*}
    mentre i complessi ``superflui'' vengono eliminati dalle codegenerazioni
    \begin{equation*}
    	(t, s_{j}(x)) \sim (s^{j}(t),x)\,, \quad \text{ce } (t,x) \in \Delta^{n+1} \times X_{n}\,,
    \end{equation*}
    che incollano gli $(n+1)$-simplessi degeneri come facce di $n$-simplessi.
    La topologia di $X_{n}$ si assume discreta,
    quindi $\Delta_{n} \times X_{n}$ è un insieme di $n$-simplessi discreto.
    Ne segue che la realizzazione $|X_{*}|$ è un CW complesso con $n$-scheletro
    \begin{equation*}
    	|X_{*}|_{n} = \coprod_{k \ge n} (\Delta^{n} \times X_{k})/\sim\,.
    \end{equation*}
\end{df}

Vedi Peter May ``Simplicial object'' per una cosa dettagliata,
oppure il paper carino ``An elementary illustrated introduction to simplicial sets'' di Greg Friedman.

\begin{prop}
	Se $X_{*}$ è un insieme simpliciale,
	possiamo associargli il complesso di catene
	$C_{\bullet}(X_{*})$, dove
	\begin{equation*}
		C_{n}(X_{*}) = \bigoplus_{x \in X_{n}} \Z x
	\end{equation*}
	è il gruppo abeliano libero generato da $X_{n}$,
	le cui mappe di bordo sono
	\begin{equation*}
		\de_{i}: C_{n}(X_{*}) \longrightarrow C_{n-1}(X_{*})\,,
		\quad \de_{i}(x) = \sum_{i=0}^{n}(-1)^{i}d_{i}(x)\,.
	\end{equation*}
	Allora $H_{*}(X_{*};\Z)$ è l'omologia di $C_{\bullet}(X_{*})$.
\end{prop}

Sia $G$ un gruppo topologico, e definiamo $\epsilon G_{*}$
l'insieme simpliciale con
\begin{equation*}
	\epsilon G_{n} = G^{\star(n+1)}\,,
\end{equation*}
in cui le facce sono
\begin{equation*}
	d_{i}: G^{\star(n+1)} \longrightarrow G^{\star n)}\,,
	\quad d_{i}(g_{0}, \dots, g_{n}) = (g_{0}, \dots, \widehat{g_{i}}, \dots, g_{n})\,,
\end{equation*}
mentre le degenrazioni sono date da
\begin{equation*}
	s_{j}: G^{\star(n+1)} \longrightarrow G^{\star n)}\,,
	\quad s_{j}(g_{0}, \dots, g_{n}) = (g_{0}, \dots, \widehat{g_{i}}, \dots, g_{n})\,,
\end{equation*}

...

Notiamo che $G$ agisce (a destra) su $|\epsilon G_{*}|$ tramite l'azione diagonale
\begin{equation*}
	\Big( t, (g_{0}, \dots, g_{n})\Big) \cdot g = \Big( t, (g_{0}\cdot g, \dots, g_{n}\cdot g)\Big)\,.
\end{equation*}
Ponendo $E := |\epsilon G_{*}|$, siccome è uno spazio asferico, allora
\begin{equation*}
	p : EG \longrightarrow EG/G = BG
\end{equation*}
è un $G$-fibrato universale. 
La base $BG$ ha una struttura di insieme simpliciale indotta da $\epsilon G_{*}$, 
più precisamente si ha $BG_{n} = G^{n}$,
con facce date da
\begin{equation*}
	d_{i}(g_{0}, \dots, g_{n}) =
\end{equation*}
e degenrazioni 
\begin{equation*}
	s_{j}(g_{0}, \dots, g_{n}) =
	\begin{cases}
		(1,g_{0}, \dots, g_{n})\,, \quad &\text{se } j=0\,;\\
		(g_{0}, \dots, g_{j}, 1, g_{j+1}, \dots, g_{n})\,, \quad &\text{se } j>0\,.
	\end{cases}
\end{equation*}
In questo modo, possiamo realizzare il fibrato $p_{*}: \epsilon G_{*} \to BG_{*}$
come
\begin{equation*}
	p(g_{0}, g_{1},\dots, g_{n}) = (g_{0} g_{1}^{-1}, g_{1}g_{2}^{-1},\dots, g_{n-1}g_{n}^{-1})\,,
\end{equation*}
che induce $p:| \epsilon G_{*} | \to |BG_{*}|\simeq BG$ tra le realizzazioni geometriche.

La costruzione appena descritta genera il \textbf{bar complex}.

\begin{oss}
Siccome la costruzione di $BG_{*}$ utilizza solamente la moltiplicazione del gruppo $G$
e il suo elemento neutro, ma \emph{non} elementi inversi, allora
lo stesso ragionamento può essere generalizzato per realizzare lo
\textbf{spazio classificante di un monoide}, e ancora più in generale di una \textbf{categoria} $\Cc$.
\end{oss}