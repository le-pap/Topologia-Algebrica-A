 %%% Lecture 7




\begin{thm}
	Un complesso di catene graduato e filtrato $(A,d,F)$ determina una successione spettrale
	$\Set{(E_{\bullet,\bullet}^{r},d^{r})}$ tale che
	$E_{s,t}^{1} = H_{s+t}(F_{s}A/F_{s-1}A)$ e il differenziale $d^{1}$ è
	l'operatore di bordo della tripla $(F_{s}A,F_{s-1}A,F_{s-2})$.
	Se la filtrazione $F_{\bullet}$ è convergente e limitata sia dal basso, sia dall'alto,
	allora la successione spettrale stabilizza e converge al limite $E^{\infty}_{p,q}$
	che è isomorfo al quoziente $F_{p}H_{p+q}(A)/F_{p-1}H_{p+q}(A)$.
\end{thm}


\lecture[Lezione.]{2023-03-20}






