 %%% Lecture 7

\chapter{Fibrati vettoriali}


\begin{thm}
	Un complesso di catene graduato e filtrato $(A,d,F)$ determina una successione spettrale
	$\Set{(E_{\bullet,\bullet}^{r},d^{r})}$ tale che
	$E_{s,t}^{1} = H_{s+t}(F_{s}A/F_{s-1}A)$ e il differenziale $d^{1}$ è
	l'operatore di bordo della tripla $(F_{s}A,F_{s-1}A,F_{s-2})$.
	Se la filtrazione $F_{\bullet}$ è convergente e limitata sia dal basso, sia dall'alto,
	allora la successione spettrale stabilizza e converge al limite $E^{\infty}_{p,q}$
	che è isomorfo al quoziente $F_{p}H_{p+q}(A)/F_{p-1}H_{p+q}(A)$.
\end{thm}


\lecture[Lezione.]{2023-03-20}

Data una varietà differenziabile $M$, lo spazio tangente al punto $p \in M$
è lo spazio vettoriale $T_{p}M$ delle derivazioni in $p$ e ``varia con continuità''
al variare del punto. Si può descrivere intrinsecamente
come l'insieme delle classi d'equivalenza di curve passanti in $p$.
Tutti gli spazi tangenti sono incollati all'interno
del \emph{fibrato tangente} $TM$, che può essere pensato come
una famiglia di spazi vettoriali 
parametrizzati con continuità dalla varietà $M$.

La generalizzazione di questo concetto è espresso dalla nozione
di \textbf{fibrato vettoriale}.

\begin{df}
	Dato uno spazio topologico $B$, definiamo la categoria $\cat{Top}/B$
	come la categoria avente come oggetti delle mappe continue $E \to B$,
	e come morfismi dei triangoli commutativi di mappe continue
	\begin{equation*}
		\begin{tikzcd}
			E \ar[rr, "\mu"] \ar[dr] & & E' \ar[dl] \\
			& B & \,.
		\end{tikzcd}
	\end{equation*}
\end{df}

\begin{ex}
	Dato uno spazio topologico $F$, il prodotto cartesiano $E = F \times B$
	è un fibrato topologico detto \textbf{fibrato banale}.
\end{ex}

La categoria $\cat{Top}/B$ ha prodotti finiti: dati due oggetti $(E,p)$
e $(E',p')$, il loro prodotto in $\cat{Top}/B$ è dato dal prodotto fibrato
$E \times_{B} E'$ in $\cat{Top}$.

\begin{df}
	Detta $ u = (B,\cat{1}_{B})$ l'identità della base,
	una \textbf{sezione} è un morfismo $s:u \to p$,
	cioè una mappa $s:B \to E$ tale che $p \circ s = \cat{1}_{B}$.
\end{df}

\begin{df}
	Definiamo un \textbf{gruppo abeliano} su $B$ un oggetto $(E,p)$
	dotato di una mappa
	\begin{equation*}
		E \times_{B} E \longrightarrow E\,, \quad (e,e') \longmapsto e + e'\,,
	\end{equation*}
	dotato di un \textbf{elemento neutro} $v \in E$ tale che ...
\end{df}

Siamo interessati al caso in cui le fibre su $B$ sono \emph{spazi vettoriali},
in cui non ci siano ``salti di dimensione'' negli spazi parametrizzati
(e.g. $xy=0$ in $\R^{2}$ non ci piace).

\begin{df}
	Un \textbf{fibrato vettoriale} su $B$ è un oggetto $(E,p)$ \emph{localemente banale},
	cioè per ogni $b \in B$, esiste un intorno aperto $U \ni p$ tale che
	\begin{equation*}
		p : p^{-1}(U) \longrightarrow U \times \R^{n}
	\end{equation*}
	è un omeomorfismo.\todo{aggiusta}
	Lo spazio $E$ viene detto \textbf{spazio totale}, $B$ è detto \textbf{spazio base}
	e un aperto $U$ che soddisfa la condizione sopra è detto
	\textbf{aperto banalizzante}.
\end{df}

Gli isomorfismi banalizzanti
\begin{equation*}
	\begin{tikzcd}
		p^{-1}(U_{i}) \ar[rr, "\Phi_{i}"] \ar[dr] 
		& & U_{i} \times \R^{n} \ar[dl] \ar[ll, bend right, "\Psi_{i}"'] \\
		& U_{i} &
	\end{tikzcd}
\end{equation*}
inducono isomorfismi \emph{di spazi vettoriali} sulle fibre $p^{-1}(b)$, per tutti i $b \in B$.
Indichiamo con
\begin{equation*}
	\Phi_{i,b} := \Phi_{i}\vert_{p^{-1}(b)} : p^{-1}(b) \longrightarrow \{b\} \times \R^{n}\,,
	\quad \Phi_{i}(e) = \big( p(e), \phi_{i}(e) \big)\,.
\end{equation*}

Nell'intersezione di due aperti banalizzanti $U_{i} \cap U_{j}$ possiamo
\emph{cambiare banalizzazione}
\missingfigure{copiare le notazioni.}
Otteniamo così delle funzioni
\begin{equation*}
	g_{ij} : U_{i} \cap U_{j} \longrightarrow \mathrm{GL}(n,\R)
\end{equation*}
dette \textbf{funzioni di transizione}.

\begin{oss}
	Si può verificare che valgono le seguenti \textbf{condizioni di cociclo}:
	\begin{itemize}
		\item $g_{ji}(x) = \big( g_{ij}(x) \big)^{-1}$;
		\item per ogni tripla di indici $i,j,h$ si ha $g_{ij} g_{jh} = g_{ih}$. 
	\end{itemize}
	Tutte le informazioni di un fibrato vettoriale possono essere 
	lette in termine delle funzioni di transizione $\{g_{ij}\}$ su
	un ricoprimento banalizzante fissato. Ad esempio,
	dato un morfismo di fibrato vettoriali $f: \xi \to \xi'$ su una stessa base $B$,
	possiamo considerare un ricoprimento banalizzante $\{U_{i}\}$ comune
	per entrambi i fibrati e ``leggere'' $f$ tramite le coordinate
	date dalle banalizzazioni, cioè
%	\begin{equation*}
%		\begin{tikzcd}
%			p^{-1}(U_{i}) \ar[]
%		\end{tikzcd}
%	\end{equation*}
	e $f$ induce così una famiglia $f_{ij}:U_{i} \cap U_{j} \to \mathrm{GL}(n,\R)$
	di funzioni che soddisfano le condizioni\todo{Copia condizioni} 
\end{oss}

\begin{thm}
	Dati due fibrati $\xi, \xi'$ su $B$ e un ricoprimento $\{U_{i}\}$ che banalizza entrambi,
	allora $\xi \simeq \xi'$ se e solo se esistono $\lambda_{i}:U_{i} \to \mathrm{GL}(n,\R)$
	che ``collegano'' le funzioni di transizione nel seguente modo:
	\begin{equation*}
		g_{ij}'(x) = \big( \lambda_{i}(x) \big)_{-1} g_{ij}(x) \lambda_{j}(x)\,.
	\end{equation*}
\end{thm}

\begin{thm}[\textbf{di esistenza}]
	Dato un ricoprimento $\{U_{i}\}$ di $B$ e delle funzioni 
	$g_{ij}:U_{i} \cap U_{j} \to \mathrm{GL}(n,\R)$ che soddisfano le \textbf{condizioni di cociclo},
	allora esiste un fibrato vettoriali di rango $n$ su $B$ che ha le $g_{ij}$ come
	funzioni di transizione.
	\begin{proof}[Idea della dimostrazione]
		Partendo da $\coprod_{i} U_{i} \times \R^{n}$,
		si identificano le coppie tramite la relazione d'equivalenza
		$(x,y)_{i} \sim (x,g_{ij}(x)y)_{j}$.
	\end{proof}
\end{thm}

\begin{ex}
	\begin{rmnumerate}
		\item Il \textbf{fibrato banale} su $B$ è $\mathrm{pr}_{1}:B \times \R^{n} \to \R^{n}$.
		
		\item Sia $\KK = \R$ oppure $\KK = \C$.
		Posto $V := \KK^{n}$, la \textbf{Grassmaniana} $Gr_{k}(V)$ si può topologizzare
		\todo{Non si legge niente} 
		
		
		La grassmaniana $Gr_{k}(V)$ si può ricoprire con $\binom{n}{k}$ aperti
		che corrispondono alle matrici $M$ in cui $k$ colonne fissate sono indipendenti.
		I parametri che \todo{aiuto non si legge proprio}
		
		\textbf{Fibrato $\xi_{n,k}$}: costruiamo un fibrato su $\mathrm{GL}(n,\R)$
		con spazio totale
		\begin{equation*}
			E(\xi_{n,k}) = \Set{(W,x) \in Gr_{n}(V) \times V | x \in W}
		\end{equation*}
		e proiezione $p(W,x)=W$. In questo modo,
		la fibra su $W$ si identifica con $W$.
		Se $k=1$, allora $Gr_{1}(V) = \PP(V) = \PP^{n-1}(\KK)$ e chiamiamo
		$\xi_{n,1}$ un \textbf{fibrato in rette}.
		
		\item Ponendo $n=2$ e $\KK = \R$ nell'esempio precedente,
		allora $Gr_{1}(\R^{2}) = \PP^{1}(\R) \simeq S^{1}$.
		Conosciamo due fibrati in rette sulla circonferenza:
		il cilindro $S^{1} \times \R$ e il nastro di Möebius.
		Vedremo che questi sono gli \emph{unici} fibrati in rette su $S^{1}$
		(a meno di isomorfismo). Chi è $\xi_{2,1}$?
		Consideriamo su $\PP^{1}(\R)$ i due aperti seguenti:
		\begin{equation*}
			U_{1} := \Set{(x,1) | x \in \R}\,, \quad
			U_{2} := \Set{(1,y) | y \in \R}\,.
		\end{equation*}
		Su questi aperti abbiamo le banalizzazioni
		\begin{equation*}
			\Psi_{1} : (p_{x}, \lambda p_{x} ) \longmapsto (p_{x}, \lambda) \in U_{1} \times \R\,,
			\quad \Psi_{2} : (q_{y}, \mu q_{y}) \longmapsto (q_{y},\mu) \in U_{2} \times \R\,.
		\end{equation*}
		Notiamo che l'intersezione $U_{1} \cap U_{2}$ è sconnessa,
		e su questo aperto la funzione di transizione è data da
		\begin{equation*}
			\begin{tikzcd}
				(q_{y},\mu) \ar[r, "\Psi_{2}"]
				& {\left([1:y], \mu [1:y] \right)} \ar[r,equals]
				& f
			\end{tikzcd}
		\end{equation*}
		e dato che il determinante cambia segno nelle due componenti
		connesse, deduciamo che $\xi_{2,1}$ \textbf{non} è banale.
	\end{rmnumerate}
\end{ex}

\begin{exercise}
	Dimostrare che per ogni $n \ge 2$ il fibrato $\xi_{n,1}$ è non banale.
\end{exercise}

\begin{ex}
	Un fibrato in rette banale ha una sezione mai nulla
	\begin{equation*}
		B \longrightarrow B \times \R\,,
		\quad b \longmapsto (b,1)\,.
	\end{equation*}
\end{ex}

\begin{oss}
	Un fibrato di rango $n$ è banale se e solo se ammette
	$n$ sezioni linearmente indipendenti.
	Infatti, dato $\xi:B \times \R^{n} \to B$ possiamo definire
	se sezioni $s_{i}(b) := (b, e_{i})$. Viceversa,
	se abbiamo $s_{1}(b), \dots, s_{n}(b)$ linearmente indipendenti per ogni $b \in B$,
	allora possiamo banalizzare il fibrato tramite l'isomorfismo
	\begin{equation*}
		B \times \R^{n} \longrightarrow E\,,
		\quad (b, \lambda_{1}, \dots, \lambda_{n}) \longmapsto \sum_{i=1}^{n}\lambda_{i}s_{i}(b)\,.
	\end{equation*}
\end{oss}

\begin{exercise}
	L'oggetto $\xi_{n,k}$ definito sopra è un \textbf{fibrato vettoriale}.
	\begin{proof}[Una soluzione]
		Si verifica che una banalizzazione $p^{-1}(U) \to U \times \R^{k}$ è data da
		\begin{equation*}
			\left( 
				\begin{pmatrix}
					1 & & 0 & a_{1,k+1} & & a_{1,n} \\
					& & & & & \\
					0 & & 1 & a_{k,k+1} & & a_{k,n}
				\end{pmatrix},
				\lambda_{1}A_{1} + \dots + \lambda_{k}A_{k}
			\right)
		\end{equation*}
	\end{proof}
\end{exercise}

\begin{exercise}
	Determinare un ricoprimento banalizzante per $\tau_{n}$, 
	il fibrato tangente di $S^{n-1}$.
\end{exercise}

A partire da fibrati vettoriali noti,
possiamo ottenere nuovi fibrati vettoriali con costruzioni
simili a operazioni che conosciamo sugli spazi vettoriali:
\begin{itemize}
	\item \textbf{prodotto}: dati
	$\begin{tikzcd}
		\xi : E \ar[r, "p"] & B
	\end{tikzcd}$ e 
	$\begin{tikzcd}
		\xi' : E' \ar[r, "p'"] & B
	\end{tikzcd}$,
	definiamo $\xi \times \xi'$ come il fibrato $p \times p' : E \times E' \to B$;
	
	\item \textbf{somma di Whitney}: dati $\xi,\xi'$ sulla stessa base $B$,
	il fibrato $\xi \oplus \xi'$
	è dato dal pullback
	\begin{equation*}
		\begin{tikzcd}
			\Delta^{*}(\xi \times \xi') \ar[r] \ar[d]
			& E \times E' \ar[d, "p \times p'"] \\
			B \ar[r, "\Delta"] & B \times B\,,
		\end{tikzcd}
	\end{equation*}
	e si verifica che le fibre sono date da $\left( \xi \oplus \xi' \right)_{b} = E_{b} \oplus E_{b}'$.
\end{itemize}

\begin{exercise}
	Verificare che $T\,Gr_{k}(\R^{n}) \simeq \Hom_{\xi}(\xi_{n,k},\xi_{n,k}^{\perp})$.
\end{exercise}



