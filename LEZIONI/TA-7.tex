 %%% Lecture 7

\chapter{Fibrati vettoriali}


\lecture[Lezione.]{2023-03-20}

Data una varietà differenziabile $M$, lo \emph{spazio tangente} al punto $p \in M$
è lo spazio vettoriale $T_{p}M$ delle derivazioni in $p$ e ``varia con continuità''
al variare del punto. Si può descrivere intrinsecamente
come l'insieme delle classi d'equivalenza di curve di curve lisce
\begin{equation*}
	\gamma : (\R,0) \longrightarrow (M,p)\,, \quad \gamma(0) = p\,,
\end{equation*}
dove $\gamma_{1} \sim \gamma_{2}$ se e solo se $\gamma_{1}'(0) = \gamma_{2}'(0)$.
La dimensione di ogni spazio vettoriale $T_{p}M$ è 
la stessa della dimensione (come varietà) di $M$.
Tutti gli spazi tangenti sono incollati all'interno
del \emph{fibrato tangente} $TM$, che può essere pensato come
una famiglia di spazi vettoriali 
parametrizzati \emph{con continuità} dalla varietà $M$.

La generalizzazione di questo concetto è espresso dalla nozione
di \textbf{fibrato vettoriale}.

\begin{df}
	Dato uno spazio topologico $B$, definiamo la categoria $\cat{Top}/B$
	come la categoria i cui oggetti $\xi = (E,p)$ sono mappe continue $E \to B$
	e i morfismi $\mu : \xi \to \xi'$ sono
	funzioni continue $\mu :E \to E'$ che rendono commutativi i triangoli
	\begin{equation*}
		\begin{tikzcd}
			E \ar[rr, "\mu"] \ar[dr, "p"'] & & E' \ar[dl,, "p'"] \\
			& B & \,.
		\end{tikzcd}
	\end{equation*}
\end{df}

\begin{ex}
	Dato uno spazio topologico $F$, il prodotto cartesiano $E = F \times B$
	dotato della proiezione $\mathrm{pr}_{1}:B \times F \to B$ è un
	%fibrato topologico 
	oggetto su $B$ detto \textbf{fibrato banale}.
\end{ex}

\begin{thm}
	Un fibrato vettoriale è una \textbf{fibrazione}.
	\begin{proof}
		Vedi \parencite[Proposition~4.48]{hatcher}.
	\end{proof}
\end{thm}

La categoria $\cat{Top}/B$ ha prodotti finiti: dati due oggetti $\xi = (E,p)$
e $\xi' = (E',p')$, il loro prodotto $\xi \times \xi'$ in $\cat{Top}/B$ 
è dato dal prodotto fibrato in $\cat{Top}$
\begin{equation*}
	E \times_{B} E' = \Set{(e,e') \in E \times E' | p(e) = p'(e')}\,.
\end{equation*}
Nella categoria $\cat{Top}/B$ troviamo anche l'oggetto finale $u = (B,\cat{1}_{B})$,
infatti ogni $\xi = (E,p)$ ha un'unico morfismo $\xi \to u$
dato da $p:E \to B$.

\begin{df}
	Detta $ u = (B,\cat{1}_{B})$ l'identità della base,
	una \textbf{sezione} è un morfismo $s:u \to \xi$,
	cioè una mappa $s:B \to E$ tale che $p \circ s = \cat{1}_{B}$.
	\begin{equation*}
		\begin{tikzcd}
			E \ar[rr,"p"'] \ar[dr,"p"'] & & B \ar[ll, "s"', bend right, dashed] \ar[dl, equals] \\
			& B &
		\end{tikzcd}
	\end{equation*}
\end{df}

\begin{ex}
	Dato un fibrato banale su $B$, è facile costruire delle sezioni:
	fissato un punto $x_{0} \in F$, la mappa 
	\begin{equation*}
		\iota_{x_{o}} : B \longrightarrow B \times F\,,
		\quad (b,x_{0})
	\end{equation*}
	definisce una sezione.
\end{ex}

\begin{df}
	Definiamo un \textbf{gruppo abeliano} su $B$ come 
	un oggetto gruppo nella categoria $\cat{Top}/B$,
	cioè $\xi = (E,p)$ dotato di una mappa
	\begin{equation*}
		+ : E \times_{B} E \longrightarrow E\,, \quad (e,e') \longmapsto e + e'\,,
	\end{equation*}
	e una sezione $s:u \to \xi$ che identifica ogni punto di $B$
	come  l'\textbf{elemento neutro} sulla corrispondente fibra di $p$:
	più precisamente, per ogni $e,e' \in E$ vale
	\begin{equation*}
		s(p(e)) + e' = e'\,, \quad e' + s(p(e)) = e'\,.
	\end{equation*}
\end{df}

Ogni volta che abbiamo un oggetto gruppo $\xi$,
chiamiamo $s$ la sua \textbf{zero sezione}
e identifichiamo la base come un sottoinsieme dello spazio totale $B \subset E$.

Siamo interessati al caso in cui le fibre su $B$ sono \emph{spazi vettoriali},
in cui non ci siano ``salti di dimensione'' tra le fibre parametrizzate:
non consideriamo oggetti del tipo
\begin{equation*}
	\mathrm{pr}_{1} : E := \Set{(x,y) \in \R^{2} | xy=0} \longrightarrow \R\,.
\end{equation*}

\begin{df}
	Un \textbf{fibrato vettoriale di rango $n$} su $B$ è un oggetto 
	$\xi=(E,p)$ \emph{localmente banale} con fibra $F=\R^{n}$,
	cioè per ogni $b \in B$, esiste un intorno aperto $U$ di $b$ tale che
	$\left( p^{-1}(U) , p\vert_{p^{-1}(U)} \right)$
	è isomorfo al fibrato banale:
	\begin{equation}\label{banalizzazione}
		\begin{tikzcd}
			p^{-1}(U_{i}) \ar[rr, "\Phi_{i}"', "\simeq"] \ar[dr, "p"'] 
			& & U_{i} \times \R^{n} \ar[dl, "\mathrm{pr}_{1}"] 
			\ar[ll, bend right, "\Psi_{i}"', dashed] \\
			& U_{i} &
		\end{tikzcd}
	\end{equation}
	
	Lo spazio $E$ viene detto \textbf{spazio totale}, $B$ è detto \textbf{spazio base}
	e un aperto $U_{i}$ che soddisfa la condizione \eqref{banalizzazione} è detto
	\textbf{aperto banalizzante}.
\end{df}

Gli isomorfismi $\Phi_{i}$ nel diagramma \eqref{banalizzazione} si
chiamano \textbf{banalizzazioni} e inducono isomorfismi 
\emph{di spazi vettoriali} sulle fibre $p^{-1}(b)$, per tutti i $b \in B$.
Indichiamo con
\begin{equation*}
	\Phi_{i,b} := \Phi_{i}\vert_{p^{-1}(b)} : p^{-1}(b) \longrightarrow \{b\} \times \R^{n}\,,
	\quad \Phi_{i}(e) = \big( p(e), \phi_{i}(e) \big)\,.
\end{equation*}

Nell'intersezione di due aperti banalizzanti $U_{i} \cap U_{j}$ possiamo
\emph{cambiare banalizzazione}
\begin{equation*}
	\begin{tikzcd}[column sep=large]
		(U_{i} \cap U_{j}) \times \R^n  \ar[r, "\Psi_{j}"', "\simeq"]
            & p^{-1}(U_{i} \cap U_{j}) \ar[r, "\Phi_{i}"', "\simeq"]
            & (U_{i} \cap U_{j}) \times \R^n \\
            \textcolor{black!50}{(b, v)} \ar[rr, mapsto, black!50] 
            & & \textcolor{black!50}{(b, g_{ij}(b) \cdot v)} \\
            & U_{i} \cap U_{j} \ar[from=uu, crossing over, "p",  near end] 
            \ar[from=uul, crossing over] \ar[from=uur, crossing over]
            & \,,
	\end{tikzcd}
\end{equation*}
dove $g_{ij}(b) = \Phi_{i,b} \circ \Phi_{j,b}^{-1}$ è un isomorfismo lineare.
Otteniamo così delle funzioni
\begin{equation*}
	g_{ij} : U_{i} \cap U_{j} \longrightarrow \mathrm{GL}(n,\R)
\end{equation*}
dette \textbf{funzioni di transizione} del fibrato.

\begin{oss}
	Si può verificare che per ogni $b \in B$
	valgono le seguenti \textbf{condizioni di cociclo}:
	\begin{itemize}
		\item $g_{ji}(b) = \big( g_{ij}(b) \big)^{-1}$;
		\item per ogni tripla di indici $i,j,h$ si ha $g_{ij}(b) g_{jh}(b) = g_{ih}(b)$. 
	\end{itemize}
	Tutte le informazioni di un fibrato vettoriale possono essere 
	lette in termine delle funzioni di transizione $\{g_{ij}\}$ su
	un ricoprimento banalizzante fissato. Ad esempio,
	dato un morfismo di fibrato vettoriali $f: \xi \to \xi'$ su una stessa base $B$,
	possiamo considerare un ricoprimento banalizzante $\{U_{i}\}$ comune
	per entrambi i fibrati e ``leggere'' $f$ tramite le coordinate
	date dalle banalizzazioni
	\begin{equation*}
		\begin{tikzcd}[row sep=small]
			(U_{i} \cap U_{j}) \times \R^n  \ar[r, "\Psi_{j}"]
            & p^{-1}(U_{i} \cap U_{j}) \ar[r, "f"]
            & p'^{-1}(U_{i} \cap U_{j}) \ar[r, "\Phi_{i}'"]
            & (U_{i} \cap U_{j}) \times \R^k \\
            (b,v) \ar[rrr, mapsto]
            & & & \left(b, \Phi'_{i,b} \circ f_{b} \circ \Phi_{j,b}^{-1} (v) \right)\,,
		\end{tikzcd}
	\end{equation*}
	dove $f_{b} : E_{b} \to E'_{b}$. Otteniamo quindi delle funzioni
	\begin{equation*}
		\overline{f}_{ij} : U_{i} \cap U_{j} \longrightarrow \Hom_{\R}(\R^{n},\R^{k})\,,
		\quad \overline{f}_{ij}(b) = \Phi'_{i,b} \circ f_{b} \circ \Phi_{j,b}^{-1}
	\end{equation*}
	che nell'intersezione di tre aperti $U_{i} \cap U_{j} \cap U_{k}$ soddisfano
	\begin{equation}\label{cociclo}
		\begin{cases}
			\overline{f}_{ij} \circ g_{jk} &= \overline{f}_{ik}\,, \\
			g'_{ij} \circ \overline{f}_{jk} &= \overline{f}_{ik}\,.
		\end{cases}
	\end{equation}

	Viceversa, una famiglia di funzioni $ \{ f_{ij} \} $ che soddisfa
	le condizioni di cociclo \eqref{cociclo} inducono
	un morfismo di fibrati vettoriali $f : \xi \to \xi'$,
	definito su $U_{i}$ da $f_{i} := (\Phi'_{i})^{-1} \circ \overline{f}_{ii} \circ \Phi_{i}$.
\end{oss}

\begin{thm}
	Dati due fibrati $\xi, \xi'$ su $B$ e un ricoprimento $\{U_{i}\}$ che banalizza entrambi,
	allora $\xi \simeq \xi'$ se e solo se esistono $\lambda_{i}:U_{i} \to \mathrm{GL}(n,\R)$
	che ``collegano'' le funzioni di transizione nel seguente modo:
	\begin{equation*}
		g_{ij}'(b) = \lambda_{i}(b) ^{-1} g_{ij}(b) \lambda_{j}(b)\,,
		\quad \text{per ogni } b \in U_{i} \cap U_{j}\,.
	\end{equation*}
	\begin{proof}
		Dato un isomorfismo $f:\xi \simeq \xi'$, 
		otteniamo delle funzioni $f_{ij}:U_{i} \cap U_{j} \to \mathrm{GL}(n,\R)$
		che soddisfano \eqref{cociclo}. 
		Se per ogni $i$ poniamo $\lambda_{i} := f_{ii}^{-1}$,
		allora vale
		\begin{equation*}
			g'_{ij} = f_{ii} \circ f_{ij}^{-1}
			= f_{ii} \circ g_{ij} \circ f_{ii}^{-1}
			= \lambda_{i}^{-1} g_{ij} \lambda_{i}\,.
		\end{equation*}
		
		Viceversa, date $\{\lambda_{i}\}$ come sopra, per ogni $b \in U_{i} \cap U_{j}$
		definiamo $f_{ij}(b) := \lambda_{i}(b)^{-1} g_{ij}(b)$.
		Queste mappe soddisfano le condizioni \eqref{cociclo}, infatti
		\begin{align*}
			f_{ij} g_{jk} &= \lambda_{i}^{-1} g_{ij} g_{jk} 
			= \lambda_{i}^{-1}g_{ik} =  \overline{f}_{ik}\,, \\
			g'_{ij} f_{jk} &= g'_{ij} \lambda_{j}^{-1} g_{jk} 
			= \lambda_{i}^{-1} g_{ij} \lambda_{j} \lambda_{j}^{-1} g_{jk} \\
			&= \lambda_{i}^{-1} g_{ij} g_{jk} = \lambda_{i}^{-1} g_{ik} = f_{ik}\,.
		\end{align*}
		Questo significa che le $f_{ij}$ definiscono una mappa $f:\xi \to \xi'$
		che è un isomorfismo, la cui inversa è indotta dalla famiglia $h_{ij} := g'_{ij} \lambda_{j}$.
	\end{proof}
\end{thm}

\begin{thm}[\textbf{di esistenza}]
	Dato un ricoprimento $\{U_{i}\}$ di $B$ e delle funzioni 
	$g_{ij}:U_{i} \cap U_{j} \to \mathrm{GL}(n,\R)$ che soddisfano le \textbf{condizioni di cociclo},
	esiste un fibrato vettoriale di rango $n$ su $B$ che ha le $g_{ij}$ come
	funzioni di transizione.
	\begin{proof}[Idea della dimostrazione]
		Partendo da $\coprod_{i} U_{i} \times \R^{n}$,
		si identificano le coppie con indici diversi 
		tramite la relazione d'equivalenza
		\begin{equation*}
			(b,v)_{i} \sim (b, g_{ij}(b) \cdot v)\,, \quad
			\text{se } b \in U_{i} \cap U_{j}\,. \qedhere
		\end{equation*}
	\end{proof}
\end{thm}

\begin{ex}
	\begin{rmnumerate}
		\item Il \textbf{fibrato banale} di rango $n$ su 
		$B$ è $\mathrm{pr}_{1}:B \times \R^{n} \to \R^{n}$.
		
		\item Sia $\KK = \R$ oppure $\KK = \C$.
		Posto $V := \KK^{n}$, la \textbf{Grassmaniana} 
		\begin{equation*}
			Gr_{n}(V) := \Set{\text{sottospazi } k\text{-dimensionali di } V}
		\end{equation*}
		si può topologizzare asserendo che ogni $W \in Gr_{n}(V)$
		è individuato da un $k$-frame $(w_{1}, \dots, w_{k})$ di $W$
		a meno dell'azione di $\mathrm{GL}(k,\KK)$ su di lui.
		Identificando le $n$-uple di vettori come matrici $n \times n$,
		le $n$-uple di rango $k$ formano un aperto di $M_{n \times n}(\KK)$
		chiamato \textbf{varietà di Stiefel} $\Vv_{n,k}$. 
		
		La grassmaniana $Gr_{k}(V)$ è il quoziente di $\Vv_{n,k}$
		per l'azione di $\mathrm{GL}(n,\KK)$, che è libera,
		quindi $Gr_{k}(V)$ è una varietà di dimensione
		\begin{equation*}
			\dim Gr_{k}(V) = kn - k^2 = k(n-k)\,.
		\end{equation*}
		La grassmaniana $Gr_{k}(V)$ si può ricoprire con $\binom{n}{k}$ aperti
		che corrispondono alle matrici $M$ in cui $k$ colonne fissate sono indipendenti.
		Tramite $\mathrm{GL}(k,\KK)$, si riportano tali colonne a un blocco $I_{k}$
		e i $k(n-k)$ parametri rimanenti parametrizzano i punti di un aperto di $Gr_{k}(V)$.
		
		\textbf{Fibrato $\xi_{n,k}$} - costruiamo un fibrato su $Gr_{k}(V)$
		con spazio totale
		\begin{equation*}
			E(\xi_{n,k}) = \Set{(W,x) \in Gr_{k}(V) \times V | x \in W}
		\end{equation*}
		e proiezione $p(W,x)=W$. In questo modo,
		la fibra su $W$ si identifica con $W$ stesso.
		Se $k=1$, allora $Gr_{1}(V) = \PP(V) = \PP^{n-1}(\KK)$ e chiamiamo
		$\xi_{n,1}$ un \textbf{fibrato in rette}.
		
		\item Ponendo $n=2$ e $\KK = \R$ nell'esempio precedente,
		allora $Gr_{1}(\R^{2}) = \PP^{1}(\R) \simeq S^{1}$.
		Conosciamo due fibrati in rette sulla circonferenza:
		il cilindro $S^{1} \times \R$ e il nastro di Möebius.
		Vedremo che questi sono gli \emph{unici} fibrati in rette su $S^{1}$
		(a meno di isomorfismo). Chi è $\xi_{2,1}$?
		Consideriamo su $\PP^{1}(\R)$ i due aperti seguenti:
		\begin{equation*}
			U_{1} := \Set{ | x \in \R}\,, \quad
			U_{2} := \Set{ | y \in \R}\,.
		\end{equation*}
		Su questi aperti abbiamo le banalizzazioni
		\begin{equation*}
			\Psi_{1} : (p_{x}, \lambda p_{x} ) \longmapsto (p_{x}, \lambda) \in U_{1} \times \R\,,
			\quad \Psi_{2} : (q_{y}, \mu q_{y}) \longmapsto (q_{y},\mu) \in U_{2} \times \R\,.
		\end{equation*}
		Notiamo che l'intersezione $U_{1} \cap U_{2}$ è sconnessa,
		e su questo aperto la funzione di transizione è data da
		\begin{equation*}
			\begin{tikzcd}
				(q_{y},\mu) \ar[r, "\Psi_{2}"]
				& {\left({[1:y]}, \mu {[1:y]} \right)} \ar[r,equals]
				& f
			\end{tikzcd}
		\end{equation*}
		e dato che il determinante cambia segno nelle due componenti
		connesse, deduciamo che $\xi_{2,1}$ \textbf{non} è banale.
	\end{rmnumerate}
\end{ex}

\begin{exercise}
	Dimostrare che per ogni $n \ge 2$ il fibrato $\xi_{n,1}$ è non banale.
\end{exercise}

\begin{ex}
	Un fibrato in rette banale ha una sezione mai nulla
	\begin{equation*}
		B \longrightarrow B \times \R\,,
		\quad b \longmapsto (b,1)\,.
	\end{equation*}
\end{ex}

\begin{oss}
	Un fibrato di rango $n$ è banale se e solo se ammette
	$n$ sezioni linearmente indipendenti.
	Infatti, dato $\xi:B \times \R^{n} \to B$ possiamo definire
	se sezioni $s_{i}(b) := (b, e_{i})$. Viceversa,
	se abbiamo $s_{1}(b), \dots, s_{n}(b)$ linearmente indipendenti per ogni $b \in B$,
	allora possiamo banalizzare il fibrato tramite l'isomorfismo
	\begin{equation*}
		B \times \R^{n} \longrightarrow E\,,
		\quad (b, \lambda_{1}, \dots, \lambda_{n}) \longmapsto \sum_{i=1}^{n}\lambda_{i}s_{i}(b)\,.
	\end{equation*}
\end{oss}

\begin{exercise}
	L'oggetto $\xi_{n,k}$ definito sopra è un \textbf{fibrato vettoriale}.
	\begin{proof}[Una soluzione]
		Si verifica che una banalizzazione $p^{-1}(U) \to U \times \R^{k}$ è data da
		\begin{equation*}
			\left( 
				\begin{pmatrix}
					1 & & 0 & a_{1,k+1} & & a_{1,n} \\
					& & & & & \\
					0 & & 1 & a_{k,k+1} & & a_{k,n}
				\end{pmatrix},
				\lambda_{1}A_{1} + \dots + \lambda_{k}A_{k}
			\right)
		\end{equation*}
	\end{proof}
\end{exercise}

\begin{exercise}
	Determinare un ricoprimento banalizzante per $\tau_{n}$, 
	il fibrato tangente di $S^{n-1}$.
\end{exercise}

A partire da fibrati vettoriali noti,
possiamo ottenere nuovi fibrati vettoriali con costruzioni
simili a operazioni che conosciamo sugli spazi vettoriali:
\begin{itemize}
	\item \textbf{prodotto}: dati
	$\begin{tikzcd}
		\xi : E \ar[r, "p"] & B
	\end{tikzcd}$ e 
	$\begin{tikzcd}
		\xi' : E' \ar[r, "p'"] & B
	\end{tikzcd}$,
	definiamo $\xi \times \xi'$ come il fibrato $p \times p' : E \times E' \to B$;
	
	\item \textbf{somma di Whitney}: dati $\xi,\xi'$ sulla stessa base $B$,
	il fibrato $\xi \oplus \xi'$
	è dato dal pullback
	\begin{equation*}
		\begin{tikzcd}
			\Delta^{*}(\xi \times \xi') \ar[r] \ar[d]
			& E \times E' \ar[d, "p \times p'"] \\
			B \ar[r, "\Delta"] & B \times B\,,
		\end{tikzcd}
	\end{equation*}
	e si verifica che le fibre sono date da $\left( \xi \oplus \xi' \right)_{b} = E_{b} \oplus E_{b}'$.
\end{itemize}

\begin{exercise}
	Verificare che $T\,Gr_{k}(\R^{n}) \simeq \Hom_{\xi}(\xi_{n,k},\xi_{n,k}^{\perp})$.
\end{exercise}



