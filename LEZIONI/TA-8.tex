 %%% Lezione 8


\lecture[Lezione 8.]{2023-03-21}


\begin{df}
	Dato un fibrato vettoriale $\xi = (E,p)$,
	una \textbf{metrica} su $\xi$ è un prodotto scalare definito positivo
	sulle fibre $E_{b}$ che \emph{varia con continuità}: più precisamente,
	è una mappa continua $\mu:E \to \R$ che ristretta a ogni fibra
	$E_{b}$ è una forma quadratica definita positiva,
\end{df}

\begin{oss}
	Le metriche definite positive formano un insieme convesso.
\end{oss}


\begin{prop}
	Se $B$ ammette una partizione dell'unità (e.g. $B$ è paracompatto),
	ogni fibrato vettoriale su $B$ ammette una metrica.
	\begin{proof}
		Preso un ricoprimento banalizzante $\{U_{i}\}$ di $B$,
prendiamo una metrica $\mu_{i}$ definita positiva
su ciascun $\xi\vert_{U_{i}} \simeq U_{i} \times \R^{n}$.
Sia $\phi_{i}:U_{i}\to [0,1]$ una partizione dell'unità, i.e.
per ogni $b \in B$ si ha $\sum \phi_{i}(b) = 1$.
Allora $\mu := \sum_{i} \phi_{i}\mu_{i}$ è una metrica definita positiva su $\xi$.
	\end{proof}
\end{prop}	


\begin{cor}
	Ogni successione esatta\footnote{Una successione di fibrati è \textbf{esatta} se, fibra per fibra, dà origine a una successione esatta di spazi vettoriali.} di fibrati sulla stessa base $B$  \emph{spezza}.
	\begin{equation*}
		\begin{tikzcd}
			0 \ar[r]
			& \xi' \ar[r]
			& \xi \ar[r]
			& \xi'' \ar[r] \ar[l, bend right, dashed]
			& 0\,.
		\end{tikzcd}
	\end{equation*}
	\begin{proof}
		Si costruisce $\xi^{\perp} \subset \xi$ utilizzando la metrica.
		Siccome l'ultima mappa è surgettiva, la corrispondenza è un isomorfismo.
	\end{proof}
\end{cor}

\begin{df}
	Data una varietà differenziabile,
	una metrica euclidea su $\tau_{M}$ si chiama \textbf{metrica riemanniana}.
\end{df}

Sia $B$ uno spazio topologico, indichiamo con
\begin{equation*}
	\mathrm{Vect}(B) := \Set{\text{fibrati vettoriali su } B}/\simeq
\end{equation*}
Notiamo che due fibrati isomorfi $\xi \simeq \xi'$ su $B$
inducono pullback isomorfi: infatti, data $f:B' \to B$,
allora
\missingfigure{Diagrammino}
dunque possiamo definire $f^{*}h : E(f^{*}\xi) \to E(f^{*}\xi')$ mandando $(b',e) \mapsto (b',h(e))$;
questo mostra che $f^{*}\xi \simeq f^{*}\xi'$.
Il pullback è una costruzione funtoriale poiché vale
\begin{equation*}
	(g \circ f)^{*}\xi = (f^{*} \circ g^{*})\xi\,, \quad \cat{1}_{}^{*} = \cat{1}\,.
\end{equation*}
Deduciamo che $\mathrm{Vect}$ definisce un \emph{funtore controvariante} 
\begin{equation*}
	\mathrm{Vect} : \cat{Top}^{op} \longmapsto \CSet\,.
\end{equation*}



Indichiamo con $I = [0,1]$ l'intervallo unitario chiuso.
\begin{thm}
	La proiezione $\mathrm{pr}_{1}:X \times I \to X$ induce un isomorfismo
	$\mathrm{pr}_{1}^{*} : \mathrm{Vect}(X) \simeq \mathrm{Vect}(X \times I)$.
	\begin{proof}
		La parte facile consiste nel dimostrare l'\textbf{iniettività}:
		dato che $\mathrm{pr}_{1}$ ammette la sezione 
		\begin{equation*}
			s : X \longrightarrow X \times I\,, \quad
			x \longmapsto (x,0)\,,
		\end{equation*}
		allora per funtorialità vediamo che 
		\begin{equation*}
			s^{*} \circ \mathrm{pr}_{1}^{*} = \cat{1}\,,
		\end{equation*}
		da cui deduciamo l'iniettività.
		La surgettività è più difficile, e la dimostriamo in seguito.
	\end{proof}
\end{thm}

\begin{cor}
	Date $f_{0},f_{1}:B' \to B$ due funzioni omotope e $\xi = (E,p)$ fibrato su $B$,
	allora i pullback $f_{0}^{*}\xi$ e $f_{1}^{*}\xi$ sono isomorfi.
	\begin{proof}
		Consideriamo le inclusioni di $B'$ in $B' \times I$ al tempo $0$ e al tempo $1$:
		\begin{equation*}
			\begin{tikzcd}
				& B' \ar[dl, equals] \ar[dr, "f_{0}"] \ar[d, "\iota_{0}"] & \\
				B' 
				& B' \times I \ar[l] \ar[r, "H"]
				& B \\
				& B' \ar[ul, equals] \ar[ur, "f_{1}"] \ar[u, "\iota_{1}"] &
			\end{tikzcd}
		\end{equation*}
	\end{proof}
\end{cor}


\begin{cor}
	Se $f:B' \to B$ è un'equivalenza omotopica, allora $f^{*}$ 
	induce un isomorfismo $\mathrm{Vect}(B') \simeq \mathrm{Vect}(B)$.
	\begin{proof}
		Data $g:B \to B'$ inversa in omotopia di $f$, allora
		$g^{*} \circ f^{*} = \cat{1}$ e $f^* \circ g^{*} = \cat{1}$.
	\end{proof}
\end{cor}

Segue che il funtore dei fibrati vettoriali
discende alla categoria omotopica $\mathrm{Vect}:cat{hTop}^{op} \to \CSet$.

Quanto appena visto per i fibrati vettoriali può essere studiato
per fibrati topologici più generali, in cui la fibra $F = E_{b}$ può essere un \emph{gruppo}
qualsiasi.

\begin{df}
	Un \textbf{fibrato principale} su un gruppo topologico $G$ è uno spazio topologico $P$
	dotato di un'azione \emph{destra} di $G$ tale che
	\begin{itemize}
		\item l'azione di $G$ è libera;
		\item la proiezione sullo spazio delle orbite $p: P \to P/G$
		è un fibrato topologico con fibre omeomorfe a $G$.
	\end{itemize}
\end{df}

Dato un aperto banalizzante $U$ di un fibrato pr

\begin{df}
	Un'azione $P \times G \to P$ ha \textbf{sezioni locali} se, per ogni $x \in P/G$,
	esiste un intorno $U \ni x$ e una sezione $s:U \to P$.
\end{df}

\begin{prop}
	Se uno spazio $P$ ammette una $G$-azione destra libera con sezioni locali,
	allora $p:P \to P/G$ è un fibrato principale.
	\begin{proof}
		Preso $x \in P/G$ e una sezione $s:U \to P$ definita su un intorno di $x$,
		allora possiamo ottenere una banalizzazione locale definendo
		\begin{equation*}
			\psi : U \times G \longrightarrow p^{-1}(U)\,, \quad
			\psi(y,g) = s(y) \cdot g\,.
		\end{equation*}
	\end{proof}
\end{prop}

\begin{thm}
	Data una varietà differenziabile $P$ con un'azione libera di un gruppo di Lie compatto $G$,
	allora la proiezione $p$ ha sezioni locali e quindi $p:P \to P/G$ è un fibrato principale.
	\begin{proof}
		Non lo dimostriamo.
	\end{proof}
\end{thm}

\begin{ex}
	Se $G$ è un gruppo di Lie e $H \subset G$ un sottogruppo di Lie,
	allora $G \to G/H$ è un fibrato principale su $H$.
\end{ex}


Sia $G$ un gruppo che agisce (a destra) transitivamente su uno spazio $X$.
Se $x \in X$, considero la funzione $p(g) := x \cdot g$, che è surgettiva.
Denotiamo con $H \subset G$ lo stabilizzatore di $X$. Allora
\begin{equation*}
	Hg' = \Set{g \in G | x \cdot g = x \cdot g'}
\end{equation*}
per qualche $g'$ fissato è una classe laterale destra di $H$ in $G$,
quindi $p:G/H \to X$ è una bigezione continua.
Se anche l'inversa è continua (e.g. $G/H$ compatto, oppure $p$ aperta),
allora $p$ è un omeomorfismo.

\begin{ex}
	Detto $N \in S^{n-1}$ il polo nord, il gruppo ortogonale $O(n)$ agisce sulla sfera $S^{n-1}$ e
	\begin{equation*}
	 	\mathrm{Stab}(N) \simeq O(n-1)\,.
	 \end{equation*} 
	 Allora $O(n)/O(n-1) \simeq S^{n-1}$, da cui deduciamo che $O(n) \to S^{n-1}$
	 è un fibrato principale su $O(n-1)$.
\end{ex}

\begin{ex}
	sia $V_{k}'(\R^{n})$ lo spazio dei $k$-frame ortonormali in $\R^{n}$.
	Allora $O(n)$ agisce su $V_{n}'(\R^{n})$ e lo stabilizzatore in ogni $k$-frame
	può essere identificato con $O(n-k)$. Quindi $O(n) \to V_{k}'(\R^{n})$ è un
	fibrato principale con gruppo $O(n-k)$.
\end{ex}

\begin{ex}
	Il gruppo ortogonale $O(n)$ agisce sulle grassmaniane $Gr_{k}(\R^{n})$,
	in cui lo stabilizzatore delle prime $k$ colonne è formato da matrici a blocchi,
	pertanto può essere identificato con $O(k) \times O(n-k)$.
	La proiezione $O(n) \to Gr_{k}(\R^{n})$ è un fibrato principale con gruppo
	$O(k) \times O(n-k)$.
\end{ex}

%\begin{ex}
%	\begin{rmnumerate}
%		\item La proiezione $p : S^{2n+1} \to \C\PP^{n}$ sul quoziente rispetto all'azione di $S^{1}$
%		è un fibrato principale su $G = S^{1}$;
%		
%		\item se indichiamo con $V_{k}(\R^{n})$ i $k$-frame in $\R^{n}$,
%		allora la proiezione $p : V_{k}(\R^{n}) \to Gr_{k}(\R^{n})$ ha sezioni
%		locali rispetto all'azione di $G = \mathrm{GL}(k,\R)$;\todo{ricontrolla questi esempi, alla luce di quelli scritti prima (erano la correzione)}
%		
%		\item sia $V_{k}'(\R^{n})$ lo spazio dei $k$-frame ortonormali in $\R^{n}$.
%		Allora $p:V_{k}'(\R^{n}) \to Gr_{k}(\R^{n})$ è un fibrato preincipale su $O(k)$.
%	\end{rmnumerate}
%\end{ex}


\begin{prop}
	Un fibrato principale su $G$ è banale se e solo se $P \to P/G$ ha una sezione globale.
	\begin{proof}
		Ovviamente un fibrato banale ha sempre una sezione globale; viceversa,
		se abbiamo $s:B \to P$ una sezione, possiamo banalizzare $P$ tramite la mappa
		\begin{equation*}
			\psi : B \times G \longrightarrow P\,,
			\quad \psi(b,g) := s(b) \cdot g\,.
		\end{equation*}
	\end{proof}
\end{prop}


Nel caso in cui $G$ è discreto, la proiezione $p : P \to P/G$ è semplicemente un rivestimento
con gruppo $G$. 

Supponiamo che $B$ abbia rivestimento universale\footnote{è sufficiente $B$ connesso e semi-localmente semplicemente connesso.}. Allora il gruppo $G:=\pi_{1}(B,\ast)$ agisce liberamente
sulle fibre del rivestimento universale $\widetilde{B} \to B$, quindi possiamo identificare
$B = \widetilde{B}/G$. In generale, $\pi_{1}(B,\ast)$ agisce sulle fibre $F = p^{-1}(B)$ 
di un rivestimento di $p:E \to B$, quindi $F$ diventa un $G$-insieme.ì (con azione \emph{sinistra}).
Viceversa, dato un $G$-insieme $F$, si può costruire un rivestimento di $B$ con
fibra $G$ tramite il prodotto
\begin{equation*}
	\widetilde{B} \times_{G} F = \widetilde{B} \times F / \sim\,,
\end{equation*}
dove la relazione d'equivalenza identifica $(g,g\cdot z) \sim ($\todo{Non capisco nulla della notazione}


Dato che l'azione di $G$ sulle fibre del rivestimento è libera, $p^{-1}(x_{0})$
ha un unico rappresentante del tipo $(y_{0},z)$, con $z \in F$, quindi la fibra
si identifica con $F$.

\begin{exercise}
	$E'$ si ottiene dal rivestimento universale
	dividendo per la relazione $E=\widetilde{B}/\sim$ 
	che è data fibra per fibra da $g_{1} \cdot x = g_{2} \cdot x$,
	$x \in F$.
\end{exercise}

Tutto questo discorso si riassume nel seguente

\begin{thm}
	C'è un'equivalenza di categorie tra
	\begin{equation*}
		\Set{\pi_{1}(B,\ast)\text{-insiemi}}/\text{bigezioni } \pi_{1}(B,\ast)\text{-equivarianti}
	\end{equation*}
	e la categoria dei rivestimenti su $B$, a meno di isomorfismo.
\end{thm}

Vogliamo generalizzare queste costruzioni anche nel caso di gruppi \emph{non} discreti.
Se $\xi = (E,p)$ è un fibrato principale sul gruppo $G$, allora $F$ è un $G$-spazio (a \emph{sinistra})
e possiamo creare un fibrato associato a $\xi$ con fibra $F$ come prima:
posto $(g \cdot y,z) \sim (y, g \cdot z)$, poniamo
\begin{equation*}
	q : P \times_{G} F  := P \times F/\sim \longrightarrow B\,,
	\quad {[y,z]} \longmapsto p(y)\,.
\end{equation*}

Come prima, lefibre si identificano con $F$, infatti se $x_{0} \in B$ e $y_{0} \in p^{-1}(x_{0})$,
allora $q^{-1}(x_{0})$ contiene tutte le coppie del tipo $\Set{(g \cdot y_{0},g^{-1} \cdot z) | g \in G}$,
quindi abbiamo una mappa $q^{-1}(x_{0}) \simeq F$. L'inversa è data da $z \mapsto [(y_{0},z)]$,
che è un omeomorfismo.

Se $F$ è uno spazio vettoriale con $G$ che avisce liberamente su $F$,
allora otteniamo un nuovo fibrato vettoriale (e.g. quando $G = \mathrm{GL}(n,\R)$).
Viceversa, dato un fibrato $\xi = (E,p)$ di rango $n$,
gli si può associare un fibrato principale $P(\xi)$ su $G=\mathrm{GL}(n,\R)$:
prendiamo
\begin{equation*}
	P(\xi)_{b} = \Set{\text{basi in } F(\xi)_{b} = \mathrm{Iso}(\R^{n},E(\xi)_{b})}
\end{equation*}
in cui la topologia è la stessa che si ottiene attaccando i pezzi di fibrato banale
$$U_{i} \times \R^{n} \to U_{i} \times \mathrm{Iso}(\R^{n},\R^{n}) 
\simeq U_{i} \times \mathrm{GL}(n,\R)\,,$$
che sulle intersezioni è data da\todo{boh, non si capisce un cazzo quando parla.}

Date banalizzazioni $\Psi_{i}:U_{i} \times \R^{n} \to p^{-1}(U_{i})$,
sulle intersezioni abbiamo le mappe di transizione
\begin{equation*}
	(U_{i} \cap U_{j}) \times \R^{n} \longrightarrow (U_{i} \cap U_{j}) \times \R^{n} \,,
	\quad (x,y) \longmapsto (x, g_{ij}(x) y)\,,
\end{equation*}
dove $g_{ij}(x) \in \mathrm{GL}(n,\R)$. Allora incollo $U_{i} \times \mathrm{GL}(n,\R)$
con $U_{j} \times \mathrm{GL}(n,\R)$ sull'intersezione 
$(U_{i} \cap U_{j}) \times \mathrm{GL}(n,\R)$ tramite le stesse funzioni di transizione,
poiché le $\{g_{ij}\}$ soddisfano le condizioni di cociclo.
In questo modo otteniamo una corrispondenza $\xi \leftrightarrow P(\xi)$.

\begin{thm}
	La corrispondenza
	\begin{equation*}
		\mathrm{Princ}_{G} := \Set{G\text{-fibrati principali su } B}/\text{iso}
	\end{equation*}
	è un funtore controvariante $\mathrm{Princ}_{G}:\cat{Top}^{op} \to \CSet$.
\end{thm}

\begin{thm}
	C'è un isomorfismo di funtori $\mathrm{Princ}_{\mathrm{GL}(n,\R)} \simeq \mathrm{Vect}(B)$.
\end{thm}














