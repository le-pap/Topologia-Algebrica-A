 %%% Lezione 8


\lecture[Metrica su fibrato. Teorema di $I$-invarianza. Corollari. Fibrati principali e esempi. Azione transitiva su un insieme e fibrazione associata. Fibrati associati (prodotto bilanciato). Isomorfismo tra $\mathrm{Vect}$ e $\mathrm{Pring}_{\mathrm{GL}}$.]{2023-03-21}


\begin{df}
	Dato un fibrato vettoriale $\xi = (E,p)$,
	una \textbf{metrica} su $\xi$ è un prodotto scalare definito positivo
	sulle fibre $E_{b}$ che \emph{varia con continuità}: più precisamente,
	è una mappa continua $\mu:E \to \R$ che ristretta a ogni fibra
	$E_{b}$ è una forma quadratica definita positiva.
\end{df}

\begin{oss}
	Le metriche definite positive formano un insieme convesso.
\end{oss}


\begin{prop}
	Se $B$ ammette una partizione dell'unità (e.g. $B$ è paracompatto),
	ogni fibrato vettoriale su $B$ ammette una metrica.
	\begin{proof}
		Preso un ricoprimento banalizzante $\{U_{i}\}$ di $B$,
prendiamo una metrica $\mu_{i}$ definita positiva
su ciascun $\xi\vert_{U_{i}} \simeq U_{i} \times \R^{n}$.
Sia $\phi_{i}:U_{i}\to [0,1]$ una partizione dell'unità, i.e.
per ogni $b \in B$ si ha $\sum \phi_{i}(b) = 1$.
Allora $\mu := \sum_{i} \phi_{i}\mu_{i}$ è una metrica definita positiva su $\xi$.
	\end{proof}
\end{prop}	


\begin{cor}
	Ogni successione esatta di fibrati\footnote{Una successione di fibrati è \textbf{esatta} se, fibra per fibra, dà origine a una successione esatta di spazi vettoriali.} sulla stessa base $B$  \emph{spezza}:
	\begin{equation*}
		\begin{tikzcd}
			0 \ar[r]
			& \xi' \ar[r]
			& \xi \ar[r]
			& \xi'' \ar[r] \ar[l, bend right, dashed, "\alpha"']
			& 0\,.
		\end{tikzcd}
	\end{equation*}
	\begin{proof}
		Si costruisce $(\xi')^{\perp} \subset \xi$ utilizzando la metrica.
		Siccome l'ultima mappa è surgettiva, la composizione
		$(\xi')^{\perp} \subset \xi \to \xi''$ è un isomorfismo,
		perciò prendendo $\alpha$ come l'isomorfismo inverso
		concludiamo che la successione spezza.
	\end{proof}
\end{cor}

\begin{df}
	Data una varietà differenziabile $M$,
	una metrica euclidea su $\tau_{M}$ si chiama \textbf{metrica riemanniana}.
\end{df}

Dato $B$ uno spazio topologico, definiamo
\begin{equation*}
	\mathrm{Vect}(B) := \Set{\text{fibrati vettoriali su } B}/\simeq
\end{equation*}
e indichiamo con $\mathrm{Vect}_{n}(B)$ le clessi d'isomorfismo
dei fibrati di rango $n$.
Notiamo che due fibrati isomorfi $\xi \simeq \xi'$ su $B$
inducono pullback isomorfi: infatti, data $h : E(\xi) \simeq E(\xi')$
e una mappa $f:B' \to B$, 
possiamo definire $f^{*}h : E(f^{*}\xi) \to E(f^{*}\xi')$ mandando $(b',e) \mapsto (b',h(e))$;
questo mostra che $f^{*}\xi \simeq f^{*}\xi'$.
\begin{equation*}
\begin{tikzcd}
                                                        & E(f^{*}\xi')  \arrow[dd] \ar[rr] &                                    & E(\xi') \arrow[dd] \\
E(f^{*}\xi) \arrow[rr] \arrow[rd] \arrow[ru, "f^{*}h"', "\simeq"] &                                    & E(\xi) \arrow[from=ll, crossing over] \arrow[rd] \arrow[ru, "h"', "\simeq"] &                    \\
                                                        & B' \arrow[rr, "f"']                &                                    & B                 
\end{tikzcd}
\end{equation*}
Il pullback è una costruzione funtoriale poiché vale
\begin{equation*}
	(g \circ f)^{*}\xi = (f^{*} \circ g^{*})\xi\,, \quad \cat{1}_{B}^{*} = \cat{1}_{\mathrm{Vect}(B)}\,.
\end{equation*}
Deduciamo che $\mathrm{Vect}$ definisce un \emph{funtore controvariante} 
%\begin{equation*}
	$\mathrm{Vect} : \cat{Top}^{op} \longmapsto \CSet$.
%\end{equation*}




\begin{thm}\label{vect-htp}
	Sia $I = [0,1]$ l'intervallo unitario chiuso.
	La proiezione $\mathrm{pr}_{1}:X \times I \to X$ induce un isomorfismo
	$\mathrm{pr}_{1}^{*} : \mathrm{Vect}(X) \simeq \mathrm{Vect}(X \times I)$.
\end{thm}

Posticipiamo la dimostrazione di questo teorema,
ma ne deduciamo subito un'importante conseguenza.

\begin{cor}
	Date $f_{0},f_{1}:B' \to B$ due funzioni omotope e $\xi = (E,p)$ fibrato su $B$,
	allora i pullback $f_{0}^{*}\xi$ e $f_{1}^{*}\xi$ sono isomorfi.
	\begin{proof}
		Consideriamo le inclusioni di $B'$ in $B' \times I$ al tempo $0$ e al tempo $1$:
		\begin{equation*}
			\begin{tikzcd}[column sep=large, row sep=large]
				& B' \ar[dl, equals] \ar[dr, "f_{0}"] \ar[d, "\iota_{0}", hook] & \\
				B' 
				& B' \times I \ar[l] \ar[r, "H"]
				& B \\
				& B' \ar[ul, equals] \ar[ur, "f_{1}"'] \ar[u, "\iota_{1}"', hook] & \,.
			\end{tikzcd}
		\end{equation*}
		Siccome per il \hyperref[vect-htp]{Teorema~\ref{vect-htp}}
		la funzione $\mathrm{pr}_{1}^{*}$ è surgettiva,
		dall'equazione $\iota_{0}^{*}\mathrm{pr}_{1}^{*} = \cat{1}_{\mathrm{Vect}(B')} 
		=\iota_{1}^{*}\mathrm{pr}_{1}^{*}$ deduciamo che $\iota_{0}^{*}=\iota_{1}^{*}$,
		da cui segue
		\begin{equation*}
			f_{0}^{*}\xi = \iota_{0}^{*} \circ H^{*} \xi = \iota_{1}^{*} \circ H^{*} \xi = f_{1}^{*}\xi\,. \qedhere
		\end{equation*}
	\end{proof}
\end{cor}


\begin{cor}
	Se $f:B' \to B$ è un'equivalenza omotopica, allora $f^{*}$ 
	induce un isomorfismo $\mathrm{Vect}(B) \simeq \mathrm{Vect}(B')$.
	\begin{proof}
		Data $g:B \to B'$ inversa in omotopia di $f$, allora
		$g^{*} \circ f^{*} = \cat{1}_{\mathrm{Vect}(B')}$ e $f^* \circ g^{*} = \cat{1}_{\mathrm{Vect}(B)}$.
	\end{proof}
\end{cor}

Come conseguenza di questo fatto, il funtore dei fibrati vettoriali
discende alla categoria omotopica $\mathrm{Vect}:\cat{hTop}^{op} \to \CSet$.

\section{Fibrati principali}

Quanto appena visto per i fibrati vettoriali può essere studiato
per fibrati topologici più generali, in cui la fibra $F = E_{b}$ può essere un \emph{gruppo}
qualsiasi.

\begin{df}
	Un \textbf{fibrato principale} su un gruppo topologico $G$ è uno spazio topologico $P$
	dotato di un'azione \emph{destra} di $G$ tale che
	\begin{itemize}
		\item l'azione di $G$ è libera;
		\item la proiezione sullo spazio delle orbite $p: P \to P/G$
		è un fibrato topologico con fibre omeomorfe a $G$.
	\end{itemize}
\end{df}

Detta $m:G \times G \to G$ l'operazione del gruppo,
su un aperto banalizzante $U$ di un fibrato principale
il seguente quadrato commuta:
\begin{equation*}
	\begin{tikzcd}[column sep=large]
		p^{-1}(U) \times G \ar[r, "\Phi \times \cat{1}_{G}"] \ar[d] 
		& U \times G \times G \ar[d, "\cat{1}_{U} \times m"] \\
		p^{-1}(U) \ar[r, "\Phi"] & U \times G\,.
	\end{tikzcd}
\end{equation*}

\begin{df}
	Un'azione $P \times G \to P$ ha \textbf{sezioni locali} se, per ogni $x \in P/G$,
	esiste un intorno $U$ di  $x$ e una sezione $s:U \to P$.
\end{df}

\begin{prop}
	Se uno spazio $P$ ammette una $G$-azione destra libera con sezioni locali,
	allora $p:P \to P/G$ è un fibrato principale.
	\begin{proof}
		Preso $x \in P/G$ e una sezione $s:U \to P$ definita su un intorno di $x$,
		allora possiamo ottenere una banalizzazione locale definendo
		\begin{equation*}
			\Psi : U \times G \longrightarrow p^{-1}(U)\,, \quad
			\Psi(y,g) = s(y) \cdot g\,.	\qedhere
		\end{equation*}
	\end{proof}
\end{prop}

\begin{thm}
	Data una varietà differenziabile $P$ con un'azione libera di un gruppo di Lie compatto $G$,
	allora la proiezione $p$ ha sezioni locali e quindi $p:P \to P/G$ è un fibrato principale.
	\begin{proof}
		Non lo dimostriamo.
	\end{proof}
\end{thm}

\begin{ex}
	Se $G$ è un gruppo di Lie e $H \subset G$ un sottogruppo di Lie,
	allora $G \to G/H$ è un fibrato principale su $H$.
\end{ex}

\begin{ex}
	% Per ogni $n \ge 0$, l
	La proiezione $p : S^{2n+1} \to \C\PP^{n}$
	è un fibrato principale su $G=S^{1}$.
\end{ex}

\begin{exercise}
	L'azione di $\mathrm{GL}(k,\R)$ su $p:\Vv_{n,k} \to Gr_{k}(\R^{n})$
	ha sezioni locali, quindi $p$ è un fibrato principale con fibra $\mathrm{GL}(k,\R)$.
\end{exercise}

Sia $G$ un gruppo che agisce (a destra) transitivamente su uno spazio $X$.
Se $x \in X$, considero la funzione $p(g) := x \cdot g$, che è surgettiva.
Denotiamo con $H \subset G$ lo stabilizzatore di $X$. 
A $g' \in G$ fissato, l'insieme
$\Set{g \in G | x \cdot g = x \cdot g'}$
è una classe laterale destra di $H$ in $G$,
cioè
\begin{equation*}
	\Set{g \in G | x \cdot g = x \cdot g'} = Hg'\,,
\end{equation*}
quindi $p:G/H \to X$ è una bigezione continua.
Se anche l'inversa è continua (e.g. $G/H$ compatto, oppure $p$ aperta),
allora $p$ è un omeomorfismo.

\begin{ex}
	Detto $N \in S^{n-1}$ il polo nord, il gruppo ortogonale $O(n)$ agisce sulla sfera $S^{n-1}$ 
	e il suo stabilizzatore è dato da
	\begin{equation*}
	 	\mathrm{Stab}(N) \simeq 
	 	\Set{
	 	\begin{pmatrix}
	 		& & & 0 \\
	 		& M & & \vdots \\
	 		& & & 0 \\
	 		0 & \dots & 0 & 1
	 	\end{pmatrix} | M \in O(n-1)
	 	}
	 	\simeq O(n-1)\,.
	 \end{equation*} 
	 Allora $O(n)/O(n-1) \simeq S^{n-1}$, da cui deduciamo che $O(n) \to S^{n-1}$
	 è un fibrato principale su $O(n-1)$.
\end{ex}

\begin{ex}
	sia $V_{k}'(\R^{n})$ lo spazio dei $k$-frame ortonormali in $\R^{n}$.
	Allora $O(n)$ agisce su $V_{k}'(\R^{n})$ e lo stabilizzatore in ogni $k$-frame
	può essere identificato con $O(n-k)$. Quindi $O(n) \to V_{k}'(\R^{n})$ è un
	fibrato principale con gruppo $O(n-k)$.
\end{ex}

\begin{ex}
	Il gruppo ortogonale $O(n)$ agisce sulle grassmaniane $Gr_{k}(\R^{n})$,
	in cui lo stabilizzatore delle prime $k$ colonne è formato da matrici a blocchi,
	pertanto può essere identificato con $O(k) \times O(n-k)$.
	La proiezione $O(n) \to Gr_{k}(\R^{n})$ è un fibrato principale con gruppo
	$O(k) \times O(n-k)$.
\end{ex}

%\begin{ex}
%	\begin{rmnumerate}
%		\item La proiezione $p : S^{2n+1} \to \C\PP^{n}$ sul quoziente rispetto all'azione di $S^{1}$
%		è un fibrato principale su $G = S^{1}$;
%		
%		\item se indichiamo con $V_{k}(\R^{n})$ i $k$-frame in $\R^{n}$,
%		allora la proiezione $p : V_{k}(\R^{n}) \to Gr_{k}(\R^{n})$ ha sezioni
%		locali rispetto all'azione di $G = \mathrm{GL}(k,\R)$;\todo{ricontrolla questi esempi, alla luce di quelli scritti prima (erano la correzione)}
%		
%		\item sia $V_{k}'(\R^{n})$ lo spazio dei $k$-frame ortonormali in $\R^{n}$.
%		Allora $p:V_{k}'(\R^{n}) \to Gr_{k}(\R^{n})$ è un fibrato preincipale su $O(k)$.
%	\end{rmnumerate}
%\end{ex}


\begin{prop}
	Un fibrato principale su $G$ è banale se e solo se $P \to P/G$ ha una sezione globale.
	\begin{proof}
		Ovviamente un fibrato banale ha sempre una sezione globale; viceversa,
		se abbiamo $s:P/G \to P$ una sezione, possiamo banalizzare $P$ tramite la mappa
		\begin{equation*}
			\psi : P/G \times G \longrightarrow P\,,
			\quad \psi(b,g) := s(b) \cdot g\,. \qedhere
		\end{equation*}
	\end{proof}
\end{prop}

\begin{ex}
	Se $G$ è un gruppo discreto, la proiezione $p : P \to P/G$ 
	è semplicemente un rivestimento con gruppo $G$. 
\end{ex}

Supponiamo che $B$ abbia rivestimento universale\footnote{\`{E} sufficiente $B$ connesso e semi-localmente semplicemente connesso.} $p:\widetilde{B} \to B$. 
Allora il gruppo $G:=\pi_{1}(B,\ast)$ agisce liberamente
sulle fibre di $p$, quindi possiamo identificare $B = \widetilde{B}/G$. 
In generale, $\pi_{1}(B,\ast)$ agisce sulle fibre $F \simeq p^{-1}(\ast)$ 
di un qualsiasi rivestimento di $p:E \to B$, 
quindi $F$ diventa un $G$-insieme (con azione \emph{sinistra}).
Viceversa, dato un $G$-insieme $F$, si può costruire un rivestimento di $B$ con
fibra $F$ tramite il prodotto
\begin{equation*}
	\widetilde{B} \times_{G} F := \widetilde{B} \times F / \sim\,,
\end{equation*}
dove la relazione d'equivalenza identifica le coppie
$(b,g \cdot z) \sim (b \cdot g, z)$ al variare di $g \in G$.
La classe di una coppia $(y,z) \in \widetilde{B} \times F$ è data da
\begin{equation*}
	[(y,z)] = \Set{(y \cdot g, g^{-1} \cdot z) | g \in G}\,,
\end{equation*}
e definendo la proiezione $\widetilde{p}:\widetilde{B} \times_{G} F \to B$
come $\widetilde{p}(y,z) = p(y)$, la fibra su $b_{0} \in B$ è data da
$\widetilde{p}^{-1}(b_{0}) = \Set{[(y,z)] | p(y) = b_{0}}$.
Dato che l'azione di $G$ sulle fibre del rivestimento è libera, 
fissato $y_{0} \in p^{-1}(b_{0})$, ogni punto di $\widetilde{p}^{-1}(b_{0})$
ha un unico rappresentante del tipo $(y_{0},z)$, %con $z \in F$, 
quindi la fibra si identifica con
\begin{equation*}
	\widetilde{p}^{-1}(b_{0}) = \Set{(y_{0},g \cdot z) | z \in F} % = G \times_{G} F 
	\simeq F\,.
\end{equation*}


%
%\begin{exercise}
%	$E'$ si ottiene dal rivestimento universale
%	dividendo per la relazione $E=\widetilde{B}/\sim$ 
%	che è data fibra per fibra da $g_{1} \cdot x = g_{2} \cdot x$,
%	$x \in F$.
%\end{exercise}

Tutto questo discorso si riassume nel seguente

\begin{thm}
	C'è un'equivalenza di categorie tra
	\begin{equation*}
		\Set{\pi_{1}(B,\ast)\text{-insiemi}}/ \Set{\text{bigezioni } \pi_{1}(B,\ast)\text{-equivarianti}}
	\end{equation*}
	e la categoria delle classi di isomorfismo di rivestimenti su $B$.
\end{thm}

Vogliamo generalizzare queste costruzioni anche nel caso di gruppi \emph{non} discreti.
Se $\xi = (P,p)$ è un fibrato principale con gruppo $G$ e
$F$ è un $G$-spazio (a \emph{sinistra}),
allora possiamo creare un fibrato principale associato a $\xi$ con fibra $F$ come prima:
identificando le coppie $(y \cdot g,z) \sim (y, g \cdot z)$
al variare di $g \in G$, posto $P \times_{G} F := P \times F/\sim$
consideriamo la mappa
\begin{equation*}
	q : P \times_{G} F  % := P \times F/\sim 
	\longrightarrow B\,,
	\quad {[(y,z)]} \longmapsto p(y)\,.
\end{equation*}

Come prima, le fibre si identificano con $F$: infatti, se $b_{0} \in B$ e $y_{0} \in p^{-1}(b_{0})$,
allora $q^{-1}(b_{0})$ contiene tutte le coppie del tipo $(y_{0} \cdot g ,g^{-1} \cdot z)$,
le quali hanno un unico rappresentante del tipo $(y_{0},z)$.
Questo definisce una mappa $q^{-1}(b_{0}) \to F$, con inversa $z \mapsto [(y_{0},z)]$,
che è un omeomorfismo.

\begin{df}
	Dato $B$ uno spazio topologico paracompatto, indichiamo con
	$\mathrm{Princ}_{G}(B)$ le classi di isomorfismo dei fibrati principali 
	con gruppo $G$ su base $B$. Questa associazione induce
	funtore controvariante $\mathrm{Princ}_{G}:\cat{Top}^{op} \to \CSet$,
	che sulle mappe $f:B' \to B$ è definito dal pullback 
	$\mathrm{Princ}_{G}(f) : \xi \mapsto f^{*}\xi$.
\end{df}


%%Se $F$ è uno spazio vettoriale e $G$ che agisce liberamente su $F$,
%%allora otteniamo un nuovo fibrato vettoriale (e.g. quando $G = \mathrm{GL}(n,\R)$).
%%Viceversa, dato un fibrato $\xi = (E,p)$ di rango $n$,
%%gli si può associare un fibrato principale $P(\xi)$ su $G=\mathrm{GL}(n,\R)$:
%%prendiamo
%%\begin{equation*}
%%	P(\xi)_{b} = \Set{\text{basi in } F(\xi)_{b} = \mathrm{Iso}(\R^{n},E(\xi)_{b})}
%%\end{equation*}
%%in cui la topologia è la stessa che si ottiene attaccando i pezzi di fibrato banale
%%$$U_{i} \times \R^{n} \to U_{i} \times \mathrm{Iso}(\R^{n},\R^{n}) 
%%\simeq U_{i} \times \mathrm{GL}(n,\R)\,,$$
%%che sulle intersezioni è data da\todo{boh, non si capisce un cazzo quando parla.}

Se $F$ è uno spazio vettoriale e $G$ che agisce liberamente su $F$ (e.g. quando $G = \mathrm{GL}(n,\R)$),
allora otteniamo un fibrato vettoriale
\begin{equation*}
	q : P \times_{G} F \longrightarrow B
\end{equation*}
che si dirà \emph{associato} al fibrato principale $p:P \to B$.
Notiamo che se $p$ ha funzioni di transizione $\{g_{ij}\}$,
% $\Set{g_{ij} : U_{i} \cap U_{j} \to \mathrm{GL}(n,\R)}$,
allora la coppia
\begin{equation*}
	(y,g)_{i} \in (U_{i} \cap U_{j}) \times G \subset U_{i} \times G
\end{equation*}
si identifica con $(x,g_{ij}(x) \cdot g)_{j} \in U_{j} \times G$.
Deduciamo così che la banalizzazione $p^{-1}(U_{i}) \simeq U_{i} \times G$
dà una banalizzazione $q^{-1}(U_{i}) \simeq U_{i} \times (G \times_{G} F)$.
Ora, identificando 
\begin{equation*}
	G \times_{G} F \xrightarrow{\sim} F\,, \quad
	[(g,z)] = [(1,g \cdot z)] \longmapsto g \cdot z\,,
\end{equation*}
si vede che $q:P \times_{G} F \to B$ ha esattamente le stesse funzioni di transizione di $P$.\footnote{Una costruzione alternativa del fibrato associato si può ottenere seguendo la tecnica del \hyperref[teo-esistenza-fibrato]{Teorema di esistenza}: i fibrati banali $U_{i} \times F$ si incollano facendo agire $G$ su $F$
tramite le stesse funzioni di transizione di $P$; questa costruzione soddisfa le condizioni ci cociclo e quindi produce un fibrato.} Si noti che un isomorfismo di fibrati principali $P \simeq P'$
induce un isomorfismo sui fibrati vettoriali associati 
$P \times_{G} F \simeq P' \times_{G} F$ 
(sfruttando ad esempio il \hyperref[iso-fibrati-vettoriali]{Teorema~\ref{iso-fibrati-vettoriali}}).


%Date banalizzazioni $\Psi_{i}:U_{i} \times \R^{n} \to p^{-1}(U_{i})$,
%sulle intersezioni abbiamo le mappe di transizione
%\begin{equation*}
%	(U_{i} \cap U_{j}) \times \R^{n} \longrightarrow (U_{i} \cap U_{j}) \times \R^{n} \,,
%	\quad (x,y) \longmapsto (x, g_{ij}(x) y)\,,
%\end{equation*}
%dove $g_{ij}(x) \in \mathrm{GL}(n,\R)$. Allora incollo $U_{i} \times \mathrm{GL}(n,\R)$
%con $U_{j} \times \mathrm{GL}(n,\R)$ sull'intersezione 
%$(U_{i} \cap U_{j}) \times \mathrm{GL}(n,\R)$ tramite le stesse funzioni di transizione,
%poiché le $\{g_{ij}\}$ soddisfano le condizioni di cociclo.
%In questo modo otteniamo una corrispondenza $\xi \leftrightarrow P(\xi)$.

Viceversa, dato un fibrato vettoriale $\xi = (E,p)$ su $B$ di rango $n$,
gli si può associare un fibrato principale $P(\xi)$ con gruppo $G=\mathrm{GL}(n,\R)$
con la seguente costruzione: su ogni banalizzazione $p^{-1}(U_{i}) \simeq U_{i} \times \R^{n}$
si sostituisce $\R^{n}$ con $G$ e si considerano le stesse funzioni di transizione
\begin{equation*}
	\Phi_{j} \circ \Psi_{i} : (U_{i} \cap U_{j}) \times \mathrm{GL}(n,\R) 
	\longrightarrow (U_{i} \cap U_{j}) \times \mathrm{GL}(n,\R)\,,
	\quad (x,M)_{i} \longmapsto (x, g_{ij}(x) \cdot M)_{j}\,.
\end{equation*}
Questa costruzione produce un fibrato con fibra $F=\mathrm{GL}(n,\R)$,
che è principale tramite l'azione di $G$ a destra.
Anche in questo caso se abbiamo un isomorfismo $\xi \simeq \xi'$,
questo induce $P(\xi) \simeq P(\xi')$ tra i fibrati principali associati.

Si verifica che per ogni fibrato vettoriale $\xi$ si ha 
$P(\xi) \times_{\mathrm{GL}(n,\R)} \R^{n} \simeq \xi$,
e d'altra parte per ogni fibrato principale $P$ sul gruppo $\mathrm{GL}(n,\R)$,
vale $P(P \times_{\mathrm{GL}(n,\R)} \R^{n}) \simeq P$. 
Tutto il discorso precedente si riassume dunque nel seguente

\begin{thm}
	La corrispondenza $\xi \mapsto P(\xi)$ dà una bigezione di insiemi %un isomorfismo di funtori 
	$\mathrm{Vect}_{n}(B) \simeq \mathrm{Princ}_{\mathrm{GL}(n,\R)}(B)$ che è naturale in $B$.
	Questo significa che induce un isomorfismo di funtori 
	$\mathrm{Vect}_{n} \simeq \mathrm{Princ}_{\mathrm{GL}(n,\R)}$.
\end{thm}

%\begin{thm}
%	La corrispondenza
%	\begin{equation*}
%		\mathrm{Princ}_{G}(B) := \Set{G\text{-fibrati principali su } B}/\text{iso}
%	\end{equation*}
%	è un funtore controvariante $\mathrm{Princ}_{G}:\cat{Top}^{op} \to \CSet$.
%\end{thm}














