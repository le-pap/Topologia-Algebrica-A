 %%% Lezione 14


\lecture[Successione spettrale e successione della coppia. Fibrati di Serre e teorema sulla successione spettrale di omologia associata ad un fibrato di Serre. Applicazione al calcolo dell'omologia di alcuni gruppi $SU(n)$. Applicazione al calcolo di alcuni gruppi di omologia dello spazio dei loop su uno spazio $X$ $n$-connesso.]{2023-04-18}

\begin{ex}[Successione di una coppia]
	Consideriamo uno spazio topologico $X$, su cui prendiamo la filtrazione
	\begin{equation*}
		\emptyset \subset Y \subset X\,,
	\end{equation*}
	o in altri termini consideriamo la coppia $(X,Y)$.
	Questo determina una filtrazione sulle catene singolari di $X$,
	dalla quale otteniamo la successione spettrale
	\begin{equation*}
		E^{1}_{p,q} = H_{p+q}(X_{p},X_{p-1}) = 
		\begin{cases}
			H_{q}(Y) \,, \quad &\text{se } p=0\,; \\
			H_{q+1}(X,Y) \,, \quad &\text{se } p=1\,\\
			0\,, \quad &\text{altrimenti.}
		\end{cases}
	\end{equation*}
	\missingfigure{Fai disegno}
	La mappa di bordo sulla prima pagina è
	\begin{equation*}
		d^{1}: E^{1}_{1,q} \longrightarrow E^{1}_{0,q}\,,
	\end{equation*}
	e si può verificare che questo è proprio l'omomorfismo di connessione della coppia
	\begin{equation*}
		\de : H_{q+1}(X,Y) \longrightarrow H_{q}(Y)\,.
	\end{equation*}
	
	Viceversa, supponiamo di avere il dato della successione spettrale;
	vogliamo ricostruire la successione della coppia.
	Il termine limite è per definizione
	\begin{equation*}
		E^{\infty}_{0,q} =
		\frac{\im \left( H_{q}(Y) \longrightarrow H_{q}(X) \right)}{\im \left( H_{q}(\emptyset) \longrightarrow H_{q}(X) \right)}
		= \im \left( i: H_{q}(Y) \longrightarrow H_{q}(X) \right)\,,
	\end{equation*}
	e l'altro termine non nullo è
	\begin{equation*}
		E^{\infty}_{1,q} = \frac{H_{q+1}(X)}{\im \left( i: H_{q+1}(Y) \longrightarrow H_{q+1}(X) \right)}\,.
	\end{equation*}
	Confrontiamo $E^{\infty}$ con la pagina $E^{2}$:\todo{finisci}
\end{ex}




\section{Successioni spettrali di Serre}

Ricordiamo la seguente definizione.

\begin{df}
	Una mappa $\pi : E \to B$ è una \textbf{fibrazione di Serre} se
	vale la \eqref{HLP} sui cubi:
	\begin{equation}\label{Serre-HLP}
		\begin{tikzcd}
			I^{n} \times \{0\} \ar[r] \ar[d, "\iota_{0}"', hook] & E \ar[d, "\pi"] \\
			I^{n+1} \ar[r] \ar[ur, dashed] & B\,. \tag{HLP'}
		\end{tikzcd}
	\end{equation}
\end{df}

Se $B$ è uno spazio connesso per archi che ammette un rivestimento universale $\widetilde{B}$,
allora per il \textbf{Teorema di approssimazione} possiamo trovare un'equivalenza
omotopica debole $f: B' \to B$, con $B'$ un CW complesso. Siccome le fibrazioni
sono una classe chiusa per pullback, allora anche $f^{*}E \to B'$ 
è una fibrazione di Serre, pertanto assumeremo 
senza perdita di generalità che la base $B$ sia un CW complesso.

Fissato $x \in B$, consideriamo  $E_{x} = \pi^{-1}(x)$ e
a questa fibra associamo $H_{*}(E_{x})$. 
Adesso vorremmo associare un omomorfismo 
$\gamma_{*} : H_{*}(E_{x}) \to H_{*}(E_{x'})$
per ogni cammino
$\gamma : x \to x'$ in $B$. 
Presa $\alpha_{0} : A_{x} \to E_{x}$ un'approssimazione CW della fibra su $x$,
possiamo usare la \eqref{Serre-HLP} induttivamente sullo scheletro di $A_{x}$
per poter sollevare l'omotopia 
\begin{equation*}
	\Gamma: A_{x} \times I \longrightarrow B\,, \quad
	\Gamma(z,t) := \gamma(t)\,,
\end{equation*}
per ottenere così una mappa diagonale
\begin{equation*}
	\begin{tikzcd}
		A_{x} \times \{0\} \ar[r, "\alpha_{0}"] \ar[d, "\iota_{0}"', hook] & E \ar[d, "\pi"] \\
		A_{x} \times I \ar[r] \ar[ur, dashed, "\alpha"] & B\,. 
	\end{tikzcd}
\end{equation*}
Allora $\alpha_{1} :A_{x} \to E_{x'}$ induce in omotopia un omomorfismo
$\gamma_{*} : H_{*}(E_{x}) \to H_{*}(E_{x'})$.
Notiamo inoltre che se $\gamma \sim \gamma'$ sono omotopi \emph{a estremi fissati},
allora producono lo stesso omomorfismo in omologia; in particolare,
se $x=x'$, notiamo allora che $\pi_{1}(B,x)$ agisce su
$H_{*}(E_{x})$ e si verifica che $\gamma$ è costante induce $\gamma_{*} = \cat{1}_{H_{*}(E_{x})}$.
Questo appena descritto è un esempio di \emph{sistema locale}.

\begin{df}
	Un \textbf{sistema locale di gruppi} $\Gg = \Set{G_{x},\tau_{\gamma}}$ 
	su uno spazio topologico $X$ è un funtore che associa a ogni punto $x \in X$
	un gruppo $G_{x}$ e a ogni cammino $\gamma : I \to X$ da $x_{0}$ a $x_{1}$
	associa un omomorfismo
	\begin{equation*}
		\tau_{\gamma} : G_{x_{0}} \longrightarrow G_{x_{1}}
	\end{equation*}
	che dipende solamente dalla classe
	\begin{equation*}
		[\gamma] \in [(I, \de I) ; (X, \{x_{0}, x_{1} \})]\,,
	\end{equation*}
	con la condizione che se $\gamma$ è costante in $x$, 
	allora l'omomorfismo associato è $\tau_{\gamma} = \cat{1}_{G_{x}}$.
\end{df}

\begin{oss}
	Il gruppo fondamentale $\pi_{1}(B,x)$ agisce (da \emph{sinistra})
	sul gruppo $G_{x}$.
\end{oss}


\begin{oss}
	Siccome ogni cammino $\gamma$ può essere ripercorso in senso
	contrario e produrre il cammino opposto $\overline{\gamma}$
	in modo che la composizione $\gamma \overline{\gamma}$ sia omotopo a una costante,
	deduciamo che ogni $\tau_{\gamma}$ è un isomorfismo.
	In particolare, se $X$ è connesso per archi, allora tutti i $G_{x}$
	di un sistema locale sono isomorfi.
\end{oss}

\begin{df}
	Se l'azione di $\pi_{1}(B,x)$ su $G_{x}$ è banale,
	allora diremo che il sistema locale $\Gg$ è \textbf{banale}.
\end{df}

\begin{ex}
	Ogni sistema locale su uno spazio semplicemente connesso è banale.
\end{ex}


Supponiamo ora che $X$ sia uno spazio connesso per archi
che ammette rivestimento universale $\widetilde{X}$.
Allora $\pi_{1}(X,x_{0})$ agisce (da \emph{destra})
su $\widetilde{X}$ per \textbf{traslazione}, ovvero
\begin{equation*}
	\widetilde{X} \times \pi_{1}(X,x_{0})\,, \quad
	(x,\gamma) \longmapsto x'
\end{equation*}
dove $x'$ è \todo{noccapito}

In particolare questo induce un'azione \emph{destra} di $\pi_{1}(X,x_{0})$
sulle catene singolari $C_{*}(\widetilde{X})$ per traslazione,
e si verifica che questa azione commuta con il differenziale $d$.

\begin{df}
	Dato $x_{0}$ in $X$ connesso per archi, il complesso
	\begin{equation*}
		C_{*}(X;\Gg) := C_{*}(\widetilde{X}) \otimes_{\Z[\pi_{1}(X,x_{0})]} G_{x_{0}}
	\end{equation*}
	viene chiamato \textbf{complesso delle catene singolari a coefficienti in $\Gg$},
	il cui bordo è indotto da $d:C_{*}(\widetilde{X}) \to C_{*}(\widetilde{X})$.
	Passando al duale, definiamo le \textbf{cocatene singolari a coefficienti in $\Gg$}
	come
	\begin{equation*}
		C^{*}(X;\Gg) := \Hom_{\Z[\pi_{1}(X,x_{0})]}\left(C_{*}(\widetilde{X};G_{x_{0}}\right)\,.
	\end{equation*}
	L'\textbf{omologia}, risp. la \textbf{coomologia}, a coefficienti
	in un sistema locale $\Gg$ sarà indicata con
	\begin{equation*}
		H_{*}(X;\Gg) := H_{*}(C_{*}(X;\Gg))\,, \quad
		\text{resp. } H^{*}(X;\Gg) := H^{*}(C^{*}(X;\Gg))\,.
	\end{equation*}
\end{df}


\begin{thm}[\textbf{Successione spettrale di Serre}]\label{Serre-SS}
	Sia $M$ uno $\Z$-modulo.\todo{Oppure modulo su un qualunque anello?}	
	Data una fibrazione di Serre $\pi:E \to B$,
	esiste una successione spettrale $\Set{(E^{r}_{p,q},d^{r})}_{r \ge 2}$
	concentrata nel primo quadrante data da
	\begin{equation*}
		E_{p,q}^{r} = H_{p}\left(B; \Set{H_{q}(E_{x};M)}\right)\,,
	\end{equation*}
	convergente a $E^{\infty}_{p,q} = F_{p}H_{p+q}(E;M)$,
	per una qualche filtrazione $F$ dell'omologia 
	$H_{*}(E;M)$.
\end{thm}

\begin{oss}
	Si noti che nel \hyperref[Serre-SS]{Teorema~\ref{Serre-SS}}
	non vengono nemmeno citati i differenziali della successione spettrale!
\end{oss}

Prima di dimostrare questo risultato,
vediamo alcuni esempi per capirne il funzionamento.

\begin{ex}
	Sia $SU(n)$ il gruppo delle matrici complesse unitarie con determinante $1$.
	Ad esempio, per $n=2$ abbiamo
	\begin{equation*}
		SU(2) : \Set{ \begin{bmatrix} \alpha & \beta \\ \overline{\beta} & \overline{\alpha} \end{bmatrix}	
		| {|alpha|}^{2}	+ {|\beta|}^{2} = 1 }
		\simeq S^{3}\,.
	\end{equation*}
	La topologia di $SU(3)$ invece è più complicata
	e usiamo il \hyperref[Serre-SS]{Teorema~\ref{Serre-SS}}
	per studiarne l'omologia. La mappa 
	
	In maniera analoga possiamo studiare l'omologia
	di $SU(4)$, infatti presa la fibrazione
	\begin{equation*}
		SU(3) \hookrightarrow SU(4) \longrightarrow S^{7}\,,
	\end{equation*}
	otteniamo la successione spettrale la cui pagina $E^{2}_{\bullet,\bullet}$ 
	è data da
	\missingfigure{Pagina}
	e quindi concludiamo che
	\begin{equation*}
		H_{p}\left( SU(4) \right) =
		\begin{cases}
			\Z\,, \quad &\text{se } p = 0,3,5,7,8,10,12\,;\\
			0\,, \quad &\text{altrimenti.}
		\end{cases}
	\end{equation*}
	
	Infine, vorremmo studiare anche l'omologia di $SU(5)$ 
	utilizzando la fibrazione
	\begin{equation*}
		SU(4) \hookrightarrow SU(5) \longrightarrow S^{9}\,.
	\end{equation*}
	A questo punto $E^{2}_{\bullet,\bullet}$ 
	è data da
	\missingfigure{pagina con differenziali non banali}
	e notiamo che potrebbero esserci due differenziali non banali.
	Purtroppo non conosciamo la struttura di $d^{9}$,
	quindi (per ora) non sappiamo come calcolare $H_{*}\left( SU(5) \right)$.
\end{ex}

Dato uno spazio topologico $X$, ricordiamo che possiamo costruire la fibrazione
\begin{equation*}
	\begin{tikzcd}
		\Omega X \ar[r] & PX \ar[r, "ev_{1}"] & X\,,
	\end{tikzcd}
\end{equation*}
dove lo spazio dei cammini $PX$ è contraibile. 
Se supponiamo che $X$ sia $n$-connesso,
preso il sistema locale $\Set{H_{q}(P_{x_{0}}^{x}X)}$,
per il  \hyperref[Serre-SS]{Teorema~\ref{Serre-SS}} si ha
\begin{equation*}
	E^{2}_{p,q} = H_{q}(X;\Set{H_{q}(P_{x_{0}}^{x}X)})\,,
\end{equation*}
e \todo{Finire che non sto capendo}






