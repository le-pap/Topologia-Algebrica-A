%%% Lezione 6 - inizio...

\lecture[Algebra delle operazioni coomologiche stabili. Definizione di algebra di Steenrod. Introduzione alle successioni spettrali: definizioni, filtrazioni, convergenza, modulo graduato associato ad una filtrazione, filtrazioni limitate, limite di una successione spettrale.]{2023-03-14}


Se prendiamo un'operazione $\phi \in \cat{Stab}\Theta(r,\pi,G)$, 
la pensiamo come una famiglia di trasformazioni naturali
$\phi_{n} \in \Theta(n,n+r,\pi,G) = H^{n+r}(K(\pi,n);G)$ che soddisfa la condizione di
stabilità $\Sigma \phi = \phi \Sigma$, ovvero gli omomorfismi
\begin{equation*}
	\begin{tikzcd}
		f_{n} \, : \, H^{n+r}(K(\pi,n);G) \ar[r, "i_{n+r}^{*}"]
		& H^{n+r}(\Sigma K(\pi,n-1);G) \ar[r, "\Sigma^{-1}"]
		& H^{n+r-1}(K(\pi,n-1);G)\,,
	\end{tikzcd}
\end{equation*}
soddisfano la condizione di stabilità $f_{n}(\phi_{n}) = \phi_{n-1}$,
dove il primo omomorfismo è indotto da $\cat{1}_{\pi}$
attraverso la seguente identificazione:
\begin{equation*}
	\begin{tikzcd}
		\Hom_{\cat{Grp}}(\pi,\pi) \ar[r, equals] \ar[d, equals]
		& H^{n-1}(K(\pi,n-1);\pi) \ar[d, "\Sigma", "\simeq"'] \\
		H^{n}(K(\pi,n);\pi) \ar[r,"i^{*}_{n}"]
		& H^{n}(\Sigma K(\pi,n-1);\pi)\,;
	\end{tikzcd}
\end{equation*}
ricordiamo che tramite l'identificazione
$H^{n-1}(K(\pi,n-1);\pi) \simeq [K(\pi,n-1),K(\pi,n)]_{\ast}$,
questa classe in coomologia induce un'unica mappa
$i_{n}:\Sigma K(\pi,n-1) \to K(\pi,n)$ a meno di omomotopia.

\begin{oss}
	La mappa $i_{n}$ è l'aggiunta di $K(\pi,n-1) \to \Omega K(\pi,n)$,
	che ricordiamo essere un'equivalenza omotopica debole.
\end{oss}

Quindi le operazioni stabili si possono vedere come il limite inverso
\begin{equation*}
	\cat{Stab}\Theta(r,\pi,G) = \varprojlim_{n}\left( H^{n+r}(K(\pi,n);G),f_{n}\right)\,,
\end{equation*}
e nel caso in cui $G$ è un anello, $H^{*}(K(\pi,n);G))$ è un anello
che \emph{non} induce la moltiplicazione in $\cat{Stab}\Theta(r,\pi,G)$.
Tuttavia, la composizione di mappe rende  $\bigoplus_{r \ge 0}\cat{Stab}\Theta(r,\pi,G)$
un \textbf{anello graduato}, unitario e in generale \emph{non} commutativo.
La coomologia $H^{*}(X;G)$ è un modulo su tale anello e una
qualsiasi funzione continua $f:X \to Y$ induce un omomorfismo di moduli.
Inoltre, se $G=k$ è un campo, allora le operazioni coomologiche diventano
una $k$-algebra, nel caso in cui $G=\Z/p$, con $p$ primo,
viene chiamata \textbf{algebra di Steenrod} $\AA_{p}$.