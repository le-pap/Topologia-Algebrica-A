 %%% Lezione 18


%\lecture[Lez18.]{2023-05-09]

Prima di vedere la dimostazione di questo fatto,
deduciamo qualche conseguenza interessante:
otteniamo una nuova versione del Teorema di Hurewicz.

\begin{cor}
	Sia $\Cc$ un anello di Serre aciclico e $X$ uno spazio \emph{semplicemente connesso}.
	Per ogni $n \ge 2$, si ha
	\begin{equation*}
		\pi_{q}(X, \ast) \in \Cc \iff H_{q}(X) \in \Cc
	\end{equation*}
	per ogni $q < n$. In tal caso, la mappa di Hurewicz
	\begin{equation*}
		h_{n} : \pi_{n}(X) \longrightarrow H_{n}(X)
	\end{equation*}
	è un isomorfismo $\mathrm{mod} \Cc$.
\end{cor}

Ad esempio, se $\pi_{1}(X,\ast)=0$, allora vale
\begin{enumerate}
	\item ...
	\item ...
	\item per ogni $q < n$ vale
	\begin{equation*}
		\overline{H}_{q}(X;\Q) = \cat{0} \iff \pi_{q}(X;\ast) \otimes \Q = \cat{0}
	\end{equation*}
	e la mappa $\pi_{n}(X;\ast) \otimes \Q \to H_{n}(X;\Q)$ è un isomorfismo.
\end{enumerate}

\begin{ex}
	L'ipotesi che $X$ sia semplicemetne connesso è necessaria.
	Se consideriamo $X = S^{1} \vee S^{2}$, allora $H_{q}(X)$ è un
	gruppo finitamente generato per ogni $q \in \N$, ma invece
	$\pi_{2}(X;\ast)$ non è finitamente generato:
	basti pensare che il rivestimento universale $\widetilde{X}$
	è una retta con infinite sfere attaccate nei punti interi.
\end{ex}

\begin{prop}\label{magia}
	Data una fibrazione di Serre $\pi : E \to B$, 
	con $B,F$ connessi per archi e base $\pi_{1}(B,b_{0})=\cat{0}$.
	Allora $\pi_{*}:H_{i}(E,F) \to H_{i}(B,b_{0})$ è un isomorfismo
	$\mathrm{mod}\Cc$ per ogni $i \ge n$.
\end{prop}

\begin{proof}[Dimostrazione del Teorema~\ref{serre-mod-c}]
	Consideriamo la fibrazione dei loop
	\begin{equation*}
		\Omega X \longrightarrow PX \longrightarrow X
	\end{equation*}
	e mostriamo che se $\pi_{q}(X,\ast) \in \Cc$ per ogni $q < n$, 
	allora $\pi_{n}(X,\ast) \to H_{n}(X)$ è un isomorfismo $\mathrm{mod}\Cc$.
	Ragioniamo per induzione su $n$.
	Se $n=2$, allora $\pi_{2}(X;\ast) \simeq H_{2}(X)$
	per il classico \textbf{Teorema di Hurewicz}.
	Supponiamo quindi $n>2$ e consideriamo il diagramma commutativo
	\begin{equation*}
		\begin{tikzcd}
			\pi_{q}(X,\ast) \ar[d, "h"]
			& \pi_{q}(PX,\Omega X) \ar[l, "\sim"'] \ar[r, "\sim"] \ar[d]
			& \pi_{q-1}(\Omega X, \ast) \ar[d, "\phi"] \\
			\overline{H}_{q}(X) 
			& H_{q}(PX, \Omega X) \ar[l, "\psi"] \ar[r, "\sim"]
			& H_{q-1}(\Omega X)\,.
		\end{tikzcd}
	\end{equation*}
	Se mostriamo che $\phi$ e $\psi$ sono isomorfismi $\mathrm{mod}\Cc$,
	allora concludiamo che anche $h$ lo è, da cui segue la tesi.
	
	Se $\pi_{2}(X,\ast)=0$, allora si ha $\pi_{1}(\Omega X, \ast) = 0$
	e possiamo applicare il Teorema di Hurewicz per ipotesi induttiva,
	da cui deduciamo che $phi$ e $\psi$ sono iso.
	Per $\psi$ usiamo la \hyperref[magia]{Proposizione~\ref{magia}}.
	Il problema è che in generale \textbf{non} vale $\pi_{2}(X,\ast)=0$,
	quindi non abbiamo a disposizione l'ipotesi $\pi_{1}(\Omega X, \ast)=0$.
	Per ovviare a questo problema, ricorriamo alle \textbf{Torri di Whitehead}:
	consideriamo la fibrazione
	\begin{equation*}
		\begin{tikzcd}
			K:=K \left(\pi_{2}(X,\ast),1\right) \ar[r]
			& Y \ar[r, "t"] & X\,,
		\end{tikzcd}
	\end{equation*}
	dove $t$ induce isomorfismi tra i gruppi di omotopia $\pi_{q}$, per $q > 2$.
	Questo ``uccide'' il $\pi_{2}$ di $X$, ma in compenso
	questo gruppo appare come gruppo fondamentale della fibra $K$.
	Dato che $\pi_{2}(X,\ast) \in \Cc$ e per ipotesi $\Cc$ è un anello aciclico,
	allora $H_{i}(K) \in \Cc$ per ogni $i>0$.
	Quindi $\overline{H}_{i}(Y) \to H_{i}(Y,K)$ è isomorfismo $\mathrm{mod}\Cc$.
	Usando la \hyperref[magia]{Proposizione~\ref{magia}}, deduciamo che per ogni $i \le n$
	anche 
	\begin{equation*}
		H_{i}(Y,y_{0}) \longrightarrow H_{i}(X,\ast)
	\end{equation*}
	è isomorfismo $\mathrm{mod}\Cc$.
	Siccome $\pi_{i}(Y,y_{0}) \simeq \pi_{i}(X,\ast)$ per $i \ge 3$,
	applichiamo l'ipotesi induttiva.
\end{proof}

\begin{cor}
	Sia $X$ è semplicemente connesso, $p$ primo e $n \ge 2$.
	Vale 
	\begin{equation*}
		\forall_{i<n} \, \pi_{i}(X, \ast) \otimes \Z_{(p)} = \cat{0}
		\iff \forall_{i<n} \, \overline{H}_{i}(X;\Z_{(p)})= \cat{0}
	\end{equation*}
	e si ha un isomorfismo 
	$\pi_{n}(X, \ast) \otimes \Z_{(p)} \simeq \overline{H}_{i}(X;\Z_{(p)})$.
	\begin{proof}
		Usiamo il Teorema di Hurewicz modulo la classe di Serre $\Cc_{p}$,
		ovvero i gruppi abeliani $A$ tali che $A \otimes \Z_{(p)}=\cat{0}$.
	\end{proof}
\end{cor}

\begin{thm}[Hurewicz $\mathrm{mod}\Cc$]\label{hurewicz-mod-c}
	Sia $\Cc$ un ideale di Serre aciclico e $(X,A)$ una coppia
	di spazi semplicemente connessi.
	Per $n \ge 1$ si ha che
	\begin{equation*}
		\forall_{2 \le i < n} \, \pi_{i}(X,A) \in \Cc \iff
		\forall_{2 \le i < n} \, H_{i}(X,A) \in \Cc
	\end{equation*}
	e in tal caso la mappa
	$\pi_{n}(X,A) \to H_{n}(X,A)$ è un isomorfismo $\mathrm{mod}\Cc$.
	\begin{proof}[Idea di dimostrazione]
		Boh sono rimasto indietro.\todo{recupera!}
		\begin{equation*}
			E^{2}_{s,t} = H_{s}\left(X,A;H_{t}(\Omega X) \right)
		\end{equation*}
		Non abbiamo ipotesi sull'omologia di $\Omega X$, ma la
		condizione che $\Cc$ sia un ideale ci serve per poter fare a meno
		di ipotesi sull'omologia della fibra,
		in quanto sia $\otimes$, sia $\ast$, sono operazioni
		che fanno rimanere in $\Cc$.
		Dal fatto che $H_{s}(X,A) \in \Cc$, per Künneth e il fatto che $\Cc$
		è ideale segue che $H_{s}\left(X,A;H_{t}(\Omega X) \right) \in \Cc$,
		quindi $p_{*}$ è un isomorfismo $\mathrm{mod}\Cc$.
	\end{proof}
\end{thm}

\begin{thm}[Whitehead $\mathrm{mod}\Cc$]
	Sia $\Cc$ un ideale di Serre aciclico e $f:X \to Y$
	una mappa continua. Per $n \ge 2$, i seguenti fatti sono equivalenti:
	\begin{rmnumerate}
		\item per ogni $i \le n-1$, la mappa $f_{\#}:\pi_{i}(X, \ast) \to \pi_{i}(Y,y_{0})$
		è un isomorfismo $\mathrm{mod}\Cc$, ed è epi per $i=n$;
		
		\item per ogni $i \le n-1$, la mappa $f_{*}:H_{i}(X) \to H_{i}(Y)$
		è un isomorfismo $\mathrm{mod}\Cc$, ed è epi per $i=n$.
	\end{rmnumerate}
	\begin{proof}
		A meno di sostituire $Y$ con il mapping cylinder di $f$,
		possiamo supporre che $f : X \subset Y$ sia un'inclusione.
		Consideriamo il diagramma commutativo
		\begin{equation*}
			\begin{tikzcd}
				\pi_{n+1}(Y,X)
			\end{tikzcd}
		\end{equation*}
		
		La condizione i)  è equivalente alla condizione
		\begin{equation*}
			\pi_{i}(Y,X) \in \Cc\,, \quad \text{per ogni } i \le n\,,
		\end{equation*}
		ma allora per il \hyperref[hurewicz-mod-c]{Teorema di Hurewicz $\mathrm{mod}\Cc$~\ref{hurewicz-mod-c}} notiamo che equivale a
		\begin{equation*}
			H_{i}(Y,X) \in \Cc\,, \quad \text{per ogni } i \le n\,,
		\end{equation*}
		ovvero la condizione ii).
	\end{proof}
\end{thm}

\begin{lemma}
	Siano $X,Y$ spazi tali che la loro omologia a coefficienti in $\Z_{(p)}$
	sia finitamente generata in ogni grado.
	Se $f : X \to Y$ induce un isomorfismo in $H_{q}(-;\Z_{(p)})$ per ogni $q$,
	allora $f$ induce isomorfismi $\mathrm{mod}\Cc_{p}$ 
	in ogni grado dell'omologia singolare
	a coefficienti in $\Z$.
	\begin{proof}
		Notiamo che un omomorfismo di gruppi $\alpha : A \to B$
		è un isomorfismo $\mathrm{mod}\Cc_{p}$ se e solo se
		\begin{equation*}
			\alpha \otimes \cat{1} : A \otimes \Z_{(p)} \longrightarrow B \otimes \Z_{(p)}
		\end{equation*}
		è un isomorfismo di gruppi.
		Dato che $\Z_{(p)}$ è piatto, allora tensorizzando
		la successione esatta
		\begin{equation*}
			\begin{tikzcd}
				\cat{0} \ar[r]
				& \ker \alpha \ar[r]
				& A \ar[r, "\alpha"]
				& B \ar[r]
				& \mathrm{coker}\alpha \ar[r]
				& \cat{0}\,,
			\end{tikzcd}
		\end{equation*}
		otteniamo una nuova successione esatta di gruppi abeliani.
		
		Uno $\Z_{(p)}$-modulo finitamente generato è banale
		se e solo se è banale modulo $p$, quindi
		se mostriamo che il nucleo e il conucleo della mappa
		\begin{equation*}
			f_{*}:H_{*}(X) \longrightarrow H_{*}(Y)
		\end{equation*}
		sono banali se tensorizziamo con $- \otimes \Z/p\Z$, allora
		abbiamo la tesi.
		
		Detto $Z$ il mapping cone di $f$, ricordiamo che vale
		\begin{equation*}
			\overline{H}_{*}(Z; \Z/p\Z) \simeq H_{*}(C_{f},X; \Z/p\Z)\,.
		\end{equation*}
		Siccome $\Z_{(p)}$ è un anello nöetheriano
		e per ogni $q$ i moduli
		$H_{q}(X;\Z_{(p)})$ e $H_{q}(Y;\Z_{(p)})$ sono finitamente generati,
		allora anche $H_{q}(Z;\Z_{(p)})$ è finitamente generato.
		Dal Teorema dei Coefficienti Universali
		ricordiamo che
		\begin{equation*}
			\overline{H}_{*}(Z;\Z_{(p)}) \otimes \Z/p\Z \hookrightarrow
			\overline{H}_{*}(Z; \Z/p\Z)\,,
		\end{equation*}
		quindi se $H_{*}(Z; \Z/p\Z) = 0$,
		segue che $\overline{H}_{*}(Z) \otimes \Z_{(p)} = \overline{H}_{*}(Z;\Z_{(p)})=0$.
		Concludiamo così che la mappa
		\begin{equation*}
			f_{*} \otimes \cat{1} : \overline{H}_{*}(X) \otimes \Z_{(p)}
			\longrightarrow \overline{H}_{*}(Y) \otimes \Z_{(p)}
		\end{equation*}
		è un isomorfismo.
	\end{proof}
\end{lemma}

\begin{cor}
	Se $X$ e $Y$ sono spazi semplicaemente connessi,
	con omologia $H_{q}(-;\Z_{(p)})$ finitamente generata per ogni $q \in \N$.
	Se $f:X \to Y$ è una mappa continua che induce un isomorfismo
	in $H_{*}(-;\Z/p\Z)$, allora
	\begin{equation*}
		f_{*}:\pi_{q}(X, \ast) \otimes \Z_{(p)}
		\longrightarrow \pi_{q}(Y, y_{0}) \otimes \Z_{(p)}
	\end{equation*}
	è un isomorfismo.
	\begin{proof}
		Segue applicando il Lemma + Whitehead mod C.
	\end{proof}
\end{cor}

Prima di enunciare il \textbf{Teorema di Serre},
facciamo alcuni calcoli dell'omologia di spazi classificanti.

\begin{ex}
	Il \textbf{Teorema di Whitehead} mod $\Cc$ applicato a un gruppo $A \in \Ab_{tor}$
	ci dice che
	\begin{equation*}
		\overline{H}_{*}\left(K(A,n) ; \Q \right) = 0.
	\end{equation*}
	Questo è chiaro per $n=1$. Se $n>1$, allora consideriamo la fibrazione
	\begin{equation*}
		\begin{tikzcd}
			K(A,n-1) \ar[r]
			& PK(A,n) \ar[r]
			& K(A,n)\,.
		\end{tikzcd}
	\end{equation*}
	Calcoliamo gli anelli di coomologia, per esempio, nel caso di $A=\Z$.
	Per $n=1$, sappiamo che $K(\Z,1)=\S^{1}$ ha coomologia
	\begin{equation*}
		H^{*}\left( K(\Z,1); \Q \right) \simeq \Lambda_{\Q}[\iota_{1}]\,,
	\end{equation*}
	con $\iota_{1}$ in grado $1$.
	Se $n=2$, allora $K(\Z,2) = \C\PP^{\infty}$ e il suo anello di coomologia è
	\begin{equation*}
		H^{*}\left( K(\Z,2) ; \Q \right) \simeq \Q[\iota_{2}]\,,
		\quad \lvert \iota_{2} \rvert = 2\,.
	\end{equation*}
	
	Tutto il calcolo con SS \todo{Copiare e capire}
	
	Ripetendo questo procedimento per ogni $n$, si dimostra che
	\begin{equation*}
		H^{*}\left( K(\Z,n) ; \Q \right) \simeq
		\begin{cases}
			\Lambda_{\Q}[\iota_{n}]\,, \quad &\text{se } n \text{ è dispari}\,; \\
			\Q[\iota_{n}]\,, \quad &\text{se } n \text{ è spari}\,,
		\end{cases}
	\end{equation*}
	dove $\iota_{n}$ è un generatore in grado $|\iota_{n}|=n$.
\end{ex}

\begin{thm}[Serre]\label{serre-thm}
	I gruppi di omotopia della sfera $S^{n}$ sono \textbf{finiti},
	eccetto $\pi_{0}(S^{n}, \ast) \simeq \Z \simeq \pi_{n}(S^{n},\ast)$.
	Se $n$ è pari, allora $\pi_{2n-1}(S^{n},\ast)$ è finitamente generato di rango $1$.
	\begin{proof}
		Conosciamo già l'omotopia di $S^{1}$.
		Se $n \ge 2$, allora consideriamo la mappa
		\begin{equation*}
			f : S^{n} \longrightarrow K(\Z,n)\,.
		\end{equation*}
		Se $n$ è dispari, allora $f_{*}$ induce un isomorfismo
		in omologia razionale, per cui applicando il
		Teorema di Whitehead mod C segue che $f$ induce
		un isomorfismo $\pi_{q}(-) \otimes \cat{1}_{\Q}$.
		
		Se $n$ è pari, allora indichiamo la fibra omotopica
		\begin{equation*}
			F \hookrightarrow S^{n} \longrightarrow K(\Z,n)\,.
		\end{equation*}
		Sappiamo che  l'anello di coomologia di $K(\Z,n)$
		è un'algebra di polinomi generata in grado $n$ da una classe $\iota_{n}$.
		Dato che $F$ ha omotopia banale in grado $n-1$ e $n$, allora 
		sulla colonna $0$ della SS, dopo l'unità $1$ in grado $0$,
		troviamo una classe $\iota_{2n-1}$ che deve uccidere $\iota_{n}^{2}$,
		cioè
		\begin{equation*}
			d_{2n-1}(\iota_{2n-1}) = \iota^{2}_{n}\,.
		\end{equation*}
		Da questo deduciamo che $d_{2n-1}(\iota_{2n-1} \iota_{n}^{k}) = \iota^{(k-1)2}_{n}$.
		Questo ci dice che
		\begin{equation*}
			H^{*}(F;\Q) = H^{*}\left( K(\Z,2n-1);\Q \right)
			\quad \implies \quad 
			H^{*}(F) \otimes \Q = H^{*}\left( K(\Z,2n-1)\right) \otimes \Q\,,
		\end{equation*}
		da cui concludiamo
		\begin{equation*}
			\pi_{q}(S^{n},\ast) \otimes \Q \simeq
			\begin{cases}
				\Q\,, \quad &\text{se } q=n,2n-1\,; \\
				0\,, \quad &\text{altrimenti}.
			\end{cases}
		\end{equation*}
	\end{proof}
\end{thm}

\begin{oss}
	La dimostrazione del \hyperref[serre-thm]{Teorema di Serre~\ref{serre-thm}} ci fornisce
	una tecnica interessante per il calcolo dell'omotopia,
	che possiamo sfruttare anche per lo studio dei gruppi
	finiti delle sfere. Consideriamo ad esempio il caso $n=3$
	e proviamo a calcolare $\pi_{4}(S^{3},\ast)$.
	Dall'approssimazione di Whitehead
	\begin{equation*}
		F := \tau_{\ge 4}S^{3} \longrightarrow
		S^{3} \longrightarrow K(\Z,3)\,,
	\end{equation*}
	tramite la successione di Barret-Puppe otteniamo
	la fibrazione
	\begin{equation*}
		K(\Z,2) \hookrightarrow F \longrightarrow S^{3}\,.
	\end{equation*}
	Guardando la successione spettrale\todo{LA SS} in $\Z$ in coomologia,
	otteniamo
	\begin{equation*}
		E_{2}^{\bullet,\bullet} \simeq \Lambda[\sigma] \otimes \Z[\iota_{2}]\,,
		\quad \lvert \sigma \rvert = 3\,,\, \lvert \iota_{2} \rvert = 2\,.
	\end{equation*}
	Per ogni $k$ si ha dunque
	\begin{equation*}
		H^{2k+1}(\tau_{\ge 4} S^{3}) \simeq \Z/k\Z
		\quad \implies \quad
		H_{2k}(\tau_{\ge 4} S^{3}) \simeq \Z/k\Z\,.
	\end{equation*}
	La prima $p$-torsione si ritrova in
	\begin{equation*}
		H_{2k}(\tau_{\ge 4} S^{3}) = \Z/p \Z\,,
	\end{equation*}
	quindi l'omotopia $\pi_{i}(S^{3},\ast)=0 \, \mathrm{mod}\Cc_{p}$ per $i < 2p$.
	Siccome $\pi_{2p}(\pi_{i}(S^{3},\ast) \otimes \Z_{(p)} \simeq \Z/p\Z$,
	per $p=2$ concludiamo
	\begin{equation*}
		\pi_{4}(S^{3},\ast) \simeq \Z/2\Z \simeq \pi_{4}(S^{2},\ast)\,,
	\end{equation*}
	dove abbiamo sfruttato la fibrazione di Hopf $S^{1} \subset S^{3} \to S^{2}$.
\end{oss}