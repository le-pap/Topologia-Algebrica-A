\documentclass[a4paper, 10pt, oneside, DIV=9, chapterprefix=true, numbers=enddot,bibliography=totoc]{scrbook}

\RedeclareSectionCommand[tocdynnumwidth]{chapter}
\RedeclareSectionCommands[tocdynindent]{section,subsection}
\usepackage{styleITA}
\usepackage{shortcutsITA}
\usepackage[normalem]{ulem}
\usepackage[outline]{contour}
\contourlength{2.25pt}

\newcommand{\embrace}[1]{\textup{(}#1\textup{)}}
\newlength{\LETTERheight}
\AtBeginDocument{\settoheight{\LETTERheight}{I}}
\newcommand*{\longrightsquigarrow}[1]{\ \raisebox{0.24\LETTERheight}{\tikz \draw [-to,
		line join=round, line cap=round,
		decorate, decoration={
			zigzag,
			segment length=4,
			amplitude=.9,
			post=lineto,
			post length=0.42ex
		}] (0,0) -- (#1,0);}\ }
	
\newlength{\HeightOfTextstyleOne}
\settoheight{\HeightOfTextstyleOne}{$\mathbf{1}$}
\newlength{\HeightOfScriptstyleOne}
\settoheight{\HeightOfScriptstyleOne}{$\scriptstyle\mathbf{1}$}
\newlength{\HeightOfScriptscriptstyleOne}
\settoheight{\HeightOfScriptscriptstyleOne}{$\scriptscriptstyle\mathbf{1}$}
\newcommand{\FancyOne}[1]{{\tikz[line cap=round,line join=round,line width=0.35*#1/\HeightOfTextstyleOne,scale=#1/\HeightOfTextstyleOne]{\draw (-0.0225,0.205) to (-0.0225,0.02) to[out=270,in=0] (-0.0425,0) to (-0.071,0) to (0.071,0) to (0.0425,0) to[out=180,in=270] (0.0225,0.02) to (0.0225,0.235) to (0.0175,0.235) to[out=210,in=0] (-0.075,0.201);}}}
\newcommand{\IOne}{\mathchoice%
	{\FancyOne{\HeightOfTextstyleOne}}%
	{\FancyOne{\HeightOfTextstyleOne}}%
	{\FancyOne{\HeightOfScriptstyleOne}}%
	{\FancyOne{\HeightOfScriptscriptstyleOne}}%
}
\newcommand{\IDigamma}{\tikz[line cap=round,line join=round,line width=0.35]{\draw (0.06,0.2286) to[out=0,in=90] (0.1353,0.172) to (0.1353,0.2286) to (-0.0525,0.2286) to (-0.0415,0.2286) to[out=0,in=90] (-0.0215,0.2086) to (-0.0215,0.02) to[out=270,in=0] (-0.0415,0) to (-0.0525,0) to (0.0605,0) to (0.0415,0) to[out=180,in=270] (0.0215,0.02) to (0.0215,0.2086) to[out=90,in=180] (0.0415,0.2286);\draw (0.025,0.1335) to[out=0,in=90] (0.0968,0.0769) to (0.0968,0.1335) to cycle;}\hspace{0.1ex}\vphantom{\IF}}


\DeclareFontFamily{U}{min}{}
\DeclareFontShape{U}{min}{m}{n}{<-> udmj30}{}

	
\makeatletter
\renewcommand{\@pnumwidth}{2em} 
\renewcommand{\@tocrmarg}{3em}
\makeatother
%\RedeclareSectionCommand[tocindent+=0.5em]{section}
%\RedeclareSectionCommand[tocindent+=0.5em]{subsection}


\subject{Appunti del corso di}
\title{Topologia Algebrica A}
\author{{\normalsize Docente}\\
	Filippo Gianluca Callegaro}
\date{{\normalsize Appunti scritti da}\\
	Filippo Papallo}
\publishers{Semestre estivo 2022/23\\
	Università di Pisa}

\usepackage{bookmark}
\begin{document}

\setlength{\parindent}{0pt}
\setlength{\parskip}{4pt}

\frontmatter
\KOMAoption{chapterprefix}{false}
\renewcommand{\thedummy}{\arabic{dummy}}
\maketitle
Questo documento conterrà (si spera) qualcosa.

\hrulefill

Ultimo aggiornamento: \today
	
%Some additions have been made by the author. To distinguish them from the lecture's actual contents, they are labelled with an asterisk. So any \emph{Proof}* or \emph{Lemma}* etc.\ that the reader might encounter are wholly the author's responsibility.
%\\[\thmsep]Please report errors, typos etc.\ through the \href{https://github.com/}{\emph{Issues}} feature of GitHub, or just tell me before or after the lecture.
	
	
\tableofcontents
\listoftoc{lol}
\setcounter{llecture}{0}
\mainmatter\KOMAoption{chapterprefix}{true}
\renewcommand{\thedummy}{\thechapter.\arabic{dummy}}
%\renewcommand{\thechapter}{\arabic{chapter}}

\renewcommand{\thechapter}{\Roman{chapter}}

	%%% Lezione 1

\chapter{Fibrazioni e cofibrazioni}

\section{Cofibrazioni}

\lecture[Cofibrazioni, proprietà ed esempi. Costruzione del mapping cone di una funzione continua, successione di cofibrazione, successioni coesatte e derivazione della successione di Barret-Puppe.]{2023-02-27}

Dati due spazi topologici $X,Y$, 
indichiamo con $Y^{X} := \Hom_{\cat{Top}}(X,Y)$
l'insieme delle funzioni continue da $X$ a $Y$.
Consideriamo un sottospazio $A \subset X$
e indichiamo con $i:A \to X$ la mappa di inclusione.
Notiamo che, per ogni funzione $f:X \to Y$,
possiamo definire la \emph{restrizione} di $f$ ad $A$
come la composizione $f\vert_{A} := f \circ i$;
questo induce una funzione $Y^{X} \to Y^{A}$.

\begin{df}
	La mappa $Y^{X} \to Y^{A}$ data dalla restrizione è una
	\textbf{fibrazione} se, dato un quadrato commutativo
	\begin{equation}\label{fibI}
		\begin{tikzcd}
			W \ar[r] \ar[d, hook, "\iota_{0}"'] & Y^{X} \ar[d] \\
			I \times W \ar[r] \ar[ur, dashed] & Y^{A}\,,
		\end{tikzcd}
	\end{equation}
	esiste un sollevamento $I \times W \to Y^{X}$ che rende
	il diagramma commutativo, dove $\iota_{t}$ è
	l'inclusione al ``tempo $t$'' data da $\iota_{t}(w) := (t,w)$.
\end{df}

Per tutto il corso assumeremo che gli spazi topologici coinvolti siano
localmente compatti e di Hausdorff:
infatti, sotto queste ipotesi vale la \emph{legge esponenziale}
\begin{equation*}
	\Hom(X,Z^{Y}) \simeq \Hom(X \times Y, Z).
\end{equation*}
Dunque, se $X$ e $Y$ sono spazi localmente compatti e di Hausdorff,
il diagramma \eqref{fibI} è equivalente a

	\begin{equation*} %\label{fibII}
		\begin{tikzcd}
			W \times A  \ar[r, "\cat{1}_{W} \times i"]
			 \ar[d, hook, "\iota_{0} \times \cat{1}_{A}"'] 
			 & W \times X \ar[d]  \ar[ddr, bend left] & \\
			I \times W \times A \ar[r]  \ar[drr, bend right=20pt]
			& I \times W \times X \ar[dr, dashed] & \\
			& & Y \,,
		\end{tikzcd}
	\end{equation*}
	
e riapplicando la legge esponenziale,
portando $W$ ``all'esponente'',
si può vedere anche come
	
	\begin{equation}\label{fibIII}
		\begin{tikzcd}
			A  \ar[r, "i"]
			 \ar[d, hook, "\iota_{0}"'] 
			 & X \ar[d]  \ar[ddr, bend left] & \\
			I \times A \ar[r]  \ar[drr, bend right=20pt]
			& I \times X \ar[dr, dashed] & \\
			& & Y^{W} \,.
		\end{tikzcd}
	\end{equation}
	
Generalizziamo la proprietà descritta nel diagramma \eqref{fibIII}
sostituendo un qualsiasi spazio topologico $Z$ a $Y^{W}$ per
ottenere la seguente

\begin{df}
	Una \textbf{cofibrazione} $i:A \to X$ è una mappa continua che
	per ogni $Z$ soddisfa la \textbf{homotopy extension property} 
	(\textbf{HEP}), cioè per ogni diagramma commutativo del tipo
	\begin{equation}\label{HEP}
		\begin{tikzcd}
			A  \ar[r, "i"]
			 \ar[d, hook, "\iota_{0}"'] 
			 & X \ar[d]  \ar[ddr, bend left] & \\
			I \times A \ar[r]  \ar[drr, bend right=20pt]
			& I \times X \ar[dr, dashed] & \\
			& & Z \,,
		\end{tikzcd}
		\tag{\textbf{HEP}}
	\end{equation}
	esiste una mappa continua $I \times X \to Z$ che completa il diagramma.
\end{df}

Data una funzione continua $f:X \to Y$, 
ricordiamo che il \textbf{mapping cylinder} è lo spazio topologico
\begin{equation*}
	M_{f} := \left( I \times X \sqcup Y \right) / \sim \,,
\end{equation*}
dove $\sim$ è la relazione d'equivalenza che identifica $(0,x) \sim f(x)$,
per ogni $x \in X$, ed è banale altrove. Ricordiamo la seguente
caratterizzazione delle cofibrazioni:

\begin{thm}
	I seguenti fatti sono equivalenti:
	\begin{rmnumerate}
		\item la mappa $i:A \to X$ è una cofibrazione;
		\item la mappa $i$ ha la \eqref{HEP} per il mapping cylinder $M_{i}$;
		\item la funzione $s:M_{i} \to I \times X$ ammette una retrazione,
		i.e. esiste $r:I \times X \to M_{i}$ 
		tale che $rs=\cat{1}_{M_{i}}$.
	\end{rmnumerate}
\end{thm}

\begin{prop}
	Una cofibrazione è un \textbf{embedding}.
\end{prop}

\begin{oss}
	In generale, se $i:A \to X$ è un embedding,
	allora $s:M_{i} \to I \times X$ è continua e iniettiva,
	ma può \emph{non} essere un embedding.
	Se $i:A \hookrightarrow X$ è un'inclusione di sottospazio,
	allora $s$ mappa bigettivamente
	\begin{equation*}
		M_{i} \longrightarrow (I \times A) \cup (\{0\} \times X)\,.
	\end{equation*}
	Ora, se $A \subset X$ è \textbf{chiuso}, allora $s$ è un omeomorfismo
	con l'immagine; altrimenti, se $A$ non chiuso oppure $i$ non è una
	cofibrazione, la topologia di $M_{i}$ può non coincidere con quella
	indotta su $N:=(I \times A) \cup (\{0\} \times X)$.
\end{oss}

\begin{ex}
	Data l'inclusione di $A=(0,1]$ 
	nell'intervallo chiuso $X=[0,1]$,
	consideriamo la successione 
	$\Set{\left(\frac{1}{n+1}, \frac{1}{n+1} \right)|n \in \N}$
	sulla diagonale del quadrato $X^{2}$.
	Notiamo che in $N \subset X^{2}$ con la topologia prodotto
	la successione converge a $(0,0)$, mentre invece non converge
	in $M_{i}$ perché nella topologia quoziente
	esistono intorni aperti $U$ dell'origine che
	intersecano la diagonale unicamente in $(0,0)$, come
	ad esempio $U = \left( \{0\} \times X \right) \cup \Set{(x,y) | x<y }$.
	Questo mostra che la topologia di $M_{i}$ è \emph{più fine} 
	della topologia prodotto di $N$, quindi i due spazi non sono omeomorfi.
\end{ex}

Si può dimostrare che, se $X$ è di Hausdorff,
allora l'immagine di una cofibrazione $i:A \to X$ è chiusa.
Si può dedurre dunque che un'inclusione chiusa $A \subset X$
è una cofibrazione se e solo se la mappa 
$(I \times A) \cup (\{0\} \times X) \hookrightarrow I \times X$
ha una retrazione.

\begin{df}
	Dato $x \in X$, diciamo che $x$ è un \textbf{punto base non degenere}
	se $\{x\} \hookrightarrow X$ è una cofibrazione; in tal caso,
	si dice che $(X,x)$ è \textbf{ben puntato}.
\end{df}

\begin{ex}
	L'inclusione $S^{n-1} \subset D^{n}$ è una cofibrazione.
\end{ex}

\begin{exercise}
	La classe delle cofibrazioni è chiusa per le seguenti operazioni:
	\begin{rmnumerate}
		\item \textbf{cambio di cobase}: se $i: A \to X$ è una cofibrazione
		e $\alpha:A \to B$ una qualsiasi funzione continua, allora
		la mappa $i^{B}$ in basso al diagramma di pushout
		\begin{equation*}
			\begin{tikzcd}
			A \ar[r,"i"] \ar[d, "\alpha"'] & X \ar[d] \\
			B \ar[r, "i^{B}"] & X \cup_{A} B\,,  
			\end{tikzcd}
		\end{equation*}
		dove $X \cup_{A} B$ è ottenuto come quoziente di $X \sqcup B$
		per la relazione d'equivalenza $\alpha(a) \sim i(a)$, per ogni $a \in A$.
		
		\item \textbf{coprodotto}: data una famiglia di cofibrazioni 
		$\Set{A_{j} \to X_{j}}$, allora anche $\coprod A_{j} \to \coprod X_{j}$
		è una cofibrazione.
		
		\item \textbf{composizione}: se $i:A \to X$ e $j:X \to Y$ sono
		cofibrazioni, allora anche $j \circ i:A \to Y$ è a sua volta una cofibrazione.
		
	\end{rmnumerate}
\end{exercise}

\begin{ex}
	L'inclusione $Y \subset X$ di un sottocomplesso CW è una cofibrazione.
\end{ex}

\begin{prop}\label{contr-htp}
	Sia $i:A \to X$ una cofibrazione. Se $A$ è uno spazio contraibile, allora
	la mappa quoziente $p:X \to X/A$ è un'equivalenza omotopica.
	\begin{proof}
		Sia $h:I \times A \to A$ omotopia di contrazione, cioè
		tale che per ogni $a \in A$ valga $h(0,a) = a$ e $h(1,a) = x$,
		per qualche $x \in A$. Siccome $i$ è una cofibrazione,
		possiamo estendere $i \circ p$ a un'omotopia $H:I \times X \to X$,
		ottenendo così il diagramma commutativo
		\begin{equation*}
		\begin{tikzcd}
			A  \ar[r, "i"]
			 \ar[d, hook, "\iota_{0}"'] 
			 & X \ar[d]  \ar[ddr, bend left, "\cat{1}_{X}"] & \\
			I \times A \ar[r]  \ar[drr, bend right=20pt, "i \circ h"']
			& I \times X \ar[dr, dashed, "H"] & \\
			& & X \,;
		\end{tikzcd}
		\end{equation*}
		dal diagramma deduciamo che $H(1,a) = x$ per ogni $a \in A$,
		quindi deduciamo che fattorizza attraverso la proiezione:
		\begin{equation*}
			\begin{tikzcd}
				X \ar[rr, "H_{1}"] \ar[dr, "p"'] & & X \\
				& X/A \ar[ur, dashed, "\pi"'] & \,.
			\end{tikzcd}
		\end{equation*}
		Notiamo che $\pi \circ p = H_{1}$ è omotopa all'identità di $X$
		dato che $H_{0} = \cat{1}_{X}$.
		
		Vogliamo ora costruire un'omotopia tra $p \circ \pi$ e $\cat{1}_{X/A}$
		per concludere che $p$ è un'equivalenza omotopica. 
		Siccome $H$ è un'omotopia che manda $I \times A \to A$,
		definisce una mappa $\overline{H}$ che fa commutare il quadrato
		\begin{equation*}
			\begin{tikzcd}
				I \times X \ar[r, "H"] \ar[d, "\cat{1}_{I} \times p"']
				& X \ar[d, "p"] \\
				I \times A \ar[r, "\overline{H}"] & A\,,
			\end{tikzcd}
		\end{equation*}
		dove notiamo che $\overline{H}$ è un'omotopia tra 
		$\overline{H}_{0} = \cat{1}_{X/A}$ e $\overline{H}_{1}=p \circ \pi$,
		dunque si ha la tesi.
	\end{proof}
\end{prop}


\section{Successione di cofibrazione}

\begin{df}
	Sia $\ast \in X$ un punto base. Un'\textbf{omotopia puntata} è una mappa
	\begin{equation*}
		X \wedge I_{\ast} := \left. X \times I \middle/ \ast \times I \right. 
		\longrightarrow X\,.
	\end{equation*}
	Data una mappa $f:X \to Y$, definiamo il suo \textbf{mapping cylinder ridotto}
	come lo spazio
	\begin{equation*}
		M(f) := \left(Y \times \{0\}  \right) \cup_{X \times \{0\}} 
		\left(X \wedge I_{\ast}\right)\,.
	\end{equation*}
	Una \textbf{cofibrazione puntata} è una mappa $i:A \to X$
	tale che $M(i)$ sia un retratto di $X \wedge I_{\ast}$.
\end{df}

\begin{fact}
	Ogni mappa continua $f:X \to Y$ fattorizza attraverso
	il proprio mapping cylinder ridotto
	\begin{equation*}
		\begin{tikzcd}
			X \ar[r, "\iota_{1}"] \ar[rr, bend right, "f"']
			& M(f) \ar[r]
			& Y\,.
		\end{tikzcd}
	\end{equation*}
	Si può inoltre dimostrare che l'immersione $\iota_{1} : X \to M(f)$
	è una cofibrazione e $M(f) \to Y$ è un'equivalenza omotopica;
	per questo motivo quando parliamo di cofibrazioni possiamo 
	considerare delle \emph{inclusioni}.
\end{fact}

\begin{ex}
	Dato uno spazio $X$, il \textbf{cono} di $X$ può essere visto come
	\begin{equation*}
		CX = M(X \to \ast)\,,
	\end{equation*}
	dunque abbiamo una fattorizzazione $X \to CX \to \ast$.
\end{ex}

\begin{df}
	Definiamo il \textbf{mapping cone} di una mappa $f:X \to Y$
	come il pushout $C(f)$ del seguente diagramma:
	\begin{equation*}
		\begin{tikzcd}
			& X \ar[r] \ar[d, "\iota_{1}"]  \ar[dl,"f"'] & \ast \ar[d] \\
			Y & M(f) \ar[l, "\sim"'] \ar[r] & C(f)\,.
		\end{tikzcd}
	\end{equation*}
\end{df}

\begin{ex}
	Possiamo vedere la sospensione di uno spazio $X$ 
	come il mapping cone di una mappa costante
	\begin{equation*}
		\Sigma X = C(X \to \ast)\,.
	\end{equation*}
\end{ex}

\begin{oss}
	Il pushout che definisce $C(f)$ può essere costruito anche in un altro modo:
	se consideriamo la cofibrazione $\iota_{1}:X \to CX$ al posto della mappa costante,
	possiamo definire il mapping cone di $f$ come il pushout del diagramma
	\begin{equation*}
		\begin{tikzcd}
			& \ast 			 \\
			X \ar[r, "\iota_{1}"'] \ar[d, "f"']  \ar[ur] & CX \ar[d] \ar[u] \\
			Y \ar[r] & C(f)\,.
		\end{tikzcd}
	\end{equation*}
	Le due costruzioni sono omeomorfe tramite la mappa che ``inverte il tempo'',
	cioè $t \mapsto 1-t$.
\end{oss}


	Se $f$ è una cofibrazione, allora anche $CX \to C(f)$ è a sua volta
	una cofibrazione (per cambio di cobase).
	Dato che il cono $CX$ è contraibile, 
	per \hyperref[contr-htp]{Proposizione~\ref*{contr-htp}}
	deduciamo che $C(f)$ è omotopicamente equivalente a
	\begin{equation*}
		C(f) / CX \simeq Y/X\,.
	\end{equation*}
	Abbiamo così provato il seguente

\begin{lemma}\label{cone-quotionet}
	Sia $f:X \to Y$ una cofibrazione. La mappa $C(f) \to Y/X$
	è un'equivalenza omotopica.
\end{lemma}

\begin{df}
	Una \textbf{successione di cofibrazione}
	è un diagramma omotopicamente equivalente a
	\begin{equation*}
		\begin{tikzcd}
			X \ar[r, "f"] & Y \ar[r, "i(f)"] & C(f)\,,
		\end{tikzcd}
	\end{equation*}
	per qualche $f$.
\end{df}

Si noti che la composizione $i(f) \circ f$ è
omotopicamente banale, con omotopia
\begin{equation*}
	h:X \times I \longrightarrow C(f)\,, \quad
	h(x,t) = [x,t]\,,
\end{equation*}
dove $h_{0} \sim f$ e $h_{1}$ è costante.

\begin{fact}
	La coppia di mappe $(i(f),h)$ soddisfa la seguente
	proprietà universale:
	data una qualsiasi funzione continua $g:Y \to Z$
	e un'omotopia $H$ tale che $H_{0} = g \circ f$ e $H_{1}$
	sia costante, esiste un'unica $C(f) \to Z$ che estende $g$
	e induce $H$.
	\begin{equation*}
		\begin{tikzcd}
			X \ar[r, "f"] \ar[drr, "g \circ f"']
			& Y \ar[rr, "i(f)"] \ar[dr, "g"] & & C(f) \ar[dl, dashed, "\exists !"] \\
			& & Z & \,.
		\end{tikzcd}
	\end{equation*}
\end{fact}

\begin{lemma}\label{cone-exactness}
	Data una mappa di spazi puntati $f:(X,x) \to (Y,y)$
	e un qualsiasi spazio puntato $(Z,z)$,
	la sequenza di insiemi puntati
		\begin{equation*}
			\begin{tikzcd}
				{[X,Z]} & {[Y,Z]} \ar[l, "f^*"'] & {[C(f),Z]} \ar[l,"i(f)^*"']
			\end{tikzcd}
		\end{equation*}
		è \textbf{esatta}, cioè $\textrm{im}\left( i(f)^*\right) 
		= \left( f^* \right)^{-1}([\ast])$.
\end{lemma}

\begin{df}
	Una successione di mappe puntate $X \to Y \to Z$ si dice \textbf{coesatta}
	se, per qualsiasi spazio puntato $(W,w)$, dopo aver applicato $[-,W]$
	si ottiene una sequenza esatta come nel 
	\hyperref[cone-exactness]{Lemma~\ref{cone-exactness}}.
\end{df}

\begin{ex}
	Il \hyperref[cone-exactness]{Lemma~\ref{cone-exactness}} mostra che
	la successione $X \to Y \to C(f)$ è coesatta.
\end{ex}


La costruzione del mapping cone $C(f)$ è funtoriale
ed è determinata solamente da $f$. 
Riapplicando la costruzione a $i(f) : Y \to C(f)$
possiamo continuare la sequenza di mappe
\begin{equation*}
	\begin{tikzcd}
		X \ar[r, "f"]
		& Y \ar[r, "i(f)"]
		& C(f) \ar[r, "i^2(f)"]
		& C(i(f)) \ar[r, "i^3(f)"]
		& \dots
	\end{tikzcd}
\end{equation*}

Abbiamo visto che $i(f)$ è il pushout dell'inclusione
$X \to CX$ lungo $f:X \to Y$; dato che il cono è contraibile,
per la Proposizione~\ref{contr-htp} si ha che la mappa
quoziente è un'equivalenza omotopica, dunque
per il Lemma~\ref{cone-quotionet} deduciamo che
\begin{equation*}
	CX \xrightarrow{\sim} C(i(f))/CY = Cf/Y = \Sigma X\,. 
\end{equation*}
In questo modo otteniamo la sequenza
\begin{equation*}
	\begin{tikzcd}
		X \ar[r, "f"]
		& Y \ar[r,"i(f)"] 
		& C(f) \ar[r,"i^2(f)"] \ar[dr] 
		& C(i(f)) \ar[r,"i^3(f)"] \ar[d, "\simeq"']
		& C(i^2(f)) \ar[r,"i^4(f)"] \ar[d, "\simeq"'] & \dots \\
		& & & \Sigma X \ar[r, "-\Sigma f"]
		& \Sigma Y \ar[r, "-\Sigma i(f)"]
		& \dots
	\end{tikzcd}
\end{equation*}
dove la sospensione di $f$ è definita come la mappa
\begin{equation*}
	-\Sigma f \left([t,x]\right) := [1-t,f(x)]\,.
\end{equation*}
Considerando dunque la riga inferiore, otteniamo la sequenza
\begin{equation}\label{BPcofib}
	X \xrightarrow{f} Y \to C(f) \to \Sigma X \to \Sigma Y 
	\to \Sigma C(f) \to \Sigma^{2} X \to \dots
\end{equation}
detta \textbf{successione di Barret-Puppe}.
La successione \eqref{BPcofib} è coesatta,
cioè che ogni coppia di mappe consecutive è coesatta
nel senso del Lemma~\ref{cone-exactness}.


\begin{oss}
	Data una coppia $(X,A)$, allora vale
	\begin{equation*}
		\widetilde{H}_{*}(X / CA) = H_{*}(X,A)\,,
	\end{equation*}
	infatti si ha
	\begin{equation*}
		H_{*}(X \cup CA) 
		= H_{*}(X \cup CA, \ast)
		= H_{*}(X \cup CA, CA)
		= H_{*}(X \cup CA\vert_{\le \frac{1}{2}}, CA\vert_{\le \frac{1}{2}})
		= H_{*}(X,A)\,,
	\end{equation*}
	mentre dal fatto che $i:A \to X$ è una cofibrazione
	si ha $X \cup CA \simeq X/A$ e dunque $H_{*}(X \cup CA)  
	= \widetilde{H}_{*}(X/CA)$.
\end{oss}



















	%%% Lecture 2

\section{Fibrazioni}

\lecture[Fibrazioni: proprietà generali. Proprietà del sollevamento dell'omotopia relativa per cofibrazioni. Successione di fibrato e definizione di fibra omotopica. Successione di fibrazione di Barratt-Puppe e successione esatta lunga di omotopia del fibrato e della coppia. Teorema di esistenza delle sezioni di Postnikov.]{2023-03-03}

Lo scopo di oggi è costruire una successione esatta per
le fibrazioni, analogamente a quanto fatto per la successione
di Barret-Puppe.

\begin{df}
	L'\textbf{$n$-esimo gruppo di omotopia} di $X$ è
	\begin{equation*}
		\pi_{n}(X,\ast) := [(I^{n},\de I^{n}),(X,\ast)]\,.
	\end{equation*}
	Definiamo la versione relativa come
	\begin{equation*}
		\pi_{n}(X,A,\ast) := [(I^{n},\de I^{n}, J^{n-1}),(X,A,\ast)]\,,
	\end{equation*}
	dove $J^{k} := (\de I^{k} \times I) \cup ( I^{k} \times \{0\} )$
	è $\de I^{n}$ a cui è stata levata la faccia $I^{n-1} \times \{1\}$.
\end{df}

Osseviamo che la restrizione alla faccia $I^{n-1} \times \{1\}$ induce
\begin{equation*}
	\pi_{n}(X,A,\ast) \to \pi_{n}(A,\ast)\,,
\end{equation*}
e ricordiamo che l'inclusione $\ast \subset A$ induce
\begin{equation*}
	\pi_{n+1}(X,\ast) \to \pi_{n+1}(X,A,\ast) \to \pi_{n+1}(X,A,\ast) \to \pi_{n}(A,\ast) \dots
\end{equation*}

Vogliamo dimostrare che questa è una successione esatta.
\begin{df}
	Una \textbf{fibrazione} è una mappa $p:E \to B$ che soddisfa
	la \textbf{homotopy lifting property} (\textbf{HLP}) per ogni spazio $W$,
	cioè ogni quadrato commutativo della forma
	\begin{equation}\label{HLP}
		\begin{tikzcd}
			W \ar[r] \ar[d, "\iota_{0}"', hook] & E \ar[d, "p"] \\
			I \times W \ar[r] \ar[ur, dashed] & B\,,
		\end{tikzcd}\tag{HLP}
	\end{equation}
	ammette un sollevamento $I \times W \to E$ che fa commutare il diagramma.
\end{df}

\begin{exercise}
	Le fibrazioni sono chiuse rispetto alle seguenti operazioni:
	\begin{rmnumerate}	
	\item \textbf{cambio di base}:
	se $p:E \to B$ è una fibrazione,
	e consideriamo il \emph{pullback}
	lungo una qualsiasi mappa $X \to B$ continua 
	\begin{equation*}
		\begin{tikzcd}
			f^*X := X \times_{B} E \ar[d] \ar[r] & E \ar[d, "p"] \\
			X \ar[r, "f"] & B\,,
		\end{tikzcd}
	\end{equation*}
	allora anche la mappa $X \times_{B} E \to X$ è una fibrazione;
	
	\item \textbf{prodotti arbitrari}: data una famiglia di fibrazioni $p_{i}:E_{i} \to B_{i}$,
	allora anche $\prod E_{i} \to \prod B_{i}$ è una fibrazione;
	
	\item \textbf{composizione}: se $p: E \to  B$ e $q:B \to B'$ sono fibrazioni,
	allora anche $q \circ p$ è una fibrazione; 
	
	\item \textbf{esponenziazione}:
	se $p:E \to B$ è fibrazione,
	allora per ogni spazio $A$ la mappa indotta per post-composizione
	$E^{A} \to B^{A}$ è ancora una fibrazione.
	\end{rmnumerate}
	\begin{proof}[Una soluzione]
		YE\todo{Fai esercizietto.}
	\end{proof}
\end{exercise}

D'ora in poi lavoreremo nella categoria
degli spazi puntati, con punto base \emph{non-degenere}.
\begin{lemma}[HLP relativa]\label{relative-HLP}
	Sia $A \subset X$ una cofibrazione chiusa
	e sia $p:E \to B$ una fibrazione. Allora esiste un sollevamento
	\begin{equation*}
		\begin{tikzcd}
			(X \times \{0\}) \cup (A \times I) \ar[d] \ar[r] & E \ar[d, "p"] \\
			X \times I \ar[r] \ar[ur,dashed] & B\,.
		\end{tikzcd}
	\end{equation*}
\end{lemma}

\begin{prop}
	Se $i:A \to B$ è una cofibrazione chiusa e puntata 
	di spazi localmente compatti,
	con punti base non degeneri, allora la mappa di restrizione di funzioni
	\begin{equation*}
		X_{*}^{B} \longrightarrow X_{*}^{A}\,,
		\quad f \longmapsto f \circ i\,,
	\end{equation*}
	è una fibrazione.
	\begin{proof}
		Dato uno spazio $V$, consideriamo il diagramma
		\begin{equation*}
			\begin{tikzcd}
				V \ar[r] \ar[d, "\iota_{0}"', hook] & X_{*}^{B} \ar[d] \\
				I \times V \ar[r] & X_{*}^{A}\,,
			\end{tikzcd}
		\end{equation*}
		e usando la legge esponenziale possiamo riscriverlo come
		\begin{equation*}
			\begin{tikzcd}
				V \times A \ar[r, "\cat{1} \times i"] \ar[d, "\iota_{0}"', hook] 
				& V \times B \ar[d, "\iota_{0}", hook] \ar[rdd, bend left] & \\
				I \times V \times A \ar[r] \ar[rrd, bend right=20]
				& I \times V \times B  \ar[dr, dashed] & \\
				& & X \,,
			\end{tikzcd}
		\end{equation*}
		e notiamo che la freccia tratteggiata esiste perché $i$ soddisfa la \eqref{HEP}.
	\end{proof}
\end{prop}

\begin{oss}
	L'ipotesi di locale compattezza ci è servita
	per dire che l'aggiunzione
	\begin{equation*}
		C(V,C(A,X)) \simeq C(V \times A,X)
	\end{equation*}
	è un omeomorfismo (vedi \parencite[2.4]{tomdieck}).
	Nel caso $(B,A,\ast) = (I,\de I, 0)$ abbiamo
	\begin{equation*}
		X^{I}_{\ast} = P(X) = \Set{\omega:I \to X | \omega(0) = \ast}\,,
	\end{equation*}
	quindi abbiamo una naturale mappa di valutazione al tempo $t=1$,
	la quale definisce una mappa continua $ev_{1} : P(X) \to X$ che inoltre
	è una fibrazione.
\end{oss}

Da Lemma~\ref{relative-HLP} otteniamo il seguente
\begin{cor}\label{htp-lifting}
	Sia $p:E \to B$ una fibrazione e siano
	$g:W \to E, f:W \to B$ mappe tali che $p \circ g \sim f$,
	cioè $g$ è un sollevamento di $f$ a meno di omotopia.
	\begin{equation*}
		\begin{tikzcd}[column sep=large]
			& E \ar[d, "p"] \\
			W \ar[r, "f"'] \ar[ru, "g", shift left=1ex] \ar[ru, "\overline{g}"', dashed, shift right=0.5ex, shorten=0.5ex]  & B\,.
		\end{tikzcd}
	\end{equation*}
	Allora $g$ è omotopa a un vero sollevamento $\overline{g}$ di $f$,
	cioè esiste $\overline{g}:W \to E$ tale che $p \circ \overline{g} = f$
	e $\overline{g} \sim g$.
\end{cor}

Dal \hyperref[htp-lifting]{Corollario~\ref{htp-lifting}} 
segue che $g:W \to E$ tale che $p \circ g \sim \ast$,
allora $g$ è omotopa a $\overline{g}: W \to p^{-1}(\ast)$
e quindi, posto $F:=p^{-1}(\ast)$, la \textbf{succesione di fibrato}
\begin{equation*}
	\begin{tikzcd}
		F \ar[r] & E \ar[r, "p"] & B
	\end{tikzcd}
\end{equation*}
è \textbf{esatta}, cioè per ogni spazio $W$ ben puntato,
otteniamo una successione esatta di insiemi puntati 
\begin{equation*}
	\begin{tikzcd}
		{[W,F]_{\ast}} \ar[r] & {[W,E]_{\ast}} \ar[r] & {[W,B]_{\ast}}\,.
	\end{tikzcd}
\end{equation*}


Se $f:X \to Y$ non è una fibrazione, 
possiamo fattorizzare
\begin{equation*}
	X \longrightarrow T(f) 
	:= \Set{(x,\omega) \in X \times Y^{I} | \omega(1) = f(x)}\,,
	\quad x \longmapsto (x, \omega_{f(x)})\,,
\end{equation*}
dove $\omega_{f(x)}$ è il cammino costante nel punto $f(x)$.
Si osserva che questa ``inclusione'' è in realtà 
un'equivalenza omotopica: per vederlo,
è sufficiente ``riavvolgere'' ogni cammino sulla seconda
componente fino al tempo $0$ in maniera continua.
Otteniamo così il diagramma
\begin{equation*}
	\begin{tikzcd}
		X \ar[rr] \ar[dr, "f"'] & & T(f) \ar[dl, "ev_{0}"] \\
		& Y & \,.
	\end{tikzcd}
\end{equation*}

\begin{exercise}
	La valutazione $p=ev_{0}:T(f) \to Y$ è una fibrazione.
\end{exercise}

\begin{df}
	La \textbf{fibra omotopica} di $f$
	è la fibra di $p:T(f) \to Y$ nel punto base di $Y$,
	che è data dallo spazio
	\begin{equation*}
		F(f) = \Set{(x,\omega) \in X \times Y^{I}_{\ast} | \omega(1) = f(x)}\,,
	\end{equation*}
	dove $0$ è il punto base dell'intervallo $I$.
\end{df}

La costruzione di $F(f)$ può essere descritta in
maniera funtoriale come il pullback del seguente diagramma:
\begin{equation}\label{htpfib-pullback}
	\begin{tikzcd}
		F(f) \ar[r] \ar[d, "p(f)"'] & P(Y) = Y^{I}_{\ast} \ar[d, "ev_{1}"] \\
		X \ar[r, "f"] & Y \,.
	\end{tikzcd}
\end{equation}
Nel caso in cui $f$ sia già una fibrazione,
questa costruzione non ci dà nulla di nuovo,
nel senso spiegato dal seguente

\begin{lemma}
	Se $p:E \to B$ è una fibrazione,
	allora l'inclusione naturale $F \hookrightarrow F(p)$
	è un'equivalenza omotopica.
	\begin{proof}
		Vediamo la fibra omotopica $F(p)$ come pullback
		tramite $p$ di $PB \to B$,
		dove lo spazio dei cammini liberi 
		$PB = B^{I}_{\ast}$ è contraibile.
		\begin{equation*}
			\begin{tikzcd}
			F(p) \ar[r] \ar[d, "p(f)"'] 
			& E \ar[d, "p"] \\
			PB \ar[r, "ev_{1}"'] & B \,.
			\end{tikzcd}
		\end{equation*}
		Questo significa che ``l'inclusione della fibra di $F(p) \to PB$
		in $F(p)$ è un'equivalenza omotopica'' (vedi Esercizio dopo),
		tale che la fibra è proprio $F = p^{-1}(\ast)$.
	\end{proof}
\end{lemma}

\begin{exercise}
	In una fibrazione a base contraibile,
	l'inclusione della fibra nello spazio totale $F \subset E$
	è una equivalenza omotopica.
\end{exercise}


La mappa $p(f):F(p) \to X$ è una fibrazione, 
con fibra lo spazio dei loop 
\begin{equation*}
	\Omega Y = \Set{\omega \in Y^{I}_{\ast} | \omega(1) = \omega(0) = \ast}\,,
\end{equation*}
quindi reiterando la costruzione
si ottiene la \textbf{successione di fibrazione di Barret-Puppe}
\begin{equation}
	\begin{tikzcd}\label{BPfib}
		\dots \ar[r] \Omega F(f) \ar[r, "\Omega p"]
		& \Omega X \ar[r, "\Omega f"]
		& \Omega Y \ar[r, "i"]
		& F(f) \ar[r, "p"]
		& X \ar[r,"f"]
		& Y\,,
	\end{tikzcd}
\end{equation}
che è una successione esatta,
dato che ogni mappa è una fibrazione
(a meno di omotopia).
Dunque, siccome
\begin{equation*}
	[S^{0},\Omega^{k}X] = [S^{k}, X] = \pi_{k}(X,\ast)\,,
\end{equation*}
applicando il funtore $[S^{0},-]$ a \eqref{BPfib}
otteniamo la \textbf{successione esatta lunga in omotopia}
della fibrazione.

\begin{lemma}[\textbf{della fibra omotopica}]\label{lemma-htpfib}
	Sia $(X, A,\ast)$ una coppia puntata e sia $F$ la fibra
	omotopica di $A \subset X$.
	Allora, per ogni $n \ge 1$, c'è un isomorfismo naturale
	\begin{equation*}
		\pi_{n}(X,A,\ast) \simeq \pi_{n-1}(F,\ast)
	\end{equation*}
	tale che il seguente diagramma commuti:
	\begin{equation*}
		\begin{tikzcd}
			\pi_{n}(X,\ast) \ar[r] \ar[dd, "\simeq"]
			& \pi_{n}(X,A,\ast) \ar[dd, "\simeq"] \ar[dr] & \\
			& & \pi_{n-1}(A, \ast) \\
			\pi_{n-1}(\Omega X, \ast) \ar[r, "i"]
			& \pi_{n-1}(F,\ast) \ar[ur, "p"] & \,.
		\end{tikzcd}
	\end{equation*}
	\begin{proof}%[Dimostrazione del Lemma]
		Notiamo innanzitutto che
		\begin{equation*}
			\pi_{1}(X,A,\ast) = [(I, \de I, 0), (X,A,\ast)]
		\end{equation*}
		corrisponde alle componenti connesse per archi di
		\begin{equation*}
			\Set{f:I \to A| f(0) = \ast, f(1) \in A} = F(A \to X)\,,
		\end{equation*}
		dunque $\pi_{0}(F, \ast)$. Invece, per $n \ge 1$ si ha
		\begin{equation*}
			\Set{f:(I^{n}, \de I^{n}, J^{n-1}) \to (X,A,\ast)}
			= \Omega^{n-1} F(A \to X)\,,
		\end{equation*}
		e si verifica facilmente che il diagramma commuta.
	\end{proof}
\end{lemma}

\begin{cor}
	La succesione di omotopia della coppia è esatta
	e per $n \ge 2$ è una successione esatta di gruppi,
	mentre per $n \ge 3$ è successione esatta di gruppi abeliani.
\end{cor}

\begin{prop}
	Sia $p:E \to B$ una fibrazione e,
	date funzioni continue
	$f_{0},f_{1}: X \to B$,
	siano $E_{0}$ e $E_{1}$ i rispettivi
	pullback di $p$.
	Se $f_{0} \sim f_{1}$, allora $E_{0}$ e $E_{1}$ sono
	omotopicamente equivalenti.
	\begin{proof}
		L'idea è costruire una fibrazione su $B^{X}$ 
		tale che la fibra di $f$ sia $f^{*}E$,
		cioè il pullback di $E$ tramite $f$.
		Scriviamo
		\begin{equation*}
			\begin{tikzcd}
				& E \times_{B} (B^{X} \times X) \ar[r] \ar[d]
				& E \ar[d, "p"] \\
				\ast \times X \ar[r] \ar[d] 
				& B^{X} \times X \ar[r, "ev"] \ar[d, "\pi_{B^{X}}"] & B \\
				\ast \ar[r, "in_{f}"] & B^{X} & \,,
			\end{tikzcd}
		\end{equation*}
		dove $in_{f}: \ast \mapsto f$. Dato che la composizione
		\begin{equation*}
			\begin{tikzcd}[row sep=small]
				X \ar[r] \ar[rr, bend left=20, "f", shorten=2ex] 
				& B^{X} \times X \ar[r, "ev"'] & B \\
				x \ar[r, mapsto] 
				& (f,x) \ar[r, mapsto] & f(x)\,,
			\end{tikzcd}
		\end{equation*}
		allora il diagramma sopra può essere completato da $f^*E$
		nel quadrato in alto a sinistra:
		\begin{equation*}
			\begin{tikzcd}
				f^*E \ar[d, dashed] \ar[r, dashed] 
				& E \times_{B} (B^{X} \times X) \ar[r] \ar[d]
				& E \ar[d, "p"] \\
				\ast \times X \ar[r] \ar[d] 
				& B^{X} \times X \ar[r, "ev"] \ar[d, "\pi_{B^{X}}"] & B \\
				\ast \ar[r, "in_{f}"] & B^{X} & \,.
			\end{tikzcd}
		\end{equation*}
		%dunque tutte le fibre sono omotopicamente equivalenti.
		Siccome un'omotopia tra $f_{0}$ e $f_{1}$ 
		corrisponde a un cammino in $B^{X}$,
		si conclude grazie all'Esercizio che segue.
	\end{proof}
\end{prop}

\begin{exercise}
	Un cammino $\gamma : I \to B$ induce un'equivalenza omotopica
	tra $p^{-1}(\gamma(0))$ e $p^{-1}(\gamma(1))$.
	\begin{equation*}
		\begin{tikzcd}
			F_{\gamma(0)} \ar[r] \ar[d, "\iota_{0}"', hook] 
			& E \ar[d] \\
			I \times F_{\gamma(0)} \ar[r, "\gamma"] \ar[ur, "\Gamma", dashed]
			& B\,.
		\end{tikzcd}
	\end{equation*}
\end{exercise}

\begin{lemma}
	Data una fibrazione $p:E \to B$, con punti base
	$e_{0} \in E, b_{0} \in B$ e $p(e_{0}) = b_{0}$,
	allora
	\begin{equation*}
		p_{*} : \pi_{*}(E,F,e_{0}) \longrightarrow \pi_{*}(B,b_{0})
	\end{equation*}
	è un isomorfismo.
	\begin{proof}
		Nella dimostrazione scriveremo $\mathrm{htpfib}$ per indicare la fibra omotopica,
		mentre $\mathrm{fib}$ è la semplice fibra (la controimmagine del punto base).
		Dato che $F(p) \to E$ è una fibrazione,
		sappiamo che $F \hookrightarrow F(p)$
		è un'equivalenza omotopica.
		La Proposizione ci dice che
		\begin{equation*}
			\mathrm{htpfib}(F \subset E) \simeq \mathrm{htpfib}(F(p) \to E)\,.
		\end{equation*}
		Inoltre, per il Lemma si ha
		\begin{equation*}
			\mathrm{htpfib}(F(p) \to E) \simeq \mathrm{fib}(F(p) \to E)\,,
		\end{equation*}
		dove quest'ultima è lo spazio
		\begin{equation*}
			\mathrm{fib}(F(p) \to E) = 
			\Set{(e_{0},\omega) \in E \times B^{X} | \omega(1) = p(e_{0}) = b_{0}}
			= \Omega B\,.
		\end{equation*}
		La tesi segue dunque dal seguente Esercizio.
	\end{proof}
\end{lemma}

\begin{exercise}
	L'isomorfismo ottenuto dalla composizione 
	\begin{equation*}
		\pi_{n}(E,F) \simeq \pi_{n-1}(\mathrm{htpfib}(F \subset E), \ast)
		\to \pi_{n-1}(\Omega B, \ast) \to \pi_{n}(B, b_{0})
	\end{equation*}
	è la mappa indotta da $p$.
\end{exercise}

\begin{exercise}
	Data la mappa puntata $f: \Sigma X \to Y$,
	sia $\Hat{f}:X \to \Omega Y$ la mappa aggiunta.
	Costruire $g:CX \to PY=Y^{I}_{*}$ tale che il diagramma
	\begin{equation*}
		\begin{tikzcd}
			X \ar[r, hook] \ar[d, "\Hat{f}"] 
			& CX \ar[d, "g"] \ar[r] 
			& \Sigma Y \ar[d, "f"] \\ 
			\Omega Y \ar[r] & PY \ar[r] & Y
		\end{tikzcd}
	\end{equation*}
	commuti.
\end{exercise}




\chapter{Spazi di Eilenberg-MacLane}

\section{Torri di Postnikov}

Ricordiamo il seguente teorema di \textbf{approssimazione CW}.
\begin{thm}[\textbf{di approssimazione CW}]\label{CW-approx}
	Sia $A$ un CW complesso, $k \ge -1$ e $Y$ uno spazio topologico qualsiasi.
	Sia $f:A \to Y$ tale che
	$f_{*}:\pi_{i}(A,\ast) \to \pi_{i}(Y,\ast)$
	siano isomorfismi per $i < k$ e surgettiva per $i = k$.
	Allora per ogni $n > k$ (eventualmente anche $n=\infty$),
	esiste un CW complesso $X$ tale che $A \subset X$ sottocomplesso,
	e esiste un'estensione $F:X \to Y$ tale che
	\begin{equation*}
		F_{i} : \pi_{q}(X,\ast) \longrightarrow \pi_{q}(Y,\ast)
	\end{equation*}
	è un isomorfismo per $q < n$ e surgettiva per $q=n$
	\begin{proof}[Idea della dimostrazione]
		L'idea è di costruire $X$ 
		incollando su $A$ celle di dimensione $d$,
		con $k \le i \le n$,
		sfruttando le informazioni contenute
		negli isomorfismi dei gruppi di omotopia.
	\end{proof}
\end{thm}

Un procedimento analogo mi dà:

\begin{thm}[\textbf{Sezioni di Postnikov}]\label{sez-Postnikov}
	Per ogni spazio topologico $X$ e
	per ogni $n \ge 0$,
	esiste uno spazio $P_{n}(X)$ e una mappa $X \to P_{n}(X)$
	tale che per ogni punto base $\ast$
	\begin{rmnumerate}
		\item la mappa indotta
		$\pi_{q}(X, \ast) \to \pi_{q}(P_{n}(X),\ast)$
		è un isomorfismo per $q \le n$;
		\item i gruppi di omotopia di grado $q > n$ si annullano: $\pi_{q}(P(X),\ast)=0$; 
		\item la coppia $(P_{n}(X),X)$ è un CW complesso relativo 
		con celle di dimensione almeno $n+2$.\footnote{Quindi capiamo che le celle sono nella ``parte non relativa'' $P_{n}(X) \setminus X$. Gli spazi $P_{n}$ sono ottenuti incollando su $X$ celle di dimensione $\ge n+2$, in modo tale che il loro bordo sia ($n+1$)-dimensionale e si riesca a ``uccidere'' l'omotopia in grado $\ge n$.}
	\end{rmnumerate}
	\begin{proof}
		Possiamo assumere che $X$ sia connesso, lavorando una componente
		connessa per volta. Il nostro scopo è quello di ``uccidere'' i
		gruppi di omotopia superiore di $X$ andando a uccidere
		una successione di spazi
		\begin{equation*}
			X = X(n) \longrightarrow X(n+1) \longrightarrow X(n+2) \longrightarrow \dots
		\end{equation*}
		con omotopia uguale a $\pi_{q}(X,\ast)$ per ogni $q \le n$,
		ma l'$m$-esimo spazio avrà $\pi_{q}(X(m),\ast)=0$ per $n < q \le m$.
		
		Poniamo $X(n):=X$ e supponiamo di aver costruito $X(m-1)$.
		Per ottenere $X(m)$ uccidendo $\pi_{m}(X(m-1),\ast)$
		e senza ottenere nuove relazioni operiamo nel seguente modo:
		scegliamo dei generatori $\gamma_{i}$ di $\pi_{m}(X(m-1),\ast)$
		e dei loro rappresentanti $g_{i}:S^{m} \to X(m-1)$,
		che usiamo come mappe di incollamento di $(m+1)$-celle su $X(m-1)$.
		In questo modo, la coppia $\big(X(m),X(m-1)\big)$ così ottenuta 
		è $m$-connessa\footnote{Ricordiamo che una coppia $(X,Y)$ è \textbf{$m$-connessa} 
		se l'inclusione $Y \subset X$ induce un isomorfismo di $pi_{q}$ 
		per ogni $0 \le q \le m$.},
		dunque considerando la successione esatta lunga in omotopia
		\begin{equation*}
			\begin{tikzcd}[column sep=small]
				\pi_{m+1}\big(X(m),X(m-1)\big) \ar[r, "\de"]
				& \pi_{m}\big(X(m-1)\big) \ar[r]
				& \pi_{m}\big(X(m)\big) \ar[r]
				& \pi_{m}\big(X(m),X(m-1)\big) = 0\,,
			\end{tikzcd}
		\end{equation*}
		deduciamo che $\pi_{m}\big(X(m),\ast\big)=0$ perché $\de$ è surgettiva per costruzione.
		Si verifica infine che lo spazio $P_{n}(X) := \bigcup_{m \ge n} X(m)$
		soddisfa le nostre richieste.
	\end{proof}
\end{thm}

\begin{df}
	Chiamiamo lo spazio $P_{n}(X)$ l'\textbf{$n$-esima sezione di Postnikov}.
\end{df}

	%%% Lecture 3

\lecture[Esistenza delle sezioni di Postnikov. Unicità delle sezioni di Postnikov a meno di equivalenza omotopica debole. Costruzione della torre di Postnikov. Spazi di Moore e spazi di Eilenberg-MacLane.]{2023-03-06}

% L'ultima volta abbiamo concluso la lezione affermando che esistono
% le \textbf{sezioni di Postnikov}.

\begin{ex}
	La sezione $P_{0}$ è uno spazio debolmente omotopicamente
	equivalente a $\pi_{0}(X)$.
\end{ex}

\begin{ex}
	Sia $X$ uno spazio connesso per archi.
	La sezione $P_{1}(X)$ è uno spazio connesso per archi con
	\begin{equation*}
		\pi_{1}(P_{1}(X),\ast) \simeq \pi_{1}(X)
	\end{equation*}
	e con rivestimento universale \emph{debolmente contraibile}\footnote{Uno spazio $Y$ è \textbf{debolmente contraibile} se $\pi_{k}(Y,\ast) = 0$, per ogni $k>0$.}.
\end{ex}

\begin{df}
	Uno spazio topologico $X$
	si dice \textbf{asferico} se il suo 
	rivestimento universale è contraibile.
\end{df}

\begin{oss}
	Per ogni gruppo $\pi$,
	esiste un CW complesso $K$ connesso per archi e asferico
	tale che $\pi_{1}(K,\ast) = \pi$ .
	Infatti, possiamo sempre considerare una presentazione
	\begin{equation*}
		\begin{tikzcd}
			F_{1} \ar[r, "a"] & F_{0} \ar[r] & \pi \ar[r] & 0\,, 
		\end{tikzcd}
	\end{equation*}
	con $F_{1}$ e $F_{0}$ gruppi liberi\footnote{Ricordiamo il \textbf{Teorema di Nielsen-Schreier}: un sottogruppo di un gruppo libero è a sua volta libero.};
	quindi, denotando con $I_{j}$ un insieme libero di generatori per $F_{j}$,
	possiamo definire una mappa
	\begin{equation*}
		\begin{tikzcd}
			\bigvee_{I_{1}} S^{1} \ar[r, "\phi_{a}"] & \bigvee_{I_{0}} S^{1}\,.
		\end{tikzcd}
	\end{equation*}
	Allora, grazie al \textbf{Teorema di Van Kampen} si può verificare
	che il mapping cone $C_{\phi_{a}}$ è un CW complesso
	che ha il gruppo fondamentale desiderato;
	dato che vogliamo uno spazio asferico, allora consideriamo
	la sua prima sezione di Postnikov
	$P_{1}\left(C_{\phi_{a}}\right)$.
\end{oss}

\begin{oss}
	Se $X$ è un CW complesso, 
	posso ottenere $P_{n}(X)$ come CW complesso.
\end{oss}

Ci chiediamo dunque se le sezioni di Postnikov siano \emph{uniche}
in qualche senso.
\begin{prop}\label{CW-est}
	Sia $n \ge 0$ e $Y$ uno spazio topologico tale che $\pi_{q}(Y, \ast)=0$,
	per ogni $q>n$. Dato un CW complesso relativo $(X,A)$ in cui
	tutte le celle in $X \setminus A$ hanno dimensione $\ge n+2$,
	allora la mappa di restrizione
	\begin{equation*}
		[X,Y] \longrightarrow [A,Y]
	\end{equation*}
	è bigettiva. Se ci sono anche ($n+1$)-celle,
	allora è iniettiva.
	\begin{proof}
		Usiamo che l'ipotesi $\pi_{q}(Y,\ast)=0$, per ogni $q > n$,
		è equivalente al fatto che	
		le mappe $S^{q} \to Y$ con immagine nella componente
		connessa contenente $\ast$ si estendono a tutto $D^{q+1}$.
		\begin{itemize}
			\item \textbf{Surgettività}: data $f:A \to Y$,
			per ogni $q \ge n+2$ la mappa di incollamento
			\begin{equation*}
				S^{q-1} \longrightarrow X^{q-1} \longrightarrow Y
			\end{equation*}
			è omotopicamente banale per ipotesi,
			dunque si estende a una mappa $D^{q} \to Y$.
			Ripetendo questo procedimento per tutte
			le celle in $X \setminus A$, definiamo così
			un'estensione $\overline{f}:X \to Y$;
			
			\item \textbf{Iniettività}:
			consideriamo la coppia
			\begin{equation*}
				\Big(X \times I, (X \times \de I) \cup (A \times I) \Big)
			\end{equation*}
			come un CW complesso relativo.
			Se $q \ge n+1$, per ipotesi sull'omotopia ogni $q$-cella di $X \setminus A$
			è bordo di una cella di dimensione $q+1$,\todo{È giusto?}
			quindi ogni coppia di mappe $f_{0},f_{1}:X \to Y$ 
			le cui restrizioni ad $A$ sono omotope inducono un'omotopia
			\begin{equation*}
				F : \Big( X \times \de I \Big) \cup \Big( A \times I \Big)
				\longrightarrow Y\,.
			\end{equation*}
			Ripetendo il ragionamento del punto precedente,
			possiamo estendere $F$ una cella alla volta,
			fino a ottenere un'omotopia $\widetilde{F}:X \times I \to Y$
			tra $f_{0}$ e $f_{1}$.
		\end{itemize}
	\end{proof}
\end{prop}

\begin{cor}
	Sia $X$ un CW complesso $n$-connesso e
	$Y$ uno spazio con omotopia concentrata in grado al più $n$\footnote{Cioè tale che $\pi_{q}(Y,\ast) = 0$ per ogni $q > n$.}.
	Ogni mappa $f:X \to Y$ è omotopa a una costante.
	\begin{proof}
		Per ipotesi, possiamo considerare $X$
		come un CW complesso senza celle di dimensione $0< d < n+1$.
		Poniamo $A = \{\ast\}$ il punto base di $X$.
		Siccome l'$n$-scheletro di $X$ consiste del solo punto base,
		i.e. $X_{n} = \{\ast\}$, allora la coppia $(X,\ast)$
		soddisfa le ipotesi della \hyperref[CW-est]{Proposizione~\ref{CW-est}}
		e abbiamo una bigezione $[X,Y] \to [\ast,Y]$.
	\end{proof}
\end{cor}

Sia $f:X \to Y$ e consideriamo le mappe $X \to P_{m}(X)$ e $Y \to P_{n}(Y)$.
Se $m \ge n$, allora per la \hyperref[CW-est]{Proposizione~\ref{CW-est}}
abbiamo una bigezione $[X,P_{n}(Y)] \simeq [P_{m}(X),P_{n}(Y)]$,
quindi \emph{esiste un'unica} mappa (\emph{a meno di omotopia}) tra
le sezioni di Postnikov che fa commutare il seguente diagramma:
\begin{equation*}
	\begin{tikzcd}
		X \ar[r, "f"] \ar[d] & Y \ar[d] \\
		P_{m}(X) \ar[r, "\exists !", dashed] & P_{n}(Y)\,.
	\end{tikzcd}
\end{equation*}
Se consideriamo il caso particolare in cui $X=Y$ e $f =\cat{1}_{X}$,
allora per ogni $m \ge n$ abbiamo un'unica mappa
\begin{equation*}
	\begin{tikzcd}
		& X \ar[dl] \ar[dr] & \\
		P_{m}(X) \ar[rr, dashed, "\exists !"] & & P_{n}(X)\,,
	\end{tikzcd}
\end{equation*}
dunque ponendo $m=n$ si deduce che deve essere un'equivalenza omotopica \emph{debole}:
questo significa che tutte le costruzioni 
che soddisfino le condizioni del \hyperref[sez-Postnikov]{Teorema~\ref{sez-Postnikov}}
producono spazi topologici debolmente equivalenti.
Quindi le sezioni di Postnikov sono uniche in questo senso!
Questo implica che se è possibile ottenere $P_{n}(X)$ 
come CW complessi (e.g. quando $X$ è un CW complesso),
allora le sezioni di Postnikov sono uniche a meno di equivalenza omotopica.

\begin{oss}
	La mappa $X \to P_{m}(X)$ è \emph{iniziale} nella
	categoria $\cat{HoTop}$ degli spazi topologici e 
	funzioni continue a meno di omotopia
	verso gli spazi con omotopia banale in grado $> n$:
	più esplicitamente, se $Y$ è uno spazio con $\pi_{q}(Y,\ast)=0$
	per $q > n$, allora ogni mappa $X \to Y$ fattorizza in maniera
	unica (a meno di omotopia) attraverso l'$n$-esima sezione di Postnikov:
	\begin{equation*}
	\begin{tikzcd}
		& P_{n}(X) \ar[d, "\exists !", dashed] \\
		X  \ar[r] \ar[ur] & Y \,.
	\end{tikzcd}
\end{equation*}
\end{oss}

Partendo da $Y=P_{0}(X)$ e $n=1$, reiterando la costruzione sopra
otteniamo una sorta di diagramma limite chiamato
\textbf{torre di Postnikov}.
\begin{equation}\label{TP}
	\begin{tikzcd}
		& \dots \ar[d] \\
		& P_{2}(X) \ar[d] \\
		& P_{1} (X) \ar[d] \\
		X \ar[uur] \ar[ur] \ar[r] & P_{0}(X) \,.
	\end{tikzcd} \tag{TP}
\end{equation}

\section{Spazi di Eilenberg-MacLane}

\begin{prop}\label{Moore-exist}
	Dato $k \ge 1$ e $A$ un gruppo abeliano,
	esiste un CW complesso $X$ tale che
	\begin{equation*}
		\widetilde{H}_{k}(X) = A\,,
		\quad \widetilde{H}_{q}(X) = 0 \text{ per } q \ne k\,.
	\end{equation*}
	Se $k > 1$, il complesso $X$ può essere costruito
	semplicemente connesso.
	\begin{proof}
		Consideriamo una presentazione
		\begin{equation*}
			\begin{tikzcd}
				0 \ar[r] & F_{1} \ar[r, "a"] & F_{0} \ar[r] & A \ar[r] & 0\,, 
			\end{tikzcd}
		\end{equation*}
		con $F_{1}$ e $F_{0}$ gruppi abeliani liberi,
		e sia $G_{i}$ una base di generatori per $F_{i}$.
		Consideriamo allora il bouquet di circonferenze
		\begin{equation*}
			X^{(k)} := \bigvee_{a \in G_{0}} S^{k}_{a}
		\end{equation*}
		e definiamo la mappa di incollamento
		\begin{equation*}
			\alpha : \bigsqcup_{b \in G_{1}} S^{k}_{b}
			\longrightarrow X^{(k)}
		\end{equation*}
		indotta tramite l'omomorfismo $F_{1} \longrightarrow F_{0}$
		dato da $b \longmapsto \sum_{a \in G_{0}} n_{a} a$, cioè:
		se $n_{a} \ne 0$, allora il disco $b$-esimo si incolla su $S_{a}^{k}$
		con una mappa $S_{b}^{k} \to S_{a}^{k}$ di grado $n_{a}$.
		\missingfigure{C'è un disegnetto con le sferette che si incollano.}
		Per costruzione $\alpha$ realizza $F_{1} \to F_{0}$ in omologia,
		quindi otteniamo
		\begin{equation*}
			\widetilde{H}_{q}(X) =
			\begin{cases}
				A\,, \quad &\text{se } q = k\,; \\
				0\,, \quad &\text{altrimenti.}
			\end{cases}\qedhere
		\end{equation*}
	\end{proof}
\end{prop}

\begin{df}
	Uno spazio $X$ come nella \hyperref[Moore-exist]{Proposition~\ref{Moore-exist}}
	si chiama \textbf{spazio di Moore} e si indica con $M(A,k)$.
\end{df}

\begin{fact}
	Una costruzione analoga permette di costruire
	anche CW complessi $X$ di dimensione $2$,
	il cui gruppo fondamentale è $\pi_{1}(X,\ast) = \pi$,
	per un qualsiasi gruppo $\pi$ prefissato.
\end{fact}

Il \textbf{Teorema di Hurewicz} ci dice che l'omotopia di
uno spazio di Moore $M(\pi, n)$, con $n \ge 2$, 
coincide con l'omologia
fino al grado $n$, ma in grado superiore non sappiamo nulla;
possiamo ``uccidere'' la sua omotopia in grado superiore a $n$
considerando il ``troncato'' $n$-esimo, ovvero la sezione di Postnikov
\begin{equation*}
	\tau_{\le n}M := P_{n}(M)\,,
\end{equation*}
così da ottenere uno spazio con omotopia $\pi_{n}(\tau_{\le n}M, \ast) = \pi$ 
e $\pi_{q}(\tau_{\le n}M,\ast)=0$ se $q \ne n$.

\begin{df}
	Uno spazio $X$ tale che $\pi_{n}(X,\ast)=\pi$ e
	\begin{equation*}
		\pi_{q}(X,\ast) = 0\,, \quad \text{se } q \ne n
	\end{equation*}
	si chiama \textbf{spazio di Eilenberg-MacLane} di tipo
	$K(\pi,n)$.
\end{df}

\begin{ex}
	Abbiamo già incontrato degli spazi di Eilenberg-MacLane nel
	corso di Elementi di Topologia Algebrica, come ad esempio:
	\begin{itemize}
		\item $K(\Z,1) = S^{1}$, il cui rivestimento universale è la retta $\R$;
		\item $K(\Z/2,1) = \R\PP^{\infty}$, 
		il cui rivestimento universale è la sfera $S^{\infty}$;
		\item $K(\Z,2) = \C\PP^{\infty}$, 
		il cui rivestimento universale è sempre la sfera $S^{\infty}$.
	\end{itemize}
\end{ex}

\begin{ex}
Se consideriamo la torre di Postnikov
\begin{equation*}
	\begin{tikzcd}
		& \dots	\ar[d] \\	
		& \tau_{\le 2}X \ar[d, "t_{2}"] \\
		& \tau_{\le 1}X \ar[d] \\
		X \ar[r] \ar[ur] \ar[uur]
		& \tau_{\le 0}X  \,,
	\end{tikzcd}
\end{equation*}
la fibra omotopica di $t_{2}$ è lo\footnote{Nella prossima lezione dimostreremo, infatti, che gli spazi di Eilenberg-MacLane sono unici a meno di equivalenza omotopica.} spazio di Eilenberg-MacLane
$F = K\big(\pi_{2}(X,\ast),2\big)$.\todo{Fai questo esercizio.}
\end{ex}






	%%% Lecture 4

%\section{Spazi di Moore}
%
%
%\textbf{COSE sta ripetendo la costruzione di ieri.}
%
%Costruiamo un CW complesso $X$ di dimensione $k+1$,
%il cui scheletro è formato da
%\begin{itemize}
%	\item una $0$-cella $\ast$;
%	\item le $k$-celle date da $\bigvee_{G_{0}} S^{k}$;
%	\item le $(k+1)$-celle avranno come bordo le $k$-celle sopra.
%\end{itemize}
%
%\textcolor{red}{Riscrivere questa parte nella lezione di ieri per aggiustare.}

\section{Torri di Whitehead}

\lecture[Torri di Whitehead. Fattorizzazione di Moore-Postnikov. Unicità spazi di Eilenberg-MacLane CW complessi. Funtorialità della costruzione degli spazi di Eilenberg-MacLane. Definizione della classe fondamentale di un $K(\pi,n)$. Teorema di rappresentabilità della coomologia. Esempi.]{2023-03-07}

Fissiamo un punto base $x \in X$ e consideriamo la fibra omotopica $F$
dell' $n$-sezione di Postnikov $X \to \tau_{\le n} X$.
Dato che i due spazi hanno la stessa omotopia fino al
grado $n$ e per $q > n$ si ha $\pi_{q}(\tau_{\le n} X,\ast) = 0$,
allora la successione esatta lunga in omotopia è
\begin{equation*}
	\begin{tikzcd}[column sep=small]
		\dots \ar[r] & \pi_{n+1}(F,\ast) \ar[r, "\sim"]
		& \pi_{n+1}(X,x) \ar[r]
		% & \pi_{q+1}(\tau_{\le n}X, \ast) \ar[r]
		& 0 \ar[r]
		& \pi_{n}(F,\ast) \ar[r]
		& \pi_{n}(X,x) \ar[r, "\sim"]
		& \pi_{n}(\tau_{\le n} X,\ast) \ar[r]
		& \dots
	\end{tikzcd}
\end{equation*}
quindi deduciamo che la fibra ha omotopia
\begin{equation*}
	\pi_{q}(F,\ast) =
	\begin{cases}
		0\,, \quad & \text{se } q \le n\,; \\
		\pi_{q}(X,x)\,, \quad & \text{se } q > n\,.
	\end{cases}
\end{equation*}
Siccome la situazione in omotopia è l'opposto di quello che succede
con le sezioni di Postnikov, denotiamo la fibra $\tau_{\ge n} X := F$.

\begin{ex}
	La fibra $\tau_{\ge 1}X$ è la componente connessa per archi del punto base $x \in X$,
	se $X \to \pi_{0}(X,x)$ è continua.\todo{Che significa?}
\end{ex}

\begin{ex}
	La sezione $\tau_{\ge 2}X$ è uno spazio semplicemente  connesso,
	con la stessa omotopia superiore di $X$: deduciamo che è il
	rivestimento univesale di $X$ (nel caso in cui $X$ sia connesso per archi
	e ammette rivestimento universale).
\end{ex}

\begin{exercise}
	La mappa $\tau_{\ge n+1} X \to X$ è unica a meno di omotopia \emph{puntata}
	ed è la \textbf{mappa terminale} nella categoria $\textbf{HoTop}_{\ast}$
	delle mappe da spazi $n$-connessi in $X$, i.e. se $Y$
	è uno spazio con omotopia banale fino al grado $n$, 
	allora ogni mappa continua $Y \to X$ fattorizza attraverso $\tau_{\ge n+1}X$.
	\begin{equation*}
		\begin{tikzcd}
			Y \ar[dr] \ar[d, dashed, "\exists !"] & \\
			\tau_{\ge n} X \ar[r] & X \,.
		\end{tikzcd}
	\end{equation*}
\end{exercise}

Partendo da $Y=X$ e ripetendo la costruzione induttivamente,
otteniamo il diagramma
\begin{equation}\label{TW}
	\begin{tikzcd}
		\dots \ar[d] & \\	
		\tau_{\ge 2}X \ar[d, "t^{2}"'] \ar[ddr] & \\
		\tau_{\ge 1}X \ar[d] \ar[dr] & \\
		\tau_{\ge 0}X  \ar[r, equals] 
		& X ,,
	\end{tikzcd}\tag{TW}
\end{equation}
chiamato \textbf{torre di Whitehead}.
In modo analogo a quanto accade per le torri di Postnikov,
si verifica che la fibra di $t^{2}$
è uno spazio di Eilenberg-MacLane.

Esiste una costruzione più generale, 
che enunciamo nel seguente teorema (che non dimostriamo).
\begin{thm}[\textbf{Fattorizzazione di Moore-Postnikov}]
	Data una mappa $f:X \to Y$ tra spazi topologici connessi per archi,
	esiste una successione di spazi $T_{0}f, T_{1}f,\dots, T_{\infty}f$
	e un diagramma commutativo
	\begin{equation*}
		\begin{tikzcd}
			X \ar[rrrr, "f"] \ar[d, "\iota_{\infty}"'] 
			\ar[rrd, "\iota_{n}"', shorten=2ex]
			& & & & Y \\
			T_{\infty} f \ar[r] 
			& \dots \ar[r]
			& T_{n} \ar[r] \ar[rru,"s_{n}", shorten=2ex]
			& \dots \ar[r]
			& T_{0} f \ar[u,"s^{0}"'] \,,
		\end{tikzcd}
	\end{equation*}
	tale che $\pi_{q}(\iota_{n})$ sia un isomorfismo per ogni $q<n$
	e $\pi_{q}(s_{n})$ sia un isomorfismo per ogni $q \le n$.
	\begin{proof}
		Vedi \parencite[hatcher]{Theorem~4.71}
	\end{proof}
\end{thm}


\section{Rappresentabilità della coomologia}

	Adesso che conosciamo la costruzione degli spazi $K(\pi,n)$,
	possiamo vedere che questi spazi hanno una natura ``algebrica'',
	che ci permettono di studiare la coomologia di spazi topologici.
	
	
	\begin{lemma}
		Sia $n > 0$ intero e $(Y,y)$ uno spazio puntato tale che $\pi_{q}(Y,y)=0$
		per $q \ne n$. Posto $G := \pi_{n}(Y,y)$, sia $M = K(\pi,n)$.
		La mappa indotta in omotopia
		\begin{equation*}
			\pi_{n} : [M,Y]_{\ast} \longrightarrow \Hom_{\cat{Grp}}(\pi,G)
		\end{equation*}
		è un isomorfismo. In particolare, se $\pi=G$, segue che $Y$
		è omotopicamente equivalente a $M$.
		\begin{proof}
			Sia $F_{0} \to F_{1} \to \pi \to 0$ una risoluzione libera di gruppi abeliani,
			e consideriamo la corrispondente successione di cofibrazione
			\begin{equation*}
				\bigvee_{I_{1}} S^{n} \longrightarrow
				\bigvee_{I_{0}} S^{n} \longrightarrow
				M\,.
			\end{equation*}
			Se denotiamo con $X$ il cono della mappa di incollamento celle
			\begin{equation*}
				X := C\left( \bigvee_{I_{1}} S^{k} \to \bigvee_{I_{0}} S^{k} \right)\,,
			\end{equation*}
			allora per la dimostrazione della \hyperref[CW-est]{Proposizione~\ref{CW-est}}\todo{Aiuto, è giusto?}
			sappiamo che $\tau_{\le n} X \simeq X$; dato che la sezione
			di Postnikov $X \to M$ è iniziale tra le mappe da $X$
			agli spazi $Y$ con omotopia nulla sopra $n$,
			questo significa che $[M,Y]_{\ast} \to [X,Y]_{\ast}$ è un isomorfismo.
			Quindi ci basta mostrare che
			\begin{equation*}
				\widetilde{\pi_{n}} : [X,Y]_{\ast} \longrightarrow \Hom_{\cat{Grp}}(\pi,G)
			\end{equation*}
			è un isomorfismo.
			Siccome $X$ è definito da una successione di cofibrazione,
			quindi una successione co-esatta, applicando $[-,Y]_{\ast}$
			otteniamo la successione esatta di insiemi puntati
			\begin{equation*}
				\begin{tikzcd}
					\left[ \bigvee_{I_{1}} S^{n},  Y\right]_{\ast} \longleftarrow
					\left[ \bigvee_{I_{0}} S^{n}, Y \right]_{\ast} \longleftarrow
					\left[ M, Y \right]_{\ast} \longleftarrow
					\left[ \bigvee_{I_{1}} S^{n+1},  Y\right]_{\ast}\,,
				\end{tikzcd}
			\end{equation*}
			dove abbiamo usato che la sospensione commuta con fare il bouquet:
			\begin{equation*}
				\Sigma \bigvee S^{k} \simeq \bigvee S^{k+1}\,.
			\end{equation*}
			Riscrivendo $[S^{n},Y]_{\ast} = \pi_{n}(Y,y) \simeq \Hom(\Z,G)$,
			si ha
			\begin{equation*}
				\left[\bigvee_{I_{i}} S^{n},Y \right]_{\ast} \simeq \Hom(F_{i},G)\,,
			\end{equation*}
			quindi la successione esatta di sopra diventa quindi
			\begin{equation*}
				\begin{tikzcd}
					\Hom(F_{1},G) 
					& \Hom(F_{0},G) \ar[l, "f"']
					& {[X,Y]_{\ast}} \ar[l]
					& 0 \,, \ar[l] 
				\end{tikzcd}
			\end{equation*}
			dove $\ker f = \Hom(\pi,G)$.
			
			In particolare, se $\pi = G$ possiamo considerare l'unica classe
			in $[M,Y]_{\ast}$ corrispondente all'identità $\cat{1}_{G} \in \Hom(G,G)$.
			Questa mappa è un'equivalenza omotopica debole,
			dunque se $Y$ è un CW complesso deduciamo che è un'equivalenza omotopica.
		\end{proof}
	\end{lemma}

	Quindi due CW complessi $K(\pi,n)$ sono omotopicamente equivalenti tramite
	una mappa \emph{unica} (\emph{a meno di omotopia}) che induce l'identità su $\pi_{n}$.
	\begin{cor}
		Per $n > 0$ esiste un funtore
		\begin{equation*}
			K(-,n) : \Ab \longmapsto \cat{HoCW}_{\ast}\,,
			\quad \pi \longmapsto K(\pi,n)
		\end{equation*}
		che è unico a meno di isomorfismo.
		Inoltre, per $n=1$ questo funtore si estende a 
		\begin{equation*}
			K(-,1) : \Grp \longmapsto \cat{HoCW}_{\ast}\,.
		\end{equation*}
	\end{cor}
	
	\begin{oss}
		Grazie ai funtori $K(-,n)$ possiamo definire, in maniera puramente
		\emph{topologica}, dei nuovi invarianti per gruppi, 
		rispettivamente l'\textbf{omologia} e la \textbf{coomologia di gruppi}
		a coefficienti in $G$:
		\begin{align*}
			\pi \longmapsto K(\pi,n) \longmapsto H_{*}(K(\pi,n);G)\,,
			\quad \pi \longmapsto K(\pi,n) \longmapsto H^{*}(K(\pi,n);G)\,.
		\end{align*}
	\end{oss}
	
	\begin{ex}
		La coomologia di $\Z/2$ a coefficienti in $\Z/2$ è
		\begin{equation*}
			\Z/2 \longmapsto H^{\ast}(\R\PP^{\infty}; \Z/2) \simeq \Z/2[x]\,.
		\end{equation*}
	\end{ex}
	
	Dato $n>0$ e $Y$ uno spazio $(n-1)$-connesso,
	il \textbf{Teorema dei Coefficienti Universali} ci dice
	che, per un gruppo abeliano $G$, si ha la successione esatta corta
	\begin{equation*}
		\begin{tikzcd}
			0 \ar[r]
			& \Ext(H_{q-1}(Y);G) \ar[r]
			& H^{q}(Y;G) \ar[r]
			& \Hom(H_{q}(Y);G) \ar[r]
			& 0\,
		\end{tikzcd}
	\end{equation*}
	 per ogni $q>0$, e quindi la coomologia $H^{q}(Y;G) = 0$ per $q < n$.
	 Quando $q=n$, si annulla il termine $\Ext$ e quindi la coomologia
	 è proprio il duale
	 \begin{equation*}
	 	H^{n}(Y;G) \simeq \Hom(H_{q}(Y),G)\,.
	 \end{equation*}
	Se $G=\pi_{n}(Y,y)$, allora l'inverso dell'isomorfismo di Hurewicz
	appartiene a $\Hom(H_{n}(Y),G)$, dunque corrisponde a una classe
	$\iota_{n} \in H^{n}(Y;G)$.
	\begin{df}
		Per $Y = K(\pi,n)$, otteniamo una classe canonica
		\begin{equation*}
			\iota_{n} \in H^{n}\big(K(\pi,n);\pi\big)
		\end{equation*}
		chiamata \textbf{classe fondamentale}.
	\end{df}
	La classe $\iota_{n}$ ci fornisce una trasformazione naturale
	\begin{equation}\label{classe-k-pi-n}
		\rho: [X,K(\pi,n)]_{\ast} \longrightarrow H^{n}(X;\pi)
		\quad [f] \longmapsto f^{*}\iota_{n}\,.
	\end{equation}
	che permette di dare una visione più \emph{geometrica} ai gruppi di coomologia:
	infatti, nel corso di Elementi di Topologia Algebrica abbiamo ottenuto $H^{n}$
	tramite un processo \emph{algebrico astratto}, 
	dualizzando il complesso di catene singolari di uno spazio $X$;
	invece, adesso possiamo interpretare le classi di coomologia
	come di uno spazio $X$ come mappe (a meno di omotopia) 
	a valori in un opportuno spazio di Eilenberg-MacLane.
	
	\begin{thm}\label{coh-rep}
		Se $X$ ha il tipo di omotopia di un CW complesso,
		allora $\rho$ è un isomorfismo.
		\begin{proof}
			Fissato $\pi$, consideriamo degli spazi di Eilenberg-MacLane
			$K(\pi,n)$ per $n \ge 0$. Per prima cosa capiamo
			la struttura di gruppo che esiste su membro di sinistra: %$[X,K(\pi,n)]_{\ast}$:
			se $n=0$, allora $K(\pi,0)=\pi$ con la topologia discreta, dunque
			\begin{equation*}
				[X,K(\pi,0)]_{\ast} = \Set{f:X \to \pi | f \text{ è localmente costante} }
				= H^{0}(X;\pi)\,.
			\end{equation*}
			
			C'è una mappa naturale $K(\pi,n) \to \Omega K(\pi,n+1)$
			che è isomorfismo per $\pi_{n}$, quindi mi dà un nuovo modello per $K(\pi,n)$,
			così come anche $K(\pi,n) \to\Omega^{2} K(\pi, n+2)$.
			Si deduce che $[-,K(\pi,n)]_{\ast}$ è un funtore a valori in $\Ab$
			perché
			\begin{equation*}
				\left[X,\Omega^{2}K(\pi,n+2)\right]_{\ast} \simeq 
				\left[\Sigma^{2} X, K(\pi,n+2)\right]_{\ast}\,.
			\end{equation*}
			Mostriamo che la mappa $\rho: [X,K(\pi,n)]_{\ast} \to \widetilde{H}^{n}(X;\pi)$ 
			è un omomorfismo di gruppi: 
			usando lo ``strozzamento'' $\Sigma X \to \Sigma X \vee \Sigma X$
			per costruire la somma
		\begin{equation*}
			\widetilde{H}^{n+1}(\Sigma X;\pi) \times \widetilde{H}^{n+1}(\Sigma X;\pi)
			\longrightarrow \widetilde{H}^{n+1}(\Sigma X \vee \Sigma X;\pi)
			\longrightarrow \widetilde{H}^{n}(\Sigma X;\pi)\,,
		\end{equation*}
		questo induce l'operazione di gruppo su $\widetilde{H}^{n}(\Sigma X;\pi)$.
		Analogamente, applicando $[-,K(\pi,n)]_{\ast}$ otteniamo un gruppo dato che
		\begin{equation*}
			[X,K(\pi,n)]_{\ast} = [X,\Omega K(\pi,n+1)]_{\ast} = [\Sigma X,K(\pi,n+1)]_{\ast}
		\end{equation*}
		e lo strozzamento della sospensione di $X$ induce la somma come prima.
		
		Mostriamo che $\rho$ induce un isomorfismo per induzione sullo scheletro di $X$.
		Se $X=X^{(0)}$, allora
		\begin{align*}
			\Hom_{\cat{HoTop}_{\ast}}(X^{(0)},\pi) = \widetilde{H}^{0}\left(X^{(0)};\pi\right)\,, \\
			\left[ X^{(0)}, K(\pi,n) \right] = 0 
			= \widetilde{H}^{n}\left(X^{(0)};\pi\right)\,, \quad \text{per } n>0\,.
		\end{align*}
		e questo conclude il passo base. 
		
		Se assumiamo $X$ connesso di dimensione $q$ finita, c'è una successione di cofibrazione
		\begin{equation*}
			\bigvee_{} S^{q-1} \longrightarrow X^{(q-1)} \longrightarrow X^{(q)}
		\end{equation*}
		che è co-esatta, quindi $[-,K(\pi,n)]_{\ast}$ induce una successione esatta in 
		coomologia ridotta. Usando il passo induttivo,
		$\rho$ è un isomorfismo per $\bigvee S^{q-1}$ e per $X^{(q-1)}$,
		che hanno dimensione al più $q-1$, quindi usando
		il \textbf{Lemma dei 5} scopriamo che è isomorfismo anche per $X^{(q)}$.
		
		Se $X$ ha dimensione infinita, sfruttiamo un argomento di limite
		e il fatto che, per ogni $N > q \ge 0$, abbiamo il diagramma commutativo
		\begin{equation*}
			\begin{tikzcd}
				{\left[ X^{(N)}, K(\pi,n) \right] } \ar[r, "\simeq", "\rho"'] \ar[d, "\simeq"]
				& {\widetilde{H}^{q}\left(X^{(N)};\pi\right)} \ar[d, "\simeq"] \\
				{\left[ X^{}, K(\pi,n) \right] } \ar[r, dashed]
				& \widetilde{H}^{q}(X;\pi)\,.
			\end{tikzcd}
		\end{equation*}
		\end{proof}
	\end{thm}
	
	\begin{ex}
		Dato che $K(\Z,1) = S^{1}$, allora $H^{1}(X;\Z) = [X,S^{1}]_{\ast}$.
	\end{ex}
	
	\begin{ex}
		Dato $X$ uno spazio topologico, vediamo che la sua coomologia
		\begin{equation*}
			H^{1}(X;\Z/2) = [X;\R\PP^{\infty}]_{\ast}\,.
		\end{equation*}
		Vedremo più avanti che una mappa $X \to \R\PP^{\infty}$
		induce un \emph{fibrato in rette}, e mappe omotope
		inducono fibrati isomorfi; deduciamo quindi che 
		$H^{1}(X;\Z/2)$ parametrizza le classi d'equivalenza di fibrati in rette su $X$.
	\end{ex}







	
%	%%% Lecture *

% \section{Torri di Whitehead}



\begin{df}
	Dato un oggetto $X$ in una categoria $\Cc$, il funtore covariante
	\begin{equation*}
		\Cc \longrightarrow \cat{Set}\,, \quad Y \longmapsto \Hom_{\Cc}(X,Y)
	\end{equation*}
	si dice \textbf{corappresentato} da $X$, mentre il funtore controvariante
	\begin{equation*}
		\Cc^{op} \longrightarrow \cat{Set}\,, \quad Y \longmapsto \Hom_{\Cc}(Y,X)
	\end{equation*}
	è \textbf{rappresentato} da $X$.
\end{df}

Dato un funtore $G:\Cc \longrightarrow \cat{Set}$,
un elemento $x \in G(X)$ determina una trasformazione naturale
\begin{equation*}
	\theta_{x}: \Hom_{\Cc}(X,-) \longrightarrow G\,,
	\quad \Big(f:X \to Y \Big) \longmapsto \theta_{x}f := G(f)(x) \in G(Y)\,.
\end{equation*}
Questo ci permette di enunciare uno dei teoremi più importanti
della teoria delle categorie:
\begin{thm}[Lemma di Yoneda]\label{yoneda-lemma}
	L'associazione $x \to \theta_{x}$ definisce una bigezione di insiemi
	\begin{equation*}
		G(X) \longrightarrow \cat{Nat}(\Hom_{\Cc}(X,-),G)
	\end{equation*}
	che è naturale sia in $X$, sia in $G$.
\end{thm}

\lecture[Copia argomenti da e-learning.]{2023-03-13}

In particolare, se $G$ è corappresentato da un oggetto $Y$,
allora abbiamo una bigezione
\begin{equation*}
	\cat{Nat}(\Hom_{\Cc}(X,-), \Hom_{\Cc}(Y,-)) \simeq \Hom_{\Cc}(Y,X)\,,
\end{equation*}
che significa che ogni trasformazione naturale $\Hom_{\Cc}(X,-) \to \Hom_{\Cc}(Y,-)$
è indotta da un'\emph{unica} mappa $Y \to X$.
In particolare, ogni isomorfismo naturale di funtori è indotto
da un isomorfismo di oggetti $Y \simeq X$.


\begin{fact}
Si potrebbe utilizzare un approccio ``inverso'',
dimostrando prima che la coomologia $H^{n}(-,\pi)$
è un \textbf{funtore rappresentabile} sui CW complessi,
e definire quindi gli spazi di Eilenberg-MacLane di tipo
$K(\pi,n)$ come gli oggetti che lo rappresentano,
cioè come quegli spazi topologici per cui vale
\begin{equation*}
	\Hom_{\cat{HoCW}_{\ast}}(X,K(\pi,n)) = [X,K(\pi,n)]_{\ast} = H^{n}(X;\pi)\,.
\end{equation*}
% e questo è proprio il contenuto del \hyperref[coh-rep]{Teorema~\ref{coh-rep}}.
Questo è espresso dal \textbf{Teorema di rappresentabilità di Brown}.
\end{fact}

Nel caso della coomologia, abbiamo visto che
\begin{equation*}
	\theta(m,n,A,B) := \cat{Nat}\Big(H^{m}(-;A), H^{n}(-;B) \Big)
	= [K(A,m),K(B,n)]_{\ast} = H^{n}(K(A,m);B)\,,
\end{equation*}
quindi queste trasformazioni naturali non sono più
così astratti, ma possiamo ``toccarli con mano'' nel caso in
cui sappiamo calcolare la coomologia di questi spazi di Eilenberg-MacLane.
Inoltre, questo procedimento ci suggerisce che
potremmo trovare un arricchimento della struttura
della coomologia, come ad esempio nuove \emph{operazioni}
naturali, che la classica struttura di anello sulla coomologia
non esprime.

\begin{ex}
	Dato un primo $p$, consideriamo la successione esatta corta
	\begin{equation*}
		\begin{tikzcd}
			0 \ar[r]
			& \Z/p \ar[r]
			& \Z/p^{2} \ar[r]
			& \Z/p \ar[r]
			& 0\,.
		\end{tikzcd}
	\end{equation*}
	Prendendo le cocatene singolari di uno spazio $X$,
	otteniamo una successione esatta di complessi
	\begin{equation*}
		\begin{tikzcd}
			0 \ar[r]
			& C^{\bullet}(X;\Z/p) \ar[r]
			& C^{\bullet}(X;\Z/p^{2}) \ar[r]
			& C^{\bullet}(X;\Z/p) \ar[r]
			& 0\,,
		\end{tikzcd}
	\end{equation*}
	la quale induce una successione esatta lunga in coomologia;
	l'omomorfismo di connessione $\beta:H^{n}(X;\Z/p) \to H^{n+1}(X;\Z/p)$
	è detto \textbf{omomorfismo di Bockstein}.
\end{ex}

Le trasformazioni naturali di $H^{*}(-,\Z/2)$
con l'operazione di composizione sono un'algebra 
(in generale \emph{non} commutativa) su $\Z/2$
che ci porterà a studiare le \textbf{operazioni di Steenrod}
per descrivere $H^{*}\big( K(\Z/2,n),\Z/2 \big)$.

\begin{df}
	Chiamiamo il gruppo $\Theta(m,n,A,B)$ \textbf{operazioni coomologiche}.
\end{df}

\begin{oss}
	Le operazioni coomologiche possono solo \emph{aumentare} il grado:\todo{recupera}
\end{oss}

\begin{oss}
	L'isomorfismo di sospensione in coomologia
	\begin{equation*}
		\widetilde{H}^{n}(X;G) \simeq \widetilde{H}^{n+1}(\Sigma X;G)
	\end{equation*}
	è rappresentato da un'equivalenza omotopica debole
		$K(\pi,n) \to \Omega K(\pi,n+1),$
	poiché usando l'aggiunzione tra sospensione e loop si ha
	\begin{equation*}
		 [X,K(\pi,n)]_{\ast} \simeq  [X,\Omega K(\pi,n+1)]_{\ast} \simeq  [\Sigma X,K(\pi,n)]_{\ast}\,.
	\end{equation*}
\end{oss}

\begin{df}
	Una successione di spazi $\dots, E_{0}, E_{1}, \dots$ e mappe
	\begin{equation*}
		E_{n} \longrightarrow \Omega E_{n+1}
		\quad (\text{oppure } \Sigma E_{n} \longrightarrow E_{n+1})
	\end{equation*}
	si chiama \textbf{spettro topologico};
	si dice \textbf{$\Omega$-spettro} se inoltre le mappe sono equivalenze omotopiche deboli.
\end{df}

Quanto visto per gli spazi di Eilenberg-MacLane per la coomologia singolare
si generalizza per \emph{teorie coomologiche} più generali.

\begin{df}
	Dato un' $\Omega$-spettro $\Set{E_{n}}$, i gruppi 
	\begin{equation*}
		\overline{E}^{n}(X) := [X,E_{n}]_{\ast}
	\end{equation*}
	sono \textbf{gruppi di una teoria} (\textbf{ridotta}) \textbf{coomologica generalizzata}
	se soddisfano i seguenti assiomi:
	\begin{itemize}
		\item \textbf{omotopia}: se $f,g:(X,A) \to (Y,B)$ sono mappe omotope, 
		allora inducono la stessa mappa $h^{n}(f) = h^{n}(g)$ tra i gruppi $\overline{E}^{n}(X)$;
		
		\item \textbf{esattezza}: una coppia $(X,A)$ induce una successione esatta
		lunga di gruppi $\overline{E}^{n}$;
		
		\item \textbf{escissione}: dato un aperto $U \subset X$ tale che 
		$\overline{U} \subset A \subset X$, allora c'è un isomorfismo
		\begin{equation*}
			\overline{E}^{n}(X,A) \simeq \overline{E}^{n}(X \setminus U, A \setminus U)\,;
		\end{equation*}
		\item \textbf{riduzione}: $\widetilde{h}(X) : \ker h(\ast \to X)$.\todo{Che è?}
	\end{itemize}
\end{df}


\section{Operazioni coomologiche stabili}

	\begin{df}
		Definiamo un'\textbf{operazione coomologica stabile} di tipo
		$(r,\pi,G)$ una successione di trasformazioni naturali 
		$\phi_{n} \in \Theta(n,n+r, \pi G)$, con $n > 0$, che
		commutano con l'isomorfismo di sospensione:
		esplicitamente, per ogni spazio $X$ e per ogni $n$,
		il seguente diagramma commuta:
		\begin{equation*}
			\begin{tikzcd}
				H^{n}(X;G) \ar[r, "\phi_{n}"] \ar[d, "\Sigma"]
				& H^{n+r}(X;\pi) \ar[d, "\Sigma"] \\
				H^{n+1}(\Sigma X;G) \ar[r, "\phi_{n+1}"] 
				& H^{n+r+1}(\Sigma X;\pi) \,. 
			\end{tikzcd}
		\end{equation*}
		Indichiamo con $\cat{Stab}\Theta(r,\pi,G)$ l'insieme delle
		operazioni stabili.
	\end{df}
	
	\begin{thm}
		Sia $(X,A)$ una coppia CW (o più in generale $A \hookrightarrow X$ una cofibrazione).
		Un'operazione coomologica stabile è compatibile con la successione esatta lunga della coppia,
		cioè per ogni $\phi \in \cat{Stab}\Theta(r,\pi,G)$ abbiamo
		il diagramma commutativo
%		\begin{equation*}
%			\begin{tikzcd}
%				H^{n}(X;\pi) \ar[r, "i^{*}"] \ar[d, "\phi_{n}"]
%				& 	H^{n}(A;\pi) \ar[r, "\delta^{*}"] \ar[d, "\phi_{n}"]
%				& H^{n+1}(X,A;\pi) \ar[d, "\phi_{n+1}"] \ar[r, "j^{*}"]
%				& H^{n+1}(X;\pi) \ar[d, "\phi_{n+1}"] \\
%				H^{n+r}(X;G) \ar[r, "i^{*}"]
%				& 	H^{n+r}(A;G) \ar[r, "\delta^{*}"]
%				& H^{n+1+r}(X,A;G)  \ar[r, "j^{*}"]
%				& H^{n+1+r}(X;G) \ar[d, "\phi_{n+1}"]
%			\end{tikzcd}
%		\end{equation*}			
		dove l'operazione $\phi$ è definita sulla coomologia relativa
		della coppia come $\phi_{X/A}$\footnote{Sappiamo che nel caso di una successione di cofibrazione la coomologia relativa è in realtà la coomologia del quoziente $X/A$.}
	\end{thm}
	
	\begin{oss}
		Si verifica che l'omomorfismo $\delta^{*}$ nel diagramma
		è la composizione
		\begin{equation*}
			\begin{tikzcd}
				H^{n}(A;\pi) \ar[r, "\Sigma"]
				& H^{n+1}(X;\pi)
			\end{tikzcd}
		\end{equation*}
		che è indotta in realtà dalla mappa di proiezione
			$p:X \cup CA \to (X \cup CA)/X \simeq \Sigma A$,
		e da questo segue la naturalità.
	\end{oss}

%	%%% Lecture 6


\lecture[Algebra delle operazioni coomologiche stabili. Definizione di algebra di Steenrod. Introduzione alle successioni spettrali: definizioni, filtrazioni, convergenza, modulo graduato associato ad una filtrazione, filtrazioni limitate, limite di una successione spettrale.]{2023-03-14}


Se prendiamo un'operazione $\phi \in \cat{Stab}\Theta(r,\pi,G)$, 
la pensiamo come una famiglia di trasformazioni naturali
$\phi_{n} \in \Theta(n,n+r,\pi,G) = H^{n+r}(K(\pi,n);G)$ che soddisfa la condizione di
stabilità $\Sigma \phi = \phi \Sigma$, ovvero gli omomorfismi
\begin{equation*}
	\begin{tikzcd}
		f_{n} \, : \, H^{n+r}(K(\pi,n);G) \ar[r, "i_{n+r}^{*}"]
		& H^{n+r}(\Sigma K(\pi,n-1);G) \ar[r, "\Sigma^{-1}"]
		& H^{n+r-1}(K(\pi,n-1);G)\,,
	\end{tikzcd}
\end{equation*}
soddisfano la condizione di stabilità $f_{n}(\phi_{n}) = \phi_{n-1}$,
dove il primo omomorfismo è indotto da $\cat{1}_{\pi}$
attraverso la seguente identificazione:
\begin{equation*}
	\begin{tikzcd}
		\Hom_{\cat{Grp}}(\pi,\pi) \ar[r, equals] \ar[d, equals]
		& H^{n-1}(K(\pi,n-1);\pi) \ar[d, "\Sigma", "\simeq"'] \\
		H^{n}(K(\pi,n);\pi) \ar[r,"i^{*}_{n}"]
		& H^{n}(\Sigma K(\pi,n-1);\pi)\,;
	\end{tikzcd}
\end{equation*}
ricordiamo che tramite l'identificazione
$H^{n-1}(K(\pi,n-1);\pi) \simeq [K(\pi,n-1),K(\pi,n)]_{\ast}$,
questa classe in coomologia induce un'unica mappa
$i_{n}:\Sigma K(\pi,n-1) \to K(\pi,n)$ a meno di omomotopia.

\begin{oss}
	La mappa $i_{n}$ è l'aggiunta di $K(\pi,n-1) \to \Omega K(\pi,n)$,
	che ricordiamo essere un'equivalenza omotopica debole.
\end{oss}

Quindi le operazioni stabili si possono vedere come il limite inverso
\begin{equation*}
	\cat{Stab}\Theta(r,\pi,G) = \varprojlim_{n}\left( H^{n+r}(K(\pi,n);G),f_{n}\right)\,,
\end{equation*}
e nel caso in cui $G$ è un anello, $H^{*}(K(\pi,n);G))$ è un anello
che \emph{non} induce la moltiplicazione in $\cat{Stab}\Theta(r,\pi,G)$.
Tuttavia, la composizione di mappe rende  $\bigoplus_{r \ge 0}\cat{Stab}\Theta(r,\pi,G)$
un \textbf{anello graduato}, unitario e in generale \emph{non} commutativo.
La coomologia $H^{*}(X;G)$ è un modulo su tale anello e una
qualsiasi funzione continua $f:X \to Y$ induce un omomorfismo di moduli.
Inoltre, se $G=k$ è un campo, allora le operazioni coomologiche diventano
una $k$-algebra, nel caso in cui $G=\Z/p$, con $p$ primo,
viene chiamata \textbf{algebra di Steenrod} $\AA_{p}$.


\chapter{Successioni spettrali}

Vogliamo studiare la (co)omologia di un complesso di (co)catene 
``affettandolo'' attraverso delle \emph{filtrazioni},
le quali ci permetteranno di studiare il complesso come se fosse un \emph{libro}:
infatti, otterremo tanti bicomplessi chiamati \emph{pagine},
ognuna delle pagine produrrà un nuovo complesso,
ciascuno con un nuovo differenziale che ci sposta ``in su'' di una diagonale.


%%%%%%%%%%%%%%%%%%%%%%%%%%%%%%%%%%%%%%%%%%%%%%%%%%%%%%%%%%%%%%%%%%%%%%%
%%%%%%%%%%%%%%%%%%%%%%%%%%%%%%%%%%%%%%%%%%%%%%%%%%%%%%%%%%%%%%%%%%%%%%%
%
%\begin{sseqdata}[ name = example,
%Adams grading,
%yscale = 0.53 ]
%\class["E_{0,2}"](0,2) \class["E_{1,2}"}(1,2) \class["E_{2,2}"}(2,2) 
%\class["E_{0,1}"](0,1) \class["E_{1,0}"}(1,1) \class["E_{2,1}"}(2,1) 
%\class["E_{0,0}"](0,0) \class["E_{1,0}"}(1,0) \class["E_{2,0}"}(2,0) 
%\d["d"]2(1,0)
%\end{sseqdata}
%\begin{equation*}
%	\printpage[ name = example, page = 2 ]
%\end{equation*}
%
%%%%%%%%%%%%%%%%%%%%%%%%%%%%%%%%%%%%%%%%%%%%%%%%%%%%%%%%%%%%%%%%%%%%%%%
%%%%%%%%%%%%%%%%%%%%%%%%%%%%%%%%%%%%%%%%%%%%%%%%%%%%%%%%%%%%%%%%%%%%%%%

Se la situazione è favorevole, i differenziali produrranno definitivamente
(co)omologia banale, e la successione delle pagine si \emph{stabilizza}.


Sia $(C_{\bullet}, d)$ un complesso di catene.
Ricordiamo che la sua omologia in grado $i$ è il quoziente 
$H_{i}(C_{\bullet}) = \ker d_{i}/\mathrm{im} d_{i+1}$.
In presenza di un complesso di cocatene $(C^{\bullet},d)$
usiamo invece la notazione \emph{crescente},
e quindi la sua coomologia in grado $i$ è 
$H^{i}(C^{\bullet}) = \ker d^{i}/\mathrm{im} d^{i-1}$.

\begin{df}
	Dato un $R$-modulo $A$, una famiglia di sottomoduli 
	totalmente ordinata per inclusione si chiama
	\begin{itemize}
		\item una \textbf{filtrazione decrescente} se 
			\begin{equation*}
				F^{\bullet}A : \quad
				\dots \subset F^{p+1}A \subset F^{p}A \subset F^{p-1}A \subset \dots
			\end{equation*}
			
		\item una \textbf{filtrazione crescente} se 
			\begin{equation*}
				F_{\bullet}A : \quad
				\dots \subset F_{p-1}A \subset F_{p}A \subset F_{p+1}A \subset \dots
			\end{equation*}
	\end{itemize}
\end{df}


\begin{ex}
	Una filtrazione descrescente $F^{\bullet}$ sul gruppo abeliano $A = \Z$ è data da
	$F^{i}A = \Z$ per $i \le 0$ e $F^{i}A = 2^{i}\Z$ per $i > 0$:
	\begin{equation*}
		0 \subset \dots \subset 16\Z \subset 8\Z \subset 4\Z \subset 2\Z \subset \Z\,.
	\end{equation*}
\end{ex}


\begin{df}
	Diciamo che un \textbf{complesso} di catene $(C_{\bullet}, d)$ è \textbf{filtrato}
	se è un $R$-modulo $C = \bigoplus_{j \in \Z} C_{j}$ filtrato $F_{\bullet}$ tale
	che il differenziale sia compatibile rispetto alla filtrazione: 
	nel caso di una filtrazione $F^{\bullet}$ decrescente,
	richiediamo che per ogni $i \in \Z$ valga
	 $d\left(F^{i}C\right) \subset F^{i}C$.
	 In maniera analoga, possiamo definire un complesso di cocatene filtrato,
	 dotato di una filtrazione crescente $F_{\bullet}$, 
	 oppure decrescente $F^{\bullet}$.
\end{df}

	Notiamo che, se $(C_{\bullet},d)$ è un complesso filtrato da $F_{\bullet}$,
	allora i sottomoduli $F_{p}C_{\bullet}=\bigoplus_{j \in \Z}F_{p}C_{j}$
	ereditano una naturale struttura di complesso di catene
	una volta che restringiamo il differenziale:
	\begin{equation*}
		d\vert_{F_{p}C_{\bullet}} : F_{p}C_{j} \longmapsto F_{p}C_{j-1}\,.
	\end{equation*}

\begin{df}
	Dato $(A,F)$ un $R$-modulo filtrato, il suo \textbf{modulo graduato associato} è
	\begin{itemize}
		\item l'$R$-modulo $E^{0}(A) = \bigoplus_{p \in \Z} E^{0}_{p}(A)$, 
		dove $E^{0}_{p}(A) := F_{p}A/F_{p-1}A$ se $F_{\bullet}$ è crescente;
		
		\item l'$R$-modulo $E_{0}(A) = \bigoplus_{p \in \Z} E_{0}^{p}(A)$, 
		dove $E^{p}_{0}(A) := F^{p}A/F^{p+1}A$ se $F^{\bullet}$ è decrescente.
	\end{itemize}
\end{df}

\begin{ex}
	Consideriamo la filtrazione $0 \subset 2\Z \subset \Z$ sul gruppo abeliano $\Z$.
	Lo $\Z$-modulo graduato associato alla filtrazione è $E^{0}(\Z)=2\Z \oplus \Z/2$;
	si osservi che non è isomorfo a $\Z$.
\end{ex}

\begin{oss}
	In generale, anche se la filtrazione $F^{\bullet}$ è \emph{limitata}, cioè
	\begin{equation*}
		0 = F^{m+n+1}A \subsetneq F^{m+n}A \subset \dots \subset F^{m}A = A\,,
	\end{equation*}
	il modulo $E^{p}_{0}(A)$ non determina il modulo $A$: 
	questo fatto è conosciuto come il \textbf{problema di estensione}.
	Infatti, se consideriamo grado per grado $F^{n+m}A \simeq E^{n+m}_{0}(A)$
	e le successioni esatte corte
	\begin{equation*}
		\begin{tikzcd}
			0 \ar[r]
			& F^{i+1}A \ar[r]
			& F^{i}A \ar[r]
			& E^{i}_{0}(A) \ar[r]
			& 0\,,
		\end{tikzcd}
	\end{equation*}
	in generale il termine $F^{i}A$ \emph{non} è univocamente determinato:
	si pensi ad esempio
	\begin{equation*}
		\begin{tikzcd}
			0 \ar[r] & \Z/2 \ar[r] & A \ar[r] & \Z/2 \ar[r] & 0\,,
		\end{tikzcd}
	\end{equation*}
	dove $A$ può essere $\Z/4$ oppure $\Z/2 \oplus \Z/2$.
\end{oss}


\begin{ex}
	Sia $X$ un CW complesso e indichiamo con $X^{(p)}$ il suo $p$-scheletro.
	Il complesso $C_{\bullet}(X)$ delle catene singolari di $X$ ammette 
	la filtrazione crescente $F_{p}C_{\bullet} := C_{\bullet}\left( X^{(p)} \right)$.
	Passando alle cocatene, invece, il complesso $C^{\bullet}(X) = \Hom_{R}(C_{\bullet}(X),R)$
	ammette la filtrazione decrescente
	\begin{equation*}
		F^{p}C^{\bullet} := \Set{\phi \in C^{\bullet}(X) | F_{p-1}C_{\bullet} \subset \ker \phi}
		= \mathrm{Ann}\left( F_{p-1}C_{\bullet} \right)\,.
	\end{equation*}
	I moduli graduati associati corrispondono alle (co)catene relative agli scheletri di
	dimensioni successive: più precisamente, abbiamo
	\begin{align*}
		E^{0}_{p}\left(C_{\bullet}\right) = C_{\bullet}\left( X^{(p)}, X^{(p-1)} \right)\,,
		\quad E_{0}^{p}\left(C^{\bullet}\right) = C^{\bullet}\left( X^{(p)}, X^{(p-1)} \right)\,.
	\end{align*}
\end{ex}

\begin{df}
	Data una mappa di complessi $f:D_{\bullet} \to C_{\bullet}$
	e una filtrazione $F_{\bullet}$ su $C_{\bullet}$,
	allora definiamo una filtrazione $G_{\bullet}$
	indotta sul modulo $D_{\bullet}$ da
	\begin{equation*}
		G_{p}D_{\bullet} = f^{-1}\left(F_{p}C_{\bullet}\right)\,.
	\end{equation*}
\end{df}

\begin{oss}
	Se $(C_{\bullet}, d)$ è un complesso di catene con una filtrazione 
	crescente $F_{\bullet}$,
	allora anche l'omologia $H_{*}(C_{\bullet})$ è un modulo filtrato,
	dove
	\begin{equation*}
		F_{p}H_{*}(C_{\bullet}) := 
		\mathrm{im}\left( H_{*}(F_{p}C_{\bullet}) \to H_{*}(C_{\bullet}) \right)
	\end{equation*}
	determina una filtrazione crescente su $H_{*}$.
	In maniera duale, dato un complesso di cocatene 
	 $(C^{\bullet}, d)$ con una filtrazione $F$ decrescente,
	allora anche la sua coomologia $H^{*}(C^{\bullet})$ 
	è un modulo filtrato in maniera naturale,
	dove la filtrazione decrescente è data da
	\begin{equation*}
		F^{p}H^{*}(C^{\bullet}) := 
		\ker \left( H^{*}(C^{\bullet}) \to H^{*}(F^{p-1}C^{\bullet}) \right)\,.
	\end{equation*}
\end{oss}

\begin{df}
	Un \textbf{modulo bigraduato} $E_{\bullet,\bullet}$ 
	(indicato con $E^{\bullet,\bullet}$ se usiamo la notazione coomologica)
	è una somma diretta di $R$-moduli,
	i cui gradi sono definiti da due interi $(s,t) \in \Z^{2}$.
	Un tale modulo $E$ si dice \textbf{differenziale} 
	se c'è una mappa $d:E \to E$, tale che $d^{2}=0$ e abbia:
		\begin{itemize}
		\item \textbf{bigrado $(-r,r-1)$}, cioè il differenziale
		in grado $(s,t)$ è $d_{s,t}:E_{s,t} \to E_{s-r,t+r-1}$;
		
		\item oppure \textbf{bigrado $(r,-r+1)$},
		cioè il differenziale
		in grado $(s,t)$ è $d^{s,t}:E^{s,t} \to E^{s+r,t-r+1}$.
		\end{itemize}
	Nel primo caso il differenziale ci fa ``scendere'' di diagonale nel bimodulo,
	nel secondo caso ``saliamo'' sulla diagonale superiore,
	come rappresentato in figura.
	\missingfigure{Paginetta}
\end{df}


\begin{oss}
	Un modulo bigraduato $E_{\bullet,\bullet}$ dà origine ad un complesso di catene
	\begin{equation*}
		K_{n} := \bigoplus_{s+t=n} E_{s+t}\,,
	\end{equation*}
	e in maniera duale, se il modulo ha la convenzione coomologica $E^{\bullet, \bullet}$,
	allora dà origine a un complesso di cocatene $K^{\bullet}$.
\end{oss}

Dato un modulo bigraduato $(E_{\bullet,\bullet},d)$, 
dato che $d^{2}=0$, possiamo calcolare la sua omologia $(p,q)$-esima 
come il quoziente
\begin{equation*}
	H_{p,q}(E_{\bullet,\bullet},d) := \dfrac{\ker\left( d : E_{p,q} \to E_{p-r,q+r-1} \right)}{\mathrm{im}\left( d: E_{p+r,q-r+1} \to E_{p,q} \right)}\,,
\end{equation*}
e la stessa cosa accade per la coomologia di $E^{\bullet,\bullet}$. 

\begin{df}
	Una \textbf{successione spettrale} di tipo \textbf{omologico}
	è una collezione di $R$-moduli differenziali bigraduati 
	$\Set{ \left(E^{r}_{\bullet,\bullet},d^{r}\right)}_{r \ge k}$,
	con $d^{r}$ di bigrado $(-r,r-1)$ tale che 
	\begin{equation*}
		E^{r+1}_{p,q} \simeq H_{p,q} \left( E_{\bullet,\bullet}^{r}, d^{r} \right)\,.
	\end{equation*}
	Analogamente, una \textbf{successione spettrale} di tipo \textbf{coomologico}
	è una collezione di $R$-moduli differenziali bigraduati 
	$\Set{ \left(E_{r}^{\bullet,\bullet},d_{r}\right)}_{r \ge k}$,
	con $d_{r}$ di bigrado $(r,-r+1)$ tale che 
	\begin{equation*}
		E_{r+1}^{p,q} \simeq H^{p,q} \left( E^{\bullet,\bullet}_{r}, d_{r} \right)\,.
	\end{equation*}
\end{df}

D'ora in avanti enunceremo definizioni e teoremi per uno solo dei due tipi
(per lo più il caso coomologico), ma è bene tenere a mente che
le costruzioni calgono in maniera analoga per entrambe le situazioni.

\begin{oss}
	Per definizione, la pagina $r$-esima $(E_{r}^{\bullet,\bullet},d_{r})$
	determina $E_{r+1}^{p,q}$, ma in generale
	\emph{non} determina il differenziale $d_{r+1}$ della pagina successiva.
\end{oss}

Una successione spettrale va pensata come una successione di pagine,
proprio come in un libro; in quanto tale, vorremmo conoscere
il suo comportamento ``alla fine'', ovvero studiare il suo \emph{limite}.
Per ogni $r \ge k$, indichiamo con $Z_{r}=\ker d_{r}$ i \textbf{cocicli} 
e con $B_{r}=\mathrm{im}d_{r}$ i \textbf{cobordi} del differenziale;
dato che $E_{r+1}=Z_{r}/B_{r}$, possiamo vedere il differenziale $(r+1)$-esimo
come un omomorfismo $d_{r+1} : Z_{r}/B_{r} \to Z_{r}/B_{r}$ e quindi 
considerare i suoi cocicli $Z_{r+1}$ come un (\emph{quoziente di un})
sottomodulo di $Z_{r}$ che contiene i cobordi $r$-esimi,
e lo stesso vale per $B_{r+1}$, quindi
\begin{equation*}
	B_{r} \subset B_{r+1} \subset Z_{r+1} \subset Z_{r}\,. 
\end{equation*}
Reiterando questo ragionamento per tutte le pagine, si ottiene la catena di inclusioni
\begin{equation*}
	B_{k} \subset B_{k+1} \subset B_{k+2} \subset \dots \subset Z_{k+2} \subset Z_{k+1} \subset Z_{k}\,.
\end{equation*}

\begin{df}
	Usando la convenzione sopra,
	poniamo $B_{\infty} := \cup_{i} B_{i}$ e $Z_{\infty} := \cap_{i} Z_{i}$.
	Definiamo il \textbf{limite della successione spettrale} come il quoziente
	$$E_{\infty} := Z_{\infty}/B_{\infty}\,.$$
\end{df}

Dalla definizione schietta è pressoché impossibile
calcolare il limite di una successione spettrale:
di solito speriamo che le cose vadano particolarmente bene
e quindi riuscire a aggirare questo problema di calcolo.
Introduciamo così una nozione di \emph{convergenza}.

\begin{df}
	Una successione spettrale $\Set{(E_{r}^{\bullet,\bullet},d_{r})}_{r \ge k}$
	si dice \textbf{convergente a $H^{*}$} se esiste 
	una filtrazione decrescente $F^{\bullet}$ su $H^{*}$ tale che
	\begin{equation*}
		E_{\infty}^{p,q} \simeq E_{0}^{p}(H^{p+q})\,.
	\end{equation*}
\end{df}

Sotto opportune ipotesi,
possiamo trovare delle situazioni dove il limite è più ``addomesticabile'',
ad esempio quando ogni bicomplesso $E^{\bullet,\bullet}_{r}$ vive nel primo quadrante,
cioè $E^{p,q}_{r} \ne 0$ solo per $p,q \ge 0$.
Allora i differenziali $d_{r}$ sono \emph{definitivamente nulli},
quindi la successione spettrale converge in un senso più forte:
si dice che la successione \textbf{stabilizza} se, per ogni $(p,q)$,
esiste un intero $r(p,q)$ tale che,
i differenziali $d_{r}^{p,q}=0$ e $d^{p-r,q+r-1}_{r}=0$ per ogni $r \ge r(p,q)$.
Da questo segue che $E_{r+1}^{p,q}=E_{r}^{p,q}$, da cui deduciamo che
se la successione in $(p,q)$ stabilizza, allora
\begin{equation*}
	E^{p,q}_{r} = E^{p,q}_{r+1} = E^{p,q}_{r+2} = \dots = E^{p,q}_{\infty}\,.
\end{equation*}

\begin{df}
	Diremo che una successione spettrale \textbf{collassa} alla pagina $N$
	se $d_{r}=0$ per ogni $r \ge N$.
\end{df}

\begin{df}
	Una filtrazione si dice \textbf{convergente} se $\cup_{s} F_{s}A = A$
	e $\cap_{s} F_{s}A = \{0\}$.
	Una filtrazione su un $R$-modulo graduato $A$ si dice 
	\begin{itemize}
		\item  \textbf{limitata dall'alto} se per ogni grado $t \in \Z$, 
		esiste un termine della filtrazione $s(t) \in \Z$ tale che $F_{s(t)}A_{t}=A_{t}$;
		\item  \textbf{limitata dal basso} se per ogni $t \in \Z$, esiste $u(t) \in \Z$
		tale che $F_{u(t)}A_{t} = 0$.
	\end{itemize}
\end{df}

%\begin{thm}
%	Un complesso di catene graduato e filtrato $(A,d,F)$ determina una successione spettrale
%	$\Set{(E_{\bullet,\bullet}^{r},d^{r})}$ tale che
%	$E_{s,t}^{1} = H_{s+t}(F_{s}A/F_{s-1}A)$ e il differenziale $d^{1}$ è
%	l'operatore di bordo della tripla $(F_{s}A,F_{s-1}A,F_{s-2})$.
%	Se la filtrazione $F_{\bullet}$ è convergente e limitata sia dal basso, sia dall'alto,
%	allora la successione spettrale stabilizza e converge al limite $E^{\infty}_{p,q}$
%	che è isomorfo al quoziente $F_{p}H_{p+q}(A)/F_{p-1}H_{p+q}(A)$.
%\end{thm}
%
%
%
%
%
%
%
%



\nocite{*}


\backmatter\KOMAoption{chapterprefix}{false}
\printbibliography[heading=bibintoc, title={References}]
\end{document}